%%% Hlavní soubor. Zde se definují základní parametry a odkazuje se na ostatní části. %%%
\RequirePackage{fix-cm}
%% Verze pro jednostranný tisk:
% Okraje: levý 40mm, pravý 25mm, horní a dolní 25mm
% (ale pozor, LaTeX si sám přidává 1in)
\documentclass[12pt,a4paper,final]{report}
\setlength\textwidth{145mm}
\setlength\textheight{247mm}
\setlength\oddsidemargin{15mm}
\setlength\evensidemargin{15mm}
\setlength\topmargin{0mm}
\setlength\headsep{0mm}
\setlength\headheight{0mm}
% \openright zařídí, aby následující text začínal na pravé straně knihy
\let\openright=\clearpage


%% Generate PDF/A-2u
\usepackage[a-2u]{pdfx}


%\usepackage{fixltx2e}   % fixes various  bugs in pre 2015 LaTeX 
\usepackage[utf8]{inputenc}
\usepackage[T1]{fontenc}

%% Prefer Latin Modern fontspdfx
\usepackage{lmodern}

\usepackage{microtype}  % microtypographical enhancements
\usepackage[english]{babel}
\usepackage{amsmath, amssymb, amsthm, amsfonts}
\usepackage{graphicx}
\usepackage{bm}             % boldface symbols (\bm)
\usepackage{pdfpages} % for inserting pdfs of published articles
\usepackage{twoopt} % for convenient commands
\usepackage{xfrac} % for \sfrac command
\usepackage{mathrsfs}   % calligraphic font \mathscr{C}
\usepackage{minted} % for pretty printing of source code

\usepackage[nottoc]{tocbibind} % makes sure that bibliography and the lists
			    % of figures/tables are included in the table
			    % of contents
\usepackage{dcolumn}        % improved alignment of table columns
\usepackage{booktabs}       % improved horizontal lines in tables
\usepackage{paralist}       % improved enumerate and itemize
\usepackage[justification=centering]{caption} % center multiline captions or e.g. [format=hang] 

%\usepackage{natbib}         % citation style AUTHOR (YEAR), or AUTHOR [NUMBER]
%\usepackage[style=alphabetic, backend=biber]{biblatex}
\usepackage[style=iso-alphabetic, backend=biber]{biblatex} 
\addbibresource{thesis.bib}

\usepackage{float}
\usepackage[toc]{appendix}
\usepackage[matrix,arrow]{xy}    % for diagrams
\usepackage{euscript}

\usepackage{tikz}
\usetikzlibrary{positioning,decorations.markings}
\tikzset{croot/.style={circle,draw,fill=black,inner sep=0pt,minimum size=2mm},
	 nroot/.style={circle,draw,inner sep=0pt,minimum size=2mm},
	 node distance = 1
	 }


\usepackage{threeparttable,array}
\newcolumntype{C}{>{$\displaystyle}c <{$}}

\usepackage{csquotes}

%%% Drobné úpravy stylu

% Tato makra přesvědčují mírně ošklivým trikem LaTeX, aby hlavičky kapitol
% sázel příčetněji a nevynechával nad nimi spoustu místa. Směle ignorujte.
\makeatletter
\def\@makechapterhead#1{
  {\parindent \z@ \raggedright \normalfont
   \Huge\bfseries \thechapter. #1
   \par\nobreak
   \vskip 20\p@
}}
\def\@makeschapterhead#1{
  {\parindent \z@ \raggedright \normalfont
   \Huge\bfseries #1
   \par\nobreak
   \vskip 20\p@
}}
\makeatother

% This macro defines a chapter, which is not numbered, but is included
% in the table of contents.
\def\chapwithtoc#1{
\chapter*{#1}
\addcontentsline{toc}{chapter}{#1}
}

\newcommand{\inserttikzfigure}[2]{
											\begin{center}
												\begin{figure}[h]
													\centering 
													\input{#1}
													\caption{#2}
												\end{figure} 
											\end{center}
										}										
%%%%%%%%%%%%%%%%%%%%%%%%%%%%%%%% THEOREMS %%%%%%%%%%%%%%%%%%%%%%%%%%%%%%%%%%
\newtheorem{theorem}{Theorem}[section]
\newtheorem{remark}[theorem]{Remark}
\newtheorem{proposition}[theorem]{Proposition}
\newtheorem{corollary}[theorem]{Corollary}
\newtheorem{definition}[theorem]{Definition}
\newtheorem{lemma}[theorem]{Lemma}
\newtheorem{notation}[theorem]{Notation}
\newtheorem{example}[theorem]{Example}

%%%%%%%% CMUC %%%%%%%%%%

%%%%%%%%%%%%%%%%%%%%%%%%%%%%%%%% SYMBOLS %%%%%%%%%%%%%%%%%%%%%%%%%%%%%%%%%%
\newcommand{\et}{\,\&\,}
\renewcommand{\k}{\Bbbk}
\renewcommand{\C}{\mathbb{C}}
\newcommand{\octo}{\mathbb{O}}
\newcommand{\R}{\mathbb{R}}
\newcommand{\N}{\mathbb{N}}
\newcommand{\Z}{\mathbb{Z}}
\newcommand{\lie}[1]{\mathfrak{#1}}
\newcommand{\rep}[1]{\mathbb{#1}}
\newcommand{\bun}[1]{\mathcal{#1}}
\newcommand{\oppar}{\overline{\lie{p}}}
\newcommand{\Hom}{\mathrm{Hom}}
\newcommand{\pproj}[1]{\Pi_{#1}^\lie{g}}
\newcommand{\floor}[1]{\left[ #1 \right]}

\newcommand{\roots}{\Phi} % the set of roots
\newcommand{\sroots}{\Delta} % the set of positive simple roots
\newcommand{\weights}{\Lambda} % weight lattice

\renewcommand{\a}{\alpha}
%%%%%%%%%%%%%%%%%%%%%%%%%%%%%%%% OPERATORS %%%%%%%%%%%%%%%%%%%%%%%%%%%%%%%%%%

%\newcommand{\im}{\mathrm{im}\,}
%\newcommand{\lap}{\square}
%\newcommand{\td}{\mathrm{d}^\lie{g}}
%\newcommand{\id}{\mathrm{Id}\,}
%\newcommand{\plap}{\lap_\lie{g}}
%\newcommand{\lcd}{\partial\,}
% \renewcommand{\lhd}{\mathop{\partial^*}}
% \renewcommand{\lhd}{\mathop{\delta}}
 \renewcommand{\lhd}{\operatorname{\delta}}

\newcommand{\bigo}{\mathcal{O}}

\newcommand{\conj}[1]{\overline{#1}}
\newcommand{\jscal}[2]{\langle #1\,|\, #2 \rangle}
\newcommand{\oscal}[2]{\left( #1 \,|\, #2 \right)}
 

% \DeclareMathOperator{\hom}
\DeclareMathOperator{\proj}{\mathrm{proj}}
\DeclareMathOperator{\im}{\mathrm{im}}
\DeclareMathOperator{\lap}{\square}
\DeclareMathOperator{\id}{\mathrm{Id}}
\DeclareMathOperator{\td}{\mathrm{d}^\lie{g}}
\DeclareMathOperator{\plap}{\lap_\lie{g}}
\DeclareMathOperator{\lcd}{\partial}
% \DeclareMathOperator{\lhd}{\partial^*}
\DeclareMathOperator{\tr}{\mathrm{Tr}}
\DeclareMathOperator{\ad}{\mathrm{ad}}
\DeclareMathOperator{\Ad}{\mathrm{Ad}}
\DeclareMathOperator{\Id}{\mathrm{Id}}

%%%%%%%%%%%%%%%%%%%%%%%%%%%%%%%% ENVIRONMENTS %%%%%%%%%%%%%%%%%%%%%%%%%%%%%%%%%%

\newenvironment{smatrix}{\left(\begin{smallmatrix}}{\end{smallmatrix}\right)}

%%%%%%%%%%%%%%%%%%%%%%%%%% CD %%%%%%%%%%%%%%%%%%%%%%%%%%%%%%%%%%%%%%%%%%%%%%%%%%%%

\newcommand{\dsum}{\oplus}               % small direct sum
\newcommand{\Dsum}{\bigoplus}            % big direct sum
\newcommand{\tens}{\mathbin{\otimes}}    % small tensor product
\newcommand{\Tens}{\bigotimes}           % big tensor product
\newcommand{\cartan}{\mathbin{\odot}}    % small Cartan product
\newcommand{\Cartan}{\bigodot}           % big Cartan product
\newcommand{\subideal}{\ltimes}          % semidirect product for algebras
\newcommand{\idealsub}{\rtimes}          %  with ideal on right or left
\newcommand{\subnormal}{\ltimes}         % semidirect product for groups with
\newcommand{\normalsub}{\rtimes}         %  normal subgroup on right or left
\newcommand{\intersect}{\mathinner{\cap}}% intersection
\newcommand{\setdif}{\mathbin{\mathrm -}}% or {\smallsetminus}
\newcommand{\skwend}{\mathinner{\vartriangle}} % skew endomorphism
%
\newcommand{\act}{\mathinner{\cdot}}
\newcommand{\dual}{^{*\!}}
\newcommand{\Cinf}{\mathrm{C}^\infty}
%
\newcommand{\oper}[3][n]{\newcommand{#2}{\mathop{\mathrm{#3}}\ifx
  n#1\nolimits\else\limits\fi}}
\newcommand{\rsoper}[3][n]{\newcommand{#2}{\mathop{\mathrmsl{#3\mkern1mu}}\ifx
  n#1\nolimits\else\limits\fi}}
\newcommand{\symb}[2]{\newcommand{#1}{{\mathit{#2}}}}
\newcommand{\rssymb}[2]{\newcommand{#1}{{\mathrmsl{#2\mkern1mu}}}}
\newcommand{\calsymb}[2]{\newcommand{#1}{{\mathcal{#2}}}}
\newcommand{\bbsymb}[2]{\newcommand{#1}{{\mathbb{#2}}}}

\rsoper\kernel{ker}
\rsoper\image{im}
% \rsoper\alt{alt}           % alternating part
\rsoper\sym{sym}           % symmetric part
\rsoper\trace{tr}          % trace of an endomorphism
\rsoper\detm{det}          % determinant
\rsoper\divg{div}
\rsoper\Sdivg{Sdiv}
\rsoper\Twist{Twist}
\rsoper\project{proj}
\rsoper\represent{repr}
\rssymb\ev{ev}
\rssymb\iden{id}
%
\newcommand{\cross}{\mathbin{{\times}\!}\low}
\newcommand{\from}{\colon}
\newcommand{\isom}{\cong}
\newcommand{\hQuabla}{\hat{\pmb\square}}
\newcommand{\Quabla}{\pmb{\square}}
\newcommand{\Lie}{{\mathcal L}}
%
\newcommand{\conf}{\mathsf{c}}
\newcommand{\cip}{\ip}
\newcommand{\cform}{\eta}
\newcommand{\g}{\lie{g}}
\newcommand{\p}{\lie{p}}
\newcommand{\Lieb}[1]{[#1]}
\bbsymb\E{E}\bbsymb\W{W}
\newcommand{\T}{\lie{m}}
\newcommand{\inner}{\mathbin{\lrcorner}}
\newcommand{\capinner}{\mathbin{\lower3pt\hbox{$\urcorner$}}}
\newcommand{\low}{^{\vphantom x}}
\newcommand{\dT}{d\low_{\T}}
\newcommand{\dTd}{d\low_{\T^*}}
\newcommand{\delT}{\delta\low_{\T}}
\newcommand{\delTd}{\delta\low_{\T^*}}
\newcommand{\delTdM}{\delta\low_{T\dual M}}
\newcommand{\gM}{\g\low_M}
\newcommand{\InvDer}{\nabla^\cform}
\newcommand{\TwConn}{\nabla^\g}
\newcommand{\deltw}{\delta^\g}
\newcommand{\dtw}{d^\g}
\newcommand{\Rtw}{R^\g}
\newcommand{\QuaTw}{\Quabla_\g}
\newcommand{\Qtw}{Q}
\newcommand{\PiTw}{\Pi}
\calsymb\cD{D}\calsymb\cG{G}\calsymb\cH{H}\calsymb\cK{K}\calsymb\cL{L}

\newcommand{\OpTw}{\cD}
\newcommand{\cptw}{\mathbin{\scriptstyle\sqcup}}
\newcommand{\delinv}{\delta^\cform}
\newcommand{\dinv}{d^\cform}
\newcommand{\Rinv}{R^\cform}
\newcommand{\hQuaInv}{\hQuabla_\cform}
\newcommand{\QuaInv}{\Quabla_\cform}
\newcommand{\Qinv}{Q_\cform}
\newcommand{\PiInv}{\Pi^\cform}
\newcommand{\OpInv}{\cD^\cform}
\newcommand{\Opinv}{\cD_\cform}
\newcommand{\cpinv}{\cptw\low_\cform}
\newcommand{\TwIso}{\Psi_\W}
\newcommand{\captw}{\mathbin{\scriptstyle\sqcap}}
\newcommand{\capinv}{\captw\low_\cform}
\newcommand{\Dirac}{{\smash{\raise1pt\hbox{$\not$}}\mkern-1.5mu D}}

%%%%%%%%%%%%% F4/Spin9
\newcommand{\field}{\Bbbk}
\newcommand{\reals}{\mathbb{R}}
\newcommand{\complex}{\mathbb{C}}
\newcommandtwoopt{\projplane}[2][2][P]{\mathbb{O}{#2}^{#1}}
\newcommandtwoopt{\jalg}[2][G][\octo]{J_{#1}(#2)}

\newcommand{\set}[1]{\left\lbrace #1\right\rbrace}
\renewcommand{\mid}{\middle \mid}
\newcommand{\sform}[1]{\mathrm{II}(#1)}

\newenvironment{psmatrix}
  {\left(\begin{smallmatrix}}
  {\end{smallmatrix}\right)}
	
	%%%%%%%%%%%%%%%%%% G2/P1
	
	\newcommand{\uenv}[1]{\mathfrak{U(#1)}}


\newcommand{\pd}[1]{\partial_{#1}}
\newcommand{\euler}{\mathbb{E}}
\newcommand{\wdeg}{\mathrm{wdeg}\,}
\newcommand{\actpol}{\widehat{\pi}_{\lambda}}

\newcommand{\cas}{\mathrm{Cas}_\mathbb{V}\,}
\newcommand{\vsingtop}{v_\mathrm{sing}^\mathrm{top}}

%%%% 1-graded 

\newcommand{\SL}{\mathrm{SL}}
\def\diag{\mathop {\rm diag} \nolimits}
\let\veps=\varepsilon
%\newcommand*\widebar[1]{\@ifnextchar^{{\wide@bar{#1}{0}}}{\wide@bar{#1}{1}}}
\newcommand{\widebar}[1]{\overline{#1}}
\def\htt{\mathop {\rm ht} \nolimits}
\newcommand{\eus}{\EuScript}
%\let\eus=\EuScript
\let\rarr=\rightarrow
\let\mfrak=\mathfrak
\newcommand{\mcal}{\mathcal}
\def\End{\mathop {\rm End} \nolimits}
\newcommand*\riso{%
  \xrightarrow[]{\raisebox{-0.25em}{\smash{\ensuremath{\sim}}}}%
}
\def\Sol{\mathop {\rm Sol} \nolimits}

\begin{document}

% Trochu volnější nastavení dělení slov, než je default.
\lefthyphenmin=2
\righthyphenmin=2

%%% Basic information on the thesis


%% The hyperref package for clickable links in PDF and also for storing
%% metadata to PDF (including the table of contents).
%% Most settings are pre-set by the pdfx package.
\hypersetup{unicode}
\hypersetup{breaklinks=true}
%\hypersetup{pdftitle=Invariant differential operators}
%\hypersetup{pdfauthor=Vít Tuček}


% Thesis title in English (exactly as in the formal assignment)
\def\ThesisTitle{Invariant differential operators for $1$-graded geometries}

% Author of the thesis
\def\ThesisAuthor{Vít Tuček}

% Year when the thesis is submitted
\def\YearSubmitted{2017}

% Name of the department or institute, where the work was officially assigned
% (according to the Organizational Structure of MFF UK in English,
% or a full name of a department outside MFF)
\def\Department{Mathematical Institute of Charles University}

% Is it a department (katedra), or an institute (ústav)?
\def\DeptType{Institute}

% Thesis supervisor: name, surname and titles
\def\Supervisor{prof. RNDr. Vladimír Souček, DrSc.}

% Supervisor's department (again according to Organizational structure of MFF)
\def\SupervisorsDepartment{Mathematical Institute of Charles University}

% Study programme and specialization
\def\StudyProgramme{Mathematics}
\def\StudyBranch{4M2 \begin{small}Geometry and topology, global analysis\end{small}}

% An optional dedication: you can thank whomever you wish (your supervisor,
% consultant, a person who lent the software, etc.)
\def\Dedication{%

I wholeheartedly thank my supervisor, prof. Vladimír Souček, for his infinite patience. I am grateful to Jirka and Petra who stood beside me all these years and to all my other friends who provided moral support. I thank Sváťa and Libor for mathematical discussions. Finally, I'm indebted to all who contributed to Sage.

\vspace{5cm}

\begin{center}
I dedicate this work to my parents.
\end{center}
}

% Abstract (recommended length around 80-200 words; this is not a copy of your thesis assignment!)
\def\Abstract{%
In this thesis we classify singular vectors in scalar parabolic Verma modules for those pairs $(\lie{sl}(n, \mathbb{C}), \lie{p})$  of complex Lie algebras where the homogeneous space $\mathrm{SL}(n, \mathbb{C}) / P$ is the Grassmannian of $k$-planes in $\mathbb{C}^n.$ We calculate cohomology of nilpotent radicals with values in certain unitarizable highest weight modules. According to \cite{boe_kostant_2009} these modules have  BGG resolutions with weights determined by this cohomology. Such resolutions induce complexes of invariant differential operators on sections of associated bundles over Hermitian symmetric spaces. We describe formal completions of unitarizable highest weight modules that one can use to modify method from \cite{calderbank_differential_2001} that constructs sequences of differential operators over any $1$-graded (aka almost Hermitian) geometry. We suggest uniform description of octonionic planes that could serve as a basis for better understanding of the exceptional Hermitian symmetric space for group $\mathrm{E}_6.$
}

% 3 to 5 keywords (recommended), each enclosed in curly braces
\def\Keywords{%
{Hermitian symmetric space}, {unitarizable highest weight module}, {nilpotent Lie algebra cohomology}, {octonionic plane}
}

\pagestyle{empty}
\hypersetup{pageanchor=false}
\begin{center}

\centerline{\mbox{\includegraphics[width=166mm]{logo-en.pdf}}}

\vspace{-8mm}
\vfill

{\bf\Large ABSTRACT OF DOCTORAL THESIS}

\vfill

{\LARGE\ThesisAuthor}

\vspace{15mm}

{\LARGE\bfseries\ThesisTitle}

\vfill

\Department

\vfill

\begin{tabular}{rl}

Supervisor of the doctoral thesis: & \Supervisor \\
\noalign{\vspace{2mm}}
Study programme: & \StudyProgramme \\
\noalign{\vspace{2mm}}
Study branch: & \StudyBranch \\
\end{tabular}

\vfill

% Zde doplňte rok
Prague \YearSubmitted

\end{center}

%%% PEOPLE

\newpage


\noindent The results of this thesis were achieved in the period of a doctoral study at the Faculty of Mathematics and Physics, Charles University in Prague in years 2008--2017.\\
\vfill{}
\begin{tabular}{lcl}
\textbf{PhD student:} & & Mgr. Vít Tuček\tabularnewline
& & \tabularnewline
\textbf{Department:} & & \Department \tabularnewline
%  , 186 75, Praha 8
& & Sokolovská 83\tabularnewline
& & 186 75 Praha 8\tabularnewline
& & \tabularnewline
\textbf{Supervisor:} & & \Supervisor \tabularnewline
& & \Department \tabularnewline
& & Sokolovská 83\tabularnewline
& & 186 75 Praha 8\tabularnewline
& & \tabularnewline
\textbf{Opponents:} & & prof. RNDr. Jan Slovák, DrSc. \tabularnewline
& & Department of Mathematics at Statistics,\tabularnewline
& & Masaryk University \tabularnewline
& & Kotlářská 2 \tabularnewline
& & 611 37 Brno\tabularnewline
& & \tabularnewline
& & doc. RNDr. Jiří Vanžura, CSc.\tabularnewline
& & Institute of Mathematics AS CR \tabularnewline
& & Žižkova 22\tabularnewline
& & 616 62 Brno\tabularnewline
& & \tabularnewline
\textbf{Chairman:} & & \Supervisor \tabularnewline
& & \Department \tabularnewline
& & Sokolovská 83\tabularnewline
& & 186 75 Praha 8\tabularnewline
\end{tabular}\\
\vfill{}
\noindent The thesis defense will take place on
\\
\\
The thesis can be viewed at the Study Department of Doctoral Studies of the Faculty of Mathematics and Physics, Charles University in Prague, Ke Karlovu 3, Prague 2.


\newpage

% CZECH TITLE PAGE
\newpage
%
\pagestyle{empty}
\hypersetup{pageanchor=false}
\begin{center}

\centerline{\mbox{\includegraphics[width=166mm]{logo-cs.pdf}}}

\vspace{-8mm}
\vfill

{\bf\Large ATUOREFERÁT DIZERTAČNÍ PRÁCE }

\vfill

{\LARGE\ThesisAuthor}

\vspace{15mm}

{\LARGE\bfseries Invariantní diferenciální operátory pro $1$-gradované geometrie}

\vfill

Matematický ústav Univerzity Karlovy

\vfill

\begin{tabular}{rl}

Školitel: & \Supervisor \\
\noalign{\vspace{2mm}}
Studijní program: & Matematika\\
\noalign{\vspace{2mm}}
Studijní obor: & 4M2 Geometrie a topologie, globální analýza \\
\end{tabular}

\vfill

% Zde doplňte rok
Praha \YearSubmitted

\end{center}


%%% LIDI

\newpage


\noindent Disertační práce byla vypracována na základě výsledků získaných v letech 2008--2017 v rámci doktorského studia na Matematicko-fyzikální fakultě Univerzity
Karlovy v Praze.\\
\vfill{}
\begin{tabular}{lcl}
\textbf{PhD student:} & & Mgr. Vít Tuček\tabularnewline
& & \tabularnewline
\textbf{Školící pracoviště:} & & Matematický ústav Univerzity Karlovy \tabularnewline
%  , 186 75, Praha 8
& & Sokolovská 83\tabularnewline
& & 186 75 Praha 8\tabularnewline
& & \tabularnewline
\textbf{Supervisor:} & & \Supervisor \tabularnewline
& & Matematický ústav Univerzity Karlovy \tabularnewline
& & Sokolovská 83\tabularnewline
& & 186 75 Praha 8\tabularnewline
& & \tabularnewline
\textbf{Oponenti:} & & prof. RNDr. Jan Slovák, DrSc. \tabularnewline
& & Department of Mathematics at Statistics,\tabularnewline
& & Masaryk University \tabularnewline
& & Kotlářská 2 \tabularnewline
& & 611 37 Brno\tabularnewline
& & \tabularnewline
& & doc. RNDr. Jiří Vanžura, CSc.\tabularnewline
& & Institute of Mathematics AS CR \tabularnewline
& & Žižkova 22\tabularnewline
& & 616 62 Brno\tabularnewline
& & \tabularnewline
\textbf{Předseda RDSO:} & & \Supervisor \tabularnewline
& & Matematický ústav Univerzity Karlovy \tabularnewline
& & Sokolovská 83\tabularnewline
& & 186 75 Praha 8\tabularnewline
\end{tabular}\\
\vfill{}
\noindent Obhajoba dizertační práce se koná
\\
\\
S disertační prací je možné se seznámit na studijním oddělení pro doktorské studium
MFF UK, Ke Karlovu 3, Praha 2.


\newpage

%%% Strana s automaticky generovaným obsahem disertační práce. U matematických
%%% prací je přípustné, aby seznam tabulek a zkratek, existují-li, byl umístěn
%%% na začátku práce, místo na jejím konci.

\pagestyle{plain}
\setcounter{page}{1}
\tableofcontents
\newpage

\chapter*{Introduction}
\addcontentsline{toc}{chapter}{Introduction}

Geometrical problems often lead to systems of PDEs which are given by differential opetrator acting between sections of natural bundles. For example, one can consider the problem of finding Killing vector fields $X$ which are infinitesimal isometris of a Riemannian manifold $(M, g)$ and obtain operator $\mathcal{L}_X g$ whose kernel is precisely given by such fields. Or one can ask for infinitesimal conformal automorphisms -- those vector fields, whose flow preserves metric only up to a multiple by a positive scalar function -- where one gets $\mathcal{L}_X - \frac{2}{n} \mathrm{div}\, X \, g.$ There is a notion of prolongation for system of PDEs which basically introduces new variables and which aims at rewriting the system in a more managable form. The ultimate goal is to obtain so called geometric prolongation which expresses the original system as an equation for parallel fields for a connection on a bundle of higher rank. For problems where such a prolongation procedure terminates after finite number of steps one obtains that the space of solutions is locally bounded by the rank of the prolongation bundle and hence one immediately sees that the original problem was overdetermined. For example in the case of conformal Killing vector fields on an $n$-dimensional manifold one obtains bundle of rank $n+2.$ There are however interesting operators which have infinite dimensional kernel in general. For example the conformally invariant  modification of the Beltrami--Laplace operator.

For the class of so called parabolic geometries there is a construction for all natural overdetermined operators which was discovered by \cite{cap_bernstein-gelfand-gelfand_2001} and later simplified by \cite{calderbank_differential_2001}. Given a semisimple Lie group $G$, it's parabolic subgroup $P$ and a $(\mathfrak{g}, P)$-representation $\mathbb{V}$ this construction produces a sequence of differential operators on any Cartan geometry modeled on $G \to G/P$ which has very favourable properties. For example, the first operator in the sequence is automatically geometrically prolonged by suitable modification of the so called tractor connection which is defined on the bundle associated to $\mathbb{V}.$  

The main aim of this work is to generalize the construction of \cite{calderbank_differential_2001} in such a way that one could obtain operators which are not overdetermined. There are two side results. Uniform description of Riemannian metrics of octonionic planes in chapter 2.3 and classification of singular vector in scalar induced parabolic Verma modules for $\mathfrak{sl}(n, \mathbb{C})$ in chapter 2.4 which is a joint work with Libor Křižka.

\chapter{Invariant differential operators}

 Let us briefly recall how a question of existence of invariant differential operators can be reduced to this algebraic problem. Consider a homogeneous space of the form $G/H$ and a homogeneous bundle $\mathcal{V} = G \times_H \mathbb{V} \to G/H$. This bundle is defined as the quotient of $G\times \mathbb{V}$ by the equivalence relation $(g, v) \simeq (gh, \rho(h^{-1}) v)$, where $\rho: H \to GL(\mathbb{V}$ is a representation of $H.$ Smooth sections $s$ of this bundle can be identified with $H$-invariant functions $\{ f \colon G \to \mathbb{V} \,|\, f(gh) = \rho(h^{-1}) f(g)\}$  by $s(gH) = [(g, f(g)].$ From this identification it is evident that there is a well defined left action of $G$ on the space of sections of $\mathcal{V}$ defined by $\widetilde{\rho(g)} f (x) = f(g^{-1}x)$. Now it is quite clear what an invariant differential operator should be -- a differential operator that respects these naturally defined actions:
\begin{gather*}
D \colon \Gamma^\infty(G/H, \mathcal{V}) \to \Gamma^\infty(G/H, \mathcal{W}) \\
D \circ \widetilde{\rho_\mathbb{V}} =  \widetilde{\rho_\mathbb{W}} \circ D.
\end{gather*}
One can construct many intereseting representations of $G$ as kernels of such invariant differential operators but in this thesis we have taken a converse approach and we construct interesting differential operators using representations. 

 So let's see what is the connection to algebra. In general, a (linear) differential operator of order $k$ is a (linear) mapping from the $k$-th jet prolongation 
\[
D\colon \Gamma^\infty(G/H,\mathcal{J}^k \mathcal{V}) \to \Gamma^\infty(G/H, \mathcal{W}).
\]
%that has nontrivial main symbol 
%\[
%0 \neq \sigma_k(D) \in \Gamma^\infty(G/H, \bigodot{\!}^k T^* G/H \otimes \mathcal{V}).
%\]

Since a jet prolongation of a homogeneous vector bundle is again a homogeneous vector bundle and since $G$ acts transitively on $G/H$, we get a correspondence between invariant differential operators and homomorphisms of $J^k \mathbb{V} \to \mathbb{W},$ where  $J^k \mathbb{V}$ denotes the algebraic jet prolongation of $\mathbb{V}.$  The space of such homomorphisms is isomorphic to the space of dual homomorphisms $\mathbb{W}^* \to (J^k\mathbb{V})^* $. Passing to limit $k \to \infty$ we get the correspondence between invariant differential operators and homomorphisms $\mathbb{W} \to  \varinjlim  (J^k\mathbb{V})^* $. It turns out that $\varinjlim  (J^k\mathbb{V})^*$ is actually the induced module $\mathfrak{U(g)\otimes_{U(h)}} \mathbb{V}^*$  and with a little bit of work one can show that the space of invariant linear operators from sections of $\mathcal{V}$ to sections of $\mathcal{W}$  is isomorphic to $\Hom_{\mathfrak{g}}(\mathfrak{U(g)\otimes_{U(h)}}\mathbb{W}^*, \mathfrak{U(g)\otimes_{U(h)}}\mathbb{V}^*)$.





\end{document}