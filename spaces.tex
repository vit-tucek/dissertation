\chapter{Hermitian symmetric spaces}\label{ch:hermitian}


\section{Real Lie algebras}

The \emph{complexification} of a real reducitve Lie algebra $\mfrak{g}_0$ is defined as $\mfrak{g} = \mfrak{g}_0\otimes_\R \C$. With Lie bracket defined naturally by complex linear extension as follows
\[
 [A\otimes z, B\otimes w] = [A,B]\otimes zw.
\]
As a real vector space we have isomorphism $\mfrak{g} \simeq \mfrak{g}_0 \oplus \imath \mfrak{g}_0$ defined by $X\otimes (a+\imath b) \mapsto aX \oplus \imath bX$. Writing elements $Z,Z'\in\mfrak{g}$ as $Z = X + \imath Y$ $Z'=X'+\imath Y'$ we get \[[Z,Z'] = [X,X'] - [Y,Y'] + \imath \left( [X,Y'] + [X',Y]\right).\]

A \emph{real form} of a complex Lie alebra $\mfrak{g}$ is a real Lie algebra $\mfrak{g}_0$ such that $\mfrak{g}$ is the complexification of $\mfrak{g}_0$. A complex Lie algebra usually has many nonisomorphic real forms. For a complexification $\mfrak{g}$ of $\mfrak{g}_0$ we will denote by $\sigma$ the conjugation of $\mfrak{g}$ with respect to the real form $\mfrak{g}_0$.

\begin{example}
 The complexification of the classical Lie algebra of trace-free real matrices $\mfrak{sl}(n,\R)$ is naturally isomorphic to the Lie algebra of trace-free complex matrices $\mfrak{sl}(n,\C)$. Anoter real form of $\mfrak{sl}(n,\C)$ is for example the algebra of trace-free skew Hermitian  matrices  $\mfrak{su}(n)$. %TODO
\end{example}

An \emph{involution} of a Lie algebra $\mfrak{g}$ is Lie algebra homomorphism $\mfrak{g}\to\mfrak{g}$ which squares to identity. An involution $\theta$ of a real semisimple Lie algebra $\mfrak{g}_0$ is called \emph{Cartan involution} if the symmetric bilinear form $B_\theta(X,Y) = -B(X,\theta Y)$ is postivive definite. We will denote the complex linear extension of $\theta$ to the complexification of $\mfrak{g}_0$ by the same symbol and we will still call it the Cartan involution. Any real semisimple Lie algebra has a Cartan involution and it is unique up to inner automorphism.


Take a matrix Lie algebra, i.e. a subalgebra of $\mfrak{gl}(n,\C)$, that is closed under transposition and define $\theta X = -X^\dag$ where $X^\dag = \overline{X}^t.$ Then
 \[
  \theta [X,Y] = -[X,Y]^\dag = - [Y^\dag,X^\dag] = [-X^\dag,-Y^\dag] = [\theta X, \theta Y]
 \]
 shows that $\theta$ is involution. Using the fact that $B$ is invariant with respect to automorphisms we get that
 \begin{align*}
  B_\theta(X,Y) &= -B(X, \theta Y) = - B(\theta X, \theta^2 Y) \\
                &= -B(\theta X, Y) = B(Y, -\theta X) = B_\theta(Y, X),
 \end{align*}
 which demonstrates that $B_\theta$ is symmetric. To show that it is also positive definite, we first need to show that for a scalar product $\langle X,Y \rangle = \Re \tr (XY^\dag)$ we have that the adjoint of $\ad X$ is $\ad X^\dag$.
 \begin{align*}
  \langle [X,Y],Z \rangle &= \Re\tr (XYZ^\dag - YXZ^\dag) = \\
			  &= \Re\tr (YZ^\dag X - YXZ^\dag) =  \\% = \Re\tr (Y (Z^\dag X - X Z^\dag))\\
                          &= \Re\tr( Y(X^\dag Z - ZX^\dag)^\dag) = \langle Y, [X^\dag,Z] \rangle.
 \end{align*}
 Now we can show that $B_\theta$ is in fact positive definite as follows
 \begin{align*}
  B_\theta(X,X) &= -B(X,\theta X) = -\tr (\ad X \circ \ad \theta X )\\
                &= \tr (\ad X \circ \ad (X^\dag))  = \tr (\ad X \circ (\ad X)^*) \geq 0.
 \end{align*}

In fact, any real semisimple Lie algebra $\mfrak{g}_0$ is isomorphic to a Lie algebra of real matrices that is closed under transpose and the isomorphism can be chosen in such a way that the Cartan involution of $\mfrak{g}_0$ is carried to negative transpose (Proposition VI.6.28 of \cite{knapp_lie_1996}).

A Cartan involution $\theta$ of $\mfrak{g}_0$ yields an eigenspace decomposition $\mfrak{g}_0 = \mfrak{k}_0 \oplus \mfrak{p}_0$ of $\mfrak{g}_0$ into $+1$ and $-1$ eigenspaces of $\theta$. Since $\theta$ is a Lie algebra homomorphism, it follows that
\begin{equation}\label{eq:cartan_decomposition_1}
 [\mfrak{k}_0,\mfrak{k}_0] \subseteq \mfrak{k}_0, \quad[\mfrak{k}_0,\mfrak{p}_0] \subseteq \mfrak{p}_0, \quad [\mfrak{p}_0,\mfrak{p}_0] \subseteq \mfrak{k}_0.
\end{equation}
From these relations it is easy to derive that $\mfrak{k}_0$ and $\mfrak{p}_0$ are orthogonal with respect to $B_\theta$ and $B$. If $X$ is in $\mfrak{k}_0$ and $Y$ is in $\mfrak{p}_0$, then $\ad X \circ \ad Y $ sends $\mfrak{k}_0$ to $\mfrak{p}_0$ and $\mfrak{p}_0$ to $\mfrak{k}_0$; i.e. as a matrix in block form subordinated to the decomposition $\mfrak{g}_0 = \mfrak{k}_0 \oplus \mfrak{p}_0$ it has the form $\begin{smatrix} 0 & * \\ * & 0\end{smatrix}$. Thus it has trace $0$ and $B(X,Y) = 0$ and since $\theta Y = - Y$ also $B_\theta (X,Y) = 0$. Because $B_\theta$ is positive definite, the eigenspacese have the property that
\begin{equation}\label{eq:cartan_decomposition_2}
 B \text{ is } \begin{cases} \text{negative definite on } \mfrak{k}_0 \\ \text{positive definite on } \mfrak{p}_0. \end{cases}
\end{equation}

A decomposition of $\mfrak{g}_0 = \mfrak{k}_0 \oplus \mfrak{p}_0$ that satisfies \eqref{eq:cartan_decomposition_1} and \eqref{eq:cartan_decomposition_2} is called \emph{Cartan decomposition} of $\mfrak{g}_0$. Conversely any Cartan decomposition defines a Cartan involution by
\[
\theta =
 \begin{cases}
  + \Id \text{ on } \mfrak{k}_0 \\
  - \Id \text{ on } \mfrak{p}_0
 \end{cases}
\]
and we see that Cartan involutions are in bijective correspondence with Cartan decompositions.


\begin{example}
  Cartan involution on $\mfrak{sl}(n,\R)$ is given by negative transpose, i.e. $\theta X = - X^t$. The subalgebra $\mfrak{k}_0$ is then $\mfrak{so}(n,\R)$ and the ideal $\mfrak{p}_0$ is the space of symmetric matrices.

 For $\mfrak{su}(n)$ is a Cartan involution given by a negative conjugate transpose $\theta X = - \overline{X}^t$. The subalgebra $\mfrak{k}_0$ is the whole $\mfrak{su}(n)$ and $\mfrak{p}_0$ is of course empty.
\end{example}

Cartan involutions and decompositions can be even pushed to the Lie group level.
\begin{theorem}[Theorem VI.6.31 of \cite{knapp_lie_1996}]
 Let $G$ be a semisimple Lie group, let $\theta$ be a Cartan involution of its Lie algebra $\mfrak{g}_0$, let $\mfrak{g}_0 = lie{k}_0 \oplus \mfrak{p}_0$ be the corresponding Catan decomposition, and let $K$ be the analytic subgroup of $G$ with Lie algebra $\mfrak{k}_0$. Then
 \begin{enumerate}
  \item there exists a Lie group automorphism $\Theta$ of $G$ with differential $\theta$, and $\Theta^2=\Id$
  \item the subgroup of $G$ fixed by $\Theta$ is $K$
  \item the mapping $K\times \mfrak{p}_0 \to G$ given by $(k,X)\mapsto k \exp{X}$ is a diffeomorphism onto
  \item $K$ is closed and contains the center $Z$ of $G$
  \item $K$ is compact if and only if $Z$ is finite in which case it is a maximal compact subgroup of $G$.
 \end{enumerate}
\end{theorem}

This theorem justifies the following definition. We will call a real form $\mfrak{g}_0$ of $\mfrak{g}$ \emph{compact} if $\mfrak{g}_0 = \mfrak{k}_0$, i.e. if the Killing form $B$ is negative definite.

Take a Cartan decomposition $\mfrak{g}_0 = \mfrak{k}_0 \oplus \mfrak{p}_0$ and consider $\mfrak{g}_0$ as a subset of its complexification $\mfrak{g}$. Inspecting the signature of $B_\theta$ we easily see that $\mfrak{k}_0\oplus \imath \mfrak{p}_0$ is a compact form, say $\mfrak{u}_0$, of $\mfrak{g}$. Denote by $\sigma$ the conjugation of $\mfrak{g}$ with respect to the real form $\mfrak{g}_0$ and let $\tau$ denote the conjugation with respect to $\mfrak{u}_0$. Both $\sigma$ and $\tau$ are either $\Id$ or $-\Id$ on $\mfrak{k}_0, \imath \mfrak{k}_0, \mfrak{p}_0, \imath \mfrak{p}_0$. This immediately implies that the two involutions commute $\sigma \circ \tau = \tau \circ \sigma$. In particular $\tau (\mfrak{g}_0) \subseteq \mfrak{g}_0$ and the restriction of $\tau$ to $\mfrak{g}_0$ is the Cartan involution $\theta$ corresponding to the Cartan decomposition $\mfrak{g}_0=\mfrak{k}_0 \oplus \mfrak{p}_0$.

Conversely, given a compact form $\mfrak{u}_0$ of a complexification $\mfrak{g}$ of $\mfrak{g}_0$ such that the corresponding conjugations $\tau$ and $\sigma$ commute, we get a Cartan involution for $\mfrak{g}_0$ by restriction of the involution $\tau$ that corresponds to the compact form $\mfrak{u}_0$. Indeed, since $\mfrak{g}_0 \subseteq \mfrak{g} = \mfrak{u}_0 \oplus \imath \mfrak{u}_0$, we get for any $X+\imath Y\in\mfrak{g}_0$
\[
 B_\theta (X+\imath Y,X+\imath Y) = -B(X+\imath Y,X-\imath Y) = -B(X,X) - B(Y,Y),
\]
which is positive definite since $X,Y$ are elements from the compact Lie algebra $\mfrak{u}_0$.

A semisimple Lie algebra has up to inner isomorphism only one compact real form and it can be even shown that the compact form can be chosen in such a way that the corresponding conjugation $\tau$ commutes with any apriori chosen involution $\sigma$ of $\mfrak{g}$. This is the core of the proof of existence of Cartan involution for arbitrary real semisimple Lie algebra $\mfrak{g}_0$.

To summarize, if $\mfrak{g}_0 = \mfrak{k}_0 \oplus \mfrak{p}_0$ is a Cartan decomposition of $\mfrak{g}_0$, then $\mfrak{k}_0 \oplus \imath \mfrak{p}_0$ is a compact  real form of the complexification $\mfrak{g}$ of $\mfrak{g}_0$. Conversely, if $\mfrak{k}_0$ and $\mfrak{p}_0$ are the $+1$ and $-1$ eigenspaces of an involution $\sigma$ of $\mfrak{g}$ then $\sigma$ is Cartan involution if and only if the real form $\mfrak{k}_0 \oplus \imath \mfrak{p}_0$  is compact.

In the following we will use this simple lemma.
\begin{lemma}\label{lem:theta_adjoint}
 If $\mfrak{g}_0$ is a real semisimple Lie algebra and $\theta$ is a Cartan involution, then the adjoint operator to $\ad X$ with respect to $B_\theta$ is $-\ad \theta X$, i.e.
 \begin{equation*}
  (\ad X)^* = -\ad \theta X, \quad \forall X\in\mfrak{g}_0
 \end{equation*}
\end{lemma}
\begin{proof}
  We have for all $Y,Z\in\mfrak{g}$
  \begin{align*}
  B_\theta((\ad X)Y,Z) &= -B([X,Y],\theta Z) = B(Y,[X,\theta Z]) \\
                       &= B(Y,[\theta^2 X, \theta Z]) = B(Y,\theta [\theta X, Z]) \\
		       &= -B_\theta(Y,[\theta X,Z]) = B_\theta(Y,(-\ad \theta X)Z).
  \end{align*}
\end{proof}

In addition to compact form, there is always another real form (also unique up to conjugation) of a complex semisimple Lie algera $\mfrak{g}$. Let $\mfrak{h}_0$ be a Cartan subalgebra of $\mfrak{g}_0$ and consider its complexification $\mfrak{h}$ which is (by definition) a Cartan subalgebra of $\mfrak{g}$. We call a real Lie algebra $\mfrak{g}_0$ a \emph{split form} of $\mfrak{g}$ if the restrictions of elements of $\mfrak{h}^*$ to $\mfrak{h}_0$ are real valued.

To construct a split real form we can just take a suitable basis of $\mfrak{g}$ and its real span. In detail, take a complex Lie algebra $\mfrak{g}$ and its Cartan subalgebra $\mfrak{h}$ and define $\mfrak{h}_0$ as the subset of $\mfrak{h}$ on which all the roots take only real values. There is always a choice (see Theorem VI.6.6 of \cite{knapp_lie_1996}) of root vectors $X_\alpha \in \mfrak{g}_\alpha$ such that $[X_\alpha, X_{-\alpha}] = H_\alpha$ and
\begin{gather*}
\beta \neq -\alpha \,\&\, \alpha + \beta \notin \roots \Longrightarrow [X_\alpha,X_\beta]=0 \\
 \alpha + \beta \in\roots \Longrightarrow [X_\alpha, X_\beta] = N_{\alpha,\beta}X_{\alpha+\beta},
\end{gather*}
where $N_{\alpha,\beta}\in\R$ and $N_{\alpha,\beta} = -N_{-\alpha,-\beta}$. Now the split form of $\mfrak{g}$ is obtained as
\[
 \mfrak{g}_\text{split} = \mfrak{h}_0 \oplus \bigoplus_{\alpha \in\roots} \R X_\alpha
\]
and a compact form of $\mfrak{g}$ can be then given as
\[
 \mfrak{k}_0 = \imath \mfrak{h}_0 \oplus \bigoplus_{\alpha\in\roots^+} \R (X_\alpha - X_{-\alpha}) \oplus \imath \R (X_\alpha + X_{-\alpha}).
\]

Let $\mfrak{g}_0$ be a real semisimple Lie algebra and let $\theta$ be its Cartan involution. A Cartan subalgebra $\mfrak{h}_0$ of $\mfrak{g}_0$ is called \emph{$\theta$-stable} if $\theta(\mfrak{h}_0)=\mfrak{h}_0$. In such a case we have $\mfrak{h}_0= (\mfrak{h}_0\cap\mfrak{k}_0) \oplus (\mfrak{h}_0\cap\mfrak{p}_0$, where $\mfrak{g}_0 = \mfrak{k}_0\oplus\mfrak{p}_0$ is the Cartan decomposition. We call the dimension of $\lie{t} = \mfrak{h}_0\cap\mfrak{k}_0$ the \emph{compact dimension} of a $\theta$-stable  Cartan subalgebra $\mfrak{h}_0$ and similarly the dimension of $\lie{a} = \mfrak{h}_0\cap\mfrak{p}_0$ is called the \emph{noncompact dimension} of $\mfrak{h}_0$. A $\theta$-stable Cartan subalgebra $\mfrak{h}_0 \leq \mfrak{g}_0$ is called \emph{maximally compact} or \emph{maximally noncompact} if and only if its compact (respectively noncompact) dimension is maximal possible.

%TODO theta stabilita, komplexni konjugace a akce thety (viz Vogan, mozna Knapp)

%\subsection{Iwasawa decomposition}
In contrast to the complex case, Cartan subalgebras of real Lie algebras are not unique up to conjugation. Up to conjugation by inner automorphism, there is a finite number of Cartan subalgebras and any Cartan subalgebra is conjugated via an inner automorphism to a $\theta$-stable Cartan subalgebra. Moreover there is up to conjugation by an element of $K$ only one maximally compact (or maximally noncompact) $\theta$-stable Cartan subalgebra of $\mfrak{g}_0$.

Let $\mfrak{g}$ be a complexification of $\mfrak{g}_0$ and let $\mfrak{h}$ be a complexification of a $\theta$-stable maximally noncompact Cartan subalgebra $\mfrak{h}_0$ of $\mfrak{g}_0$. Let $\sigma$ be a complex conjugation of $\mfrak{g}$ with respect to the real form $\mfrak{g}_0$. For $\alpha\in\roots$ we define $\sigma^* \alpha$ by $(\sigma^* \alpha) (H) = \overline{\alpha(\sigma H)}$ for all $H\in\mfrak{h}$. The mapping $\sigma^*$ is an involutive automorphism of the root system $\roots(\mfrak{g},\mfrak{h})$. A root $\alpha \in \roots(\mfrak{g},\mfrak{h})$ is called \emph{compact root} if $\sigma^*\alpha = -\alpha$. The set of compact roots is denoted by $\roots_c$.

\begin{proposition}[Proposition 2.3.8 of \cite{cap_parabolic_2009}]
 The set of compact roots is given by $\roots_c = \{ \alpha \in \roots \,|\,\alpha \text{ restricted to } \mfrak{a} \text{ is } 0 \}$. It is an abstract root system on the Euclidean subspace of $\imath \mfrak{t}\oplus \mfrak{a}$ spanned by its elements. For $\alpha\in\roots_c$, the root space $\mfrak{g}_\alpha$ is contained in $\mfrak{k}\leq \mfrak{g}$.
\end{proposition}

%\subsection{Satake diagrams}

Choose a set of positive roots $\roots^+$ such that $\sigma^* \alpha \in \roots^+$ for all $\alpha\in\roots^+\setminus\roots_c$. One way to obtain such a system of positive roots is to choose ordering of $\roots$ by choosing a basis $\{H_1,\ldots, H_p\}$ of $\mfrak{a}$ and extending it to a basis $\{H_1,\ldots,H_r\}$ of $\imath\mfrak{t}\oplus\mfrak{a}$. By definition $\alpha\in\roots^+$ if and only if there is an index $j$ such that $\alpha(H_j) > 0$ and $\alpha(H_i) = 0 $ for all $i<j$. By definition $\alpha(H_i)\neq 0$ for some $i\leq p$ for $\alpha \in \roots^+\setminus \roots_c$ and since all $\alpha(H_j)$ are real, $\sigma H_i = H_i$ for $i\leq p$ and $\sigma(H_i) = -H_i$ for $i>p$.

Let $\sroots$ be the set of simple roots of $\roots^+$ and put $\sroots_c = \sroots \cap \roots_c$. Then $\sroots_c$ are a system of simple roots for $\roots_c$ and we order our system of simple roots $\sroots$ in such a way that elements of $\sroots_c$ come last. The following lemma is due to Ichir{o} Satake.
\begin{lemma}
\begin{enumerate}
 \item The element $\sigma^*\alpha - \alpha$ is not a root for any $\alpha\in\roots$.
 \item For $\alpha \in \sroots \setminus \sroots_c$, there is a unique element $\alpha'\in\sroots\setminus\sroots_c$ such that $\sigma^*\alpha-\alpha'$ is a linear combination of compact roots.
\end{enumerate}
\end{lemma}

The \emph{Satake diagram} of the real Lie algebra $\mfrak{g}_0$ is defined as follows. In the Dynkin diagram associated to the simple system $\sroots$, represent compact roots by a black dot and roots in $\sroots\setminus\sroots_c$ by a white dot. Moreover, for any $\alpha\in\sroots\setminus\sroots_c$ such that $\sigma^*\alpha\neq\alpha$, connect $\alpha$ by an arrow  to the unique simple root $\alpha'\in\sroots\setminus\sroots_c$ such that $\sigma^*\alpha -\alpha'$ is a linear combination of compact roots.



%\subsection{Parabolic subalgebras of real Lie algebras}

\begin{definition}[page 274, \cite{knapp_cohomological_1995}]\label{def:thetastableparabolic}
 Let $\mfrak{q}$ be a parabolic subalgebra of the complexification of a real semisimple Lie algebra $\mfrak{g}_0$ and let $\theta$ be the corresponding Cartan involution. We will call $\mfrak{q}$ to be $\theta$-stable if $\theta \mfrak{q} = \mfrak{q}$. It follows then that also the opposite parabolic subalgebra $\mfrak{q}^-$ is $\theta$-stable and that the Levi part is $\theta$-stable $\theta \mfrak{l} = \mfrak{l}$.
\end{definition}

For a real Lie algebra $\mfrak{g}_0$ we define $\mfrak{q}_0$ to be a parabolic subalgebra if and only if the complexification $\mfrak{q}$ is a parabolic subalgebra of the complexification $\mfrak{g}$. Similarly to the complex case, there is an equivalence between parabolic subalgebra $\mfrak{g}$ and gradings on $\mfrak{g}$ also in the real category. The standard parabolic subalgebras are given by crossed roots in Satake diagrams, where we are allowed to cross only white roots (i.e. those that are not in $\roots_c$.

%TODO rozepsat, real Iwasawa?s

Let's look at some specific examples of real Lie algebas and groups.

\[
 I_{m,n} = \begin{pmatrix} \Id_m & 0 \\ 0 & -\Id_n \end{pmatrix}, \quad J = \begin{pmatrix} 0 & \Id_n \\ -\Id_n & 0 \end{pmatrix}
\]

\[X^\dag = \overline{X}^t\]

\begin{equation*}\label{eq:algebras_definition}
\begin{aligned}
 \mfrak{sl}(n,\C) &= \{ X \in \mfrak{gl}(n,\C) \,|\, \tr X = 0 \} & \text{for } n \geq 1\\
 \mfrak{so}(m,n)  &= \{ X\in \mfrak{gl}(m+n,\R) \,|\, X^\dag I_{m,n}+I_{m,n}X = 0 \} & \text{for } m+n\geq 3\\
 \mfrak{su}(m,n)  &= \{ X \in \mfrak{sl}(m+n,\C)  \,|\, X^\dag I_{m,n}+I_{m,n}X = 0 \} & \text{for } m+n \geq 2\\
 \mfrak{so}^\dag(2n) &= \{ X \in \mfrak{su}(n,n) \,|\, X^t I_{n,n}J + I_{n,n} JX = 0 \} & \text{for } n \geq 2\\
 \mfrak{sp}(n,\R) &= \{ X \in \mfrak{gl}(2n,\R) \,|\, X^tJ + JX = 0 \} & \text{for } n\geq 1
\end{aligned}
\end{equation*}

\begin{equation*}
\begin{aligned}
\mathrm{SU}(p,q) &= \{ g \in \mathrm{GL}(p +q, \mathbb{C} \,|\, g I_{p,q}g^\dag = I_{p,q} \}\\
\mathrm{Sp}(n, \mathbb{C}) &= \{ g \in \mathrm{GL}(2n, \mathbb{C} \,|\, g^t J g  = J \}\\
\mathrm{Sp}(n, \mathbb{R}) &= \mathrm{Sp}(n, \mathbb{C}) \cap \mathrm{SU}(n,n) \\
\mathrm{O}(2n, \mathbb{C}) &= \{ g \in \mathrm{GL}(n, \mathbb{C} \,|\, g^t JI_{n,n} g  = JI_{n,n} \}\\
\mathrm{SO}^*(2n, \mathbb{C}) &=  \mathrm{O}(2n, \mathbb{C}) \cap \mathrm{SU}(n,n) \\
\end{aligned}
\end{equation*}


The Satake diagram of $\mfrak{su}(p,q)$ for $p+q = n+1$, $1 \leq p \leq \frac{n-1}{2}$.

\begin{center}
\resizebox{\textwidth}{!}{
     \begin{tikzpicture}
	\node[nroot] (a1) [label=above:$\alpha_1$] {};
%	\node[nroot] (a2) [right= of a1] [label=above:$\alpha_2$] {};
	\node (a3) [right= of a1] {};
	\node (a4) [right= of a3] {};
	\node[nroot] (ap) [right=of a4] [label=above:$\alpha_p$] {};
	\node[croot] (ap1) [right= of ap] [label=above:$\alpha_{p+1}$] {};
	\node (d1) [right=of ap1] {}; \node (d2) [right=of d1] {};
	\node[croot] (aq) [right=of d2] [label=above:$\alpha_{q}$] {};
	\node[nroot] (aq1) [right=of aq] [label=above:$\alpha_{q+1}$] {};

	\node (e1) [right=of aq1] {}; \node (e2) [right=of e1] {};
	\node[nroot] (an) [right=of e2] [label=above:$\alpha_n$] {};

	\draw (a1) to  (a3); \draw [dotted] (a3) to (a4);
	\draw (a4) to (ap) to (ap1)  to (d1); \draw [dotted] (d1) to (d2);
	\draw (d2) to (aq) to (aq1) to (e1); \draw [dotted] (e1) to (e2); \draw (e2) to (an);

	\draw [<->, bend right] (a1) to (an); \draw [<->, bend right] (ap) to (aq1);
	\draw [<->, bend right] (a3) to (e2); 	\draw [<->, bend right] (a4) to (e1);
     \end{tikzpicture}
     }
\end{center}

%  Another version of the same diagram.
% \begin{center}\begin{tikzpicture}
%  	\node[nroot] (a1)                         [label=above:$\alpha_1$] {};
% 	\node[nroot] (a2)  [right=of a1]  [label=above:$\alpha_2$] {};
% 	\node[nroot] (an)  [below=of a1] [label=below:$\alpha_n$] {};
% 	\node[nroot] (am) [below=of a2] [label=below:$\alpha_{n-1}$] {};
% 	\node            (d1)  [right=of a2] {};
% 	\node		(d2)  [right=of d1] {};
% 	\node[nroot] (ap) [right=of d2]  [label=above:$\alpha_p$] {};
% 	\node            (e1)  [right=of am] {};
% 	\node		(e2)  [right=of e1] {};
% 	\node[nroot] (ar) [right=of e2]  [label=below:$\alpha_{q+1}$] {};
% 	\node[croot] (app) [right=of ap] [label=above:$\alpha_{p+1}$] {};
% 	\node[croot] (aq) [right=of ar] [label=below:$\alpha_{q}$] {};
% 	\node[croot] (p1) [right=of app] [label=above:$\alpha_{p+2}$] {};
% 	\node[croot] (p2) [right=of aq] [label=below:$\alpha_{q-1}$] {};
%
%
% 	\draw (a1) to (a2) to (d1); \draw [dotted] (d1) to (d2); \draw (d2) to (ap) to (app) to (p1);
% 	\draw (an) to (am) to (e1); \draw  [dotted] (e1) to (e2); \draw (e2) to (ar) to (aq) to (p2);
%
% 	\draw [<->] (a1) to (an); \draw [<->]  (a2) to (am);\draw [<->] (ap) to (ar); \draw [dotted, bend left] (p1) to (p2);
% \end{tikzpicture}\end{center}

The Satake diagram of $\mfrak{su}(p,p)$ for $n=2p+1$, $p\leq 2$.

% \begin{center}\begin{tikzpicture}
%  	\node[nroot] (a1)                         [label=above:$\alpha_1$] {};
% 	\node[nroot] (a2)  [right=of a1]  [label=above:$\alpha_2$] {};
% 	\node[nroot] (an)  [below=of a1] [label=below:$\alpha_n$] {};
% 	\node[nroot] (am) [below=of a2] [label=below:$\alpha_{n-1}$] {};
% 	\node            (d1)  [right=of a2] {};
% 	\node		(d2)  [right=of d1] {};
% 	\node[nroot] (ap1) [right=of d2]  [label=above:$\alpha_{p-1}$] {};
% 	\node            (e1)  [right=of am] {};
% 	\node		(e2)  [right=of e1] {};
% 	\node[nroot] (ap2) [right=of e2]  [label=below:$\alpha_{p+1}$] {};
% 	\node[nroot] (ap) [right=of ap1] [label=right:$\alpha_p$] {};
%
%
%
% 	\draw (a1) to (a2) to (d1); \draw [dotted] (d1) to (d2); \draw (d2) to (ap1) to (ap);
% 	\draw (an) to (am) to (e1); \draw  [dotted] (e1) to (e2); \draw (e2) to (ap2) to (ap);
%
% 	\draw [<->] (a1) to (an); \draw [<->]  (a2) to (am);\draw [<->] (ap1) to (ap2);
% \end{tikzpicture}\end{center}

\begin{center}
     \begin{tikzpicture} % SU(p,q) ~ A_{n-1} relative
	\node[nroot] (a1) [label=above:$\alpha_1$] {};
	\node[nroot] (a2) [right= of a1] [label=above:$\alpha_2$] {};
	\node (a3) [right= of a2] {};
	\node (a4) [right= of a3] {};
	\node[nroot] (a5) [right=of a4] [label=above:$\alpha_p$] {};
	\node (a6) [right= of a5] {};
	\node (a7) [right=of a6] {};
	\node[nroot] (a8) [right=of a7] [label=above:$\alpha_{n-2}$] {};
	\node[nroot] (a9) [right=of a8] [label=above:$\alpha_{n-1}$] {};
	\draw (a1) to (a2) to (a3); \draw [dotted] (a3) to (a4);
	\draw (a4) to (a5) to (a6); \draw [dotted](a6) to (a7);
	\draw (a7) to (a8) to (a9);

	\draw [<->, bend right] (a1) to (a9); \draw [<->, bend right] (a2) to (a8); \draw [<->, bend right] (a4) to (a6);
     \end{tikzpicture}
\end{center}

The Satake diagram for $\mfrak{so}(2,2n-1)$.

\begin{center}
     \begin{tikzpicture}[decoration={markings,mark=at position .6 with {\arrow[line width=2pt]{>}}}] % SO(2n-1,\R) ~ B_n relative
	\node[nroot] (a1) [label=above:$\alpha_1$] {};
	\node[nroot] (a2) [right= of a1] [label=above:$\alpha_2$] {};
	\node[croot] (a3) [right= of a2] [label=above:$\alpha_3$] {};
	\node (d1) [right= of a3] {}; \node (d2) [right= of d1] {};
	\node[croot] (a5) [right=of d2] [label=above:$\alpha_{n-1}$] {};
	\node[croot] (a6) [right=of a5] [label=above:$\alpha_{n}$] {};
	\draw (a1) to (a2) to (a3) to (d1); \draw [dotted] (d1) to (d2);
	\draw (d2) to (a5); \draw [postaction={decorate},double distance=1.5pt] (a5) to (a6);
     \end{tikzpicture}
\end{center}

The Satake diagram for $\mfrak{so}(2,2n-2)$.

\begin{center}
     \begin{tikzpicture}] % SO^*(2n) ~ D_n relative
	\node[nroot] (a1) [label=above:$\alpha_1$] {};
	\node[nroot] (a2) [right= of a1] [label=above:$\alpha_2$] {};
	\node[croot] (a3) [right= of a2] [label=above:$\alpha_3$] {};
	\node (d1) [right= of a3] {}; \node (d2) [right= of d1] {};
	\node[croot] (a5) [right=of d2] [label=above:$\alpha_{n-2}$] {};
	\node[croot] (a6) [above right= of a5] [label=above:$\alpha_{n}$] {};
	\node[croot] (a7) [below right= of a5] [label=above right:$\alpha_{n-1}$] {};
	\draw (a1) to (a2) to (a3) to (d1);
	\draw [dotted] (d1) to (d2);
	\draw (d2) to (a5) to (a6);
	\draw (a5) to (a7);
     \end{tikzpicture}
\end{center}

The Satake diagram for $\mfrak{sp}(n,\R)$.

\begin{center}
     \begin{tikzpicture}[decoration={markings,mark=at position .5 with {\arrowreversed[line width=2pt]{>}}}] % Sp(n,\R) ~ C_n relative
	\node[nroot] (a1) [label=above:$\alpha_1$] {};
	\node[nroot] (a2) [right= of a1] [label=above:$\alpha_2$] {};
	\node (a3) [right= of a2] {};
	\node (a4) [right= of a3] {};
	\node[nroot] (a5) [right=of a4] [label=above:$\alpha_{n-1}$] {};
	\node[nroot] (a6) [right=of a5] [label=above:$\alpha_{n}$] {};
	\draw (a1) to (a2) to (a3); \draw [dotted] (a3) to (a4);
	\draw (a4) to (a5); \draw [postaction={decorate},double distance=1.5pt] (a5) to (a6);
     \end{tikzpicture}
\end{center}

The Satake diagram for $\mfrak{so}^*(2n)$ for  even $n$.

\begin{center}
     \begin{tikzpicture}] % SO^*(2n) ~ D_n relative
	\node[croot] (a1) [label=above:$\alpha_1$] {};
	\node[nroot] (a2) [right= of a1] [label=above:$\alpha_2$] {};
	\node[croot] (a3) [right= of a2] [label=above:$\alpha_3$] {};
	\node (d1) [right= of a3] {}; \node (d2) [right= of d1] {};
	\node[croot] (a4) [right= of d2] [label=above:$\alpha_{n-3}$] {};
	\node[nroot] (a5) [right=of a4] [label=above:$\alpha_{n-2}$] {};
	\node[nroot] (a6) [above right= of a5] [label=above:$\alpha_{n}$] {};
	\node[croot] (a7) [below right= of a5] [label=above right:$\alpha_{n-1}$] {};
	\draw (a1) to (a2) to (a3) to (d1);
	\draw [dotted] (d1) to (d2);
	\draw (d2) to (a4) to (a5) to (a6);
	\draw (a5) to (a7);
     \end{tikzpicture}
\end{center}

The Satake diagram for $\mfrak{so}^*(2n)$ for  odd $n$.
\begin{center}
     \begin{tikzpicture}] % SO^*(2n) ~ D_n relative
	\node[croot] (a1) [label=above:$\alpha_1$] {};
	\node[nroot] (a2) [right= of a1] [label=above:$\alpha_2$] {};
	\node[croot] (a3) [right= of a2] [label=above:$\alpha_3$] {};
	\node (d1) [right= of a3] {}; \node (d2) [right= of d1] {};
	\node[nroot] (a4) [right= of d2] [label=above:$\alpha_{n-3}$] {};
	\node[croot] (a5) [right=of a4] [label=above:$\alpha_{n-2}$] {};
	\node[nroot] (a6) [above right= of a5] [label=above:$\alpha_{n}$] {};
	\node[nroot] (a7) [below right= of a5] [label=above right:$\alpha_{n-1}$] {};
	\draw (a1) to (a2) to (a3) to (d1);
	\draw [dotted] (d1) to (d2);
	\draw (d2) to (a4) to (a5) to (a6);
	\draw (a5) to (a7); \draw [<->] (a6) to (a7);
     \end{tikzpicture}
\end{center}

The Satake diagram for $\mfrak{e}_6^{-14}$ - EIII.

\begin{center}
    \begin{tikzpicture}% E6 relative
        \node[nroot] (a1) [label=above:$\alpha_1$] {};
        \node[croot] (a3) [right=of a1] [label=above:$\alpha_3$] {};
        \node[croot] (a4) [right=of a3] [label=above:$\alpha_4$] {};
        \node[croot] (a5) [right=of a4] [label=above:$\alpha_5$] {};
        \node[nroot] (a6) [right=of a5] [label=above:$\alpha_6$] {};
        \node[croot] (a2) [below=of a4] [label=right:$\alpha_2$] {};
        \draw (a1) to (a3) to (a4) to (a5) to (a6);
        \draw (a4) to (a2); \draw [<->, bend left] (a1) to (a6);
     \end{tikzpicture}
\end{center}

The Satake diagram for $\mfrak{e}_7^{-25}$ - EVII.

\begin{center}
     \begin{tikzpicture} % E7 relative
        \node[nroot] (a1) [label=above:$\alpha_1$] {};
        \node[croot] (a3) [right=of a1] [label=above:$\alpha_3$] {};
        \node[croot] (a4) [right=of a3] [label=above:$\alpha_4$] {};
        \node[croot] (a5) [right=of a4] [label=above:$\alpha_5$] {};
        \node[nroot] (a6) [right=of a5] [label=above:$\alpha_6$] {};
        \node[nroot] (a7) [right=of a6] [label=above:$\alpha_7$] {};
        \node[croot] (a2) [below=of a4] [label=right:$\alpha_2$] {};
        \draw (a1) to (a3) to (a4) to (a5) to (a6) to (a7);
        \draw (a4) to (a2);
     \end{tikzpicture}
\end{center}

While Satake diagrams show best the structure of a real Lie algebra they classify, there is an alternative approach which starts with a maximally compact $\theta$-stable Cartan subalgebra. We won't go into details and refer reader to \cite{knapp_lie_1996}. In the following we will actually use the Vogan diagrams to denote the Hermitian symmetric pairs. The reason is that there is always only one white node corresponding to the unique simple noncompact root and it is the same simple root that defines the complex parabolic subalgebra of the Hermitian pair. I.e. by turning all white nodes in these Vogan diagrams into crossed nodes one gets the usual notation for the parabolic pair. 

\section{Classical Hermitian symmetric spaces}


Let $G$ be simply connected, connected simple Lie group, let $Z$ be its center and let $K$ be a closed  maximal subgroup of $G$ such that $K/Z$ is compact. A unitary representation $(\rho,\mathbb{V})$  of $G$ such that the underlying $(\mfrak{g},K)$-module is an irreducible highest weight module is called unitary highest weight module. From the universal property of (generalized) Verma modules it follows that any unitarizable highest weight module is the unique irreducible quotient of a (generalized) Verma module. It is a result of Harish-Chandra that nontrivial unitarizable highest weight modules occur only when $G/K$ is a noncompact Hermitian symmetric space. The table \ref{fig:herm_pairs} of all such Hermitian pairs $(G,K)$ is given below.

Let $\mfrak{g}_0$ and $\mfrak{k}_0$ be the corresponding Lie algebras of $G$ and $K$ and let $\mfrak{g}$ and $\mfrak{k}$ denote their complexifications. By $G_\C$ and $K_C$ we denote the complexifications of $G$ and $K$. The Cartan decomposition gives us $\mfrak{g}_0 = \mfrak{k}_0 \oplus \mfrak{q}_0$ and upon complexification we get a splitting of  $\mfrak{q} = \mfrak{p}_-\oplus\mfrak{p}_+$. % into eigenspaces of the complex structure that is defined on the tangent space $T_{eK}G/K \simeq \mfrak{q}$.
There is a choice of a Cartan subalgebra $\mfrak{h}$ such that $\mfrak{h} \leq \mfrak{k}$. With respect to this Cartan subalgebra we have a triangular decomposition $\mfrak{g}=\mfrak{n}^-\oplus\mfrak{h}\oplus\mfrak{n}$ with $\mfrak{h} \leq \mfrak{k}$ and $\mfrak{p}_-\leq \mfrak{n}^-$, $\mfrak{p}_+ \leq \mfrak{n}$.

Both algebras $\mfrak{p} := \mfrak{k}\oplus \mfrak{p}_+$ and $\oppar := \mfrak{k}\oplus\mfrak{p}_-$ are parabolic subalgebras of $\mfrak{g}$. Moreover their nilradicals $\mfrak{p}_-$ and $\mfrak{p}_+$ are not only nilpotent but even abelian. By $P$ and $\overline{P}$ we denote the corresponding parabolic subgroups of $G_\C$. The homogeneous space $G_\C/P$ is diffeomorphic to the compact Hermitian symmetric space and $\mfrak{p}_-$ is naturally mapped via exponential map and projection to an open and dense subset of this compact manifold. The so called Harish-Chandra embedding gives us a realization of the noncompact dual $G/K$ as an orbit in this embedded $\mfrak{p}_-$. This realizes $G/K$ as a bounded Hermitian symmetric domain in $\mfrak{p}_-$ and it manifests the Hermitian structure on $G/K$.

% \begin{figure}\label{fig:herm_pairs}
% \begin{center}
% \begin{tabular}{llll}
% $G_\mathbb{C}$ & {\centering $G$ }& $G'$ & $K$\\\hline
% $SL(p+q,\mathbb{C})$ & $SU(p+q)$ &$SU(p,q)$ &$S(U(p)\times U(q))$\\
% $SO(p+2,\mathbb{C})$ & $SO(p+2)$ & $SO(2,p)$ & $SO(p)\times SO(2)$\\
% $SO(2n,\mathbb{C})$ & $SO(2n)$ & $SO^*(2n)$ & $U(n)$\\
% $Sp(2n,\mathbb{C})$ & $Sp(n)$ & $Sp(n,\mathbb{R})$ & $U(n)$\\
% $E_6^\mathbb{C}$ & $E_6$ & $E_6^{-14}$& $SO(10)\times SO(2)$\\
% $E_7^\mathbb{C}$ & $E_7$ & $E_7^{-25} $ & $E_6\times SO(2)$
% \end{tabular}
% \end{center}
% \end{figure}
\begin{table}[H]\label{fig:herm_pairs}
\begin{center}%TODO zkontrolovat
\begin{tabular}{llll}
$G_\mathbb{C}$ & {\centering $G$ }&  $K$\\\hline
$SL(p+q,\mathbb{C})$ & $SU(p,q)$ &$S(U(p)\times U(q))$\\
$SO(p+2,\mathbb{C})$ &  $SO(2,p)$ & $S(O(p)\times O(2))$\\
$SO(2n,\mathbb{C})$ &   $SO^*(2n)$ & $U(n)$\\
$Sp(2n,\mathbb{C})$ &   $Sp(n,\mathbb{R})$ & $U(n)$\\
$E_6^\mathbb{C}$ &   $E_6^{-14}$& $\mathrm{Spin}(10)\times SO(2)$\\
$E_7^\mathbb{C}$ &  $E_7^{-25} $ & $E_6\times SO(2)$
\end{tabular}\caption{Hermitian symmetric pairs}
\end{center}
\end{table}

Let $\roots$ be the set of roots of $(\mfrak{g},\mfrak{h})$ and let $\roots_c$ denote the set of roots of $(\mfrak{k},\mfrak{h})$. We call elements of $\roots_c$ the compact roots and the remaining roots in $\roots_n = \roots \setminus \roots_c$ are called noncompact. We define the positive roots $\roots^+$ in such a way that elements of $\roots_n^+ = \roots^+ \cap \roots_n$ span $\mfrak{p}_-$. We denote the positive compact roots by $\roots^+_c = \roots_c \cap \roots^+$. By $\omega_i$ we denote the $i$-th fundamental weight in the standard ordering.

%As usual we denote the Weyl group of $(\mfrak{g},\mfrak{b})$ by $W$ and the subgroup generated by reflections $s_\alpha$ for $\alpha\in \roots_c$ we denote by $W_c$. Then $W_c$ is isomorphic to the Weyl group of the root system $\roots_c$. Also we let $\rho$ be the sum of positive roots and $\leq$ denotes the Bruhat order on $\mfrak{h}^*$.

Let $\sroots$ denote the set of  simple roots of $\roots$. From now on we will denote the set of compact roots by $\sroots_c$. The set $\sroots\setminus \sroots_c$ contains a single element, the unique \emph{non-compact simple root}, which we will denote by $\gamma_1$. We define a totally ordered sequence $\xi_1,\xi_2,\ldots \xi_t$ of \emph{strongly orthogonal} non-compact positive roots inductively by $\gamma_i$ being the unique minimal element in $\{ \alpha \in \roots_n^+ \,|\, (\alpha, \gamma_j) = 0, 1 \leq j < i\}$. We define
\[\mu_j := \sum_{i=j}^t \xi_i\]
and note that they are all dominant weights. This result as well as the following one is due to Schmid, Kostant and Hua. It plays an important role in the proof of classification of unitarizable highest weight modules \cite{enright_intrinsic_1990} which are the topic of chapter \ref{ch:unitarizable}.

\begin{theorem}
 Let $I$ denote the set of integral multiindices $\underline{i}=(i_1,\ldots,l_t)$ with $i_1\geq i_2 \geq \cdots  \geq i_t \geq 0$. Let $F_{\underline{i}}$ denote the irreducible finite-dimensional $\mfrak{k}$-module with highest weight $\sum_{j=1}^t i_j\xi_i$. Then $S(\mfrak{p}_+)$ has a multiplicity free decomposition
 \begin{equation}\label{eq:khs}
  S(\mfrak{p}_+) \simeq \bigoplus_{\underline{i}\in I} F_{\underline{i}}.
 \end{equation}
The table \ref{tbl:strongly_og} gives sets of  strongly orthogonal roots for  Hermitian symmetric pair.
\begin{table}[H]\label{tbl:strongly_og}\begin{center}
  \begin{tabular}{CCCCC}
  \mfrak{g}_0 & \beta& t & \xi_{t-i} & \mu_{t-i} \\\hline
   \mfrak{su}(p,q) &\alpha_p& p & \epsilon_i-\epsilon_{t-i+1} & \omega_i + \omega_{t-i+1} \\
   \mfrak{so}(2,2n-1) & \alpha_1 & 2 & \epsilon_1+\epsilon_2, \epsilon_1 - \epsilon_2 & \omega_2, 2\omega_1\\
   \mfrak{sp}(n,\R) & \alpha_n & n & 2\epsilon_i & 2\omega_1\\
   \mfrak{so}(2,2n-2) &\alpha_1 & 2 & \epsilon_1 + \epsilon_2, \epsilon_1 - \epsilon_2 & \omega_2,2\omega_1\\
%    \mfrak{e}_6 & \alpha_1\text{ or }\alpha_6 & 2 &
  \end{tabular}\caption{Strongly orthogonal roots}\end{center}
\end{table}
\end{theorem}
Moreover, up to a scalar multiple, there exists for each $j$ a unique non-zero $v_j\in S(\mfrak{p}_+)^{\mfrak{n}}$ of weight $\mu_j.$ Furthemore $S(\mfrak{p}_+)^{\mfrak{n}}$ is just the polynomial algebra on the $v_j,j=1,\ldots, t.$ We will use this in the section \ref{sec:invariant} and we refer to \cite{goodman_symmetry_2009} for details.

\section{Octonionic planes}\label{sec:octo}

The classical Hermitian symmetric spaces (i.e. those corresponding to the classical Lie groups) are treated thoroughly in \cite{faraut_analysis_1994}. The exceptional cases were descibed in \cite{drucker_exceptional_1978, drucker_simplified_1981}. The smaller of the two exceptional cases can be seen as a complexification of the octonionic projective plane \cite{landsberg_projective_2001, landsberg_projective_2003} and as such it plays a prominent role in \cite{atiyah_projective_2003}. It's hyperplane section was described in \cite{pazourek_hyperplane_2011}. The real octonionic projective plane is a symmetric Riemannian space $\mathrm{F_4}/\mathrm{Spin}(9)$ with exceptional holonomy $\mathrm{Spin}(9).$ Elementary description of this space as well as other it's noncompact (pseudo)Riemannian cousins was given in \cite{held_semi-riemannian_2009}. It required a lot of case by case calculations and classification of Osserman manifolds to identify the manifolds with the appropriate homogeneous spaces. In this section we show how to obtain the Riemannian metric and curvature in a uniform and elemental way.

\subsection{Octonions and the exceptional Jordan algebra}

Let $\octo$ denote the normed algebra of octonions over a field $\field$ of characteristic $0$ (we will consider only $\field=\reals$ and $\field =\complex$) and let $N\colon\octo \to \reals$ denote the corresponding norm. Let $\oscal{\,}{\,}$ denote the scalar product of octonions  defined by polarization $\oscal{x}{y} = \frac{1}{2} (N(x+y) - N(x) - N(y))$. Sometimes it is defined without the factor of $1/2$, because then some formulas are simpler and one can also work over a field of characteristic $2$. The conjugation is defined by $\conj{x} = 2\oscal{x}{1}1 - x$, where $1$ is the unit of the octonionic algebra $\octo$. We define the real part of $x$ as $\Re(x) = \frac{x + \conj{x}}{2}$ and the imaginary part as $\Im(x) = \frac{x-\conj{x}}{2}$. 

One can construct $\octo$ e.g. by the Cayley-Dickson process. Basic relations concerning the scalar product are 
\begin{equation}\label{eq:scal_product}
\begin{aligned}
\oscal{x}{y} & = \frac{ x\conj{y} + y\conj{x}}{2}\\
\oscal{x}{y} & = \oscal{\conj{x}}{\conj{y}}.
\end{aligned}
\end{equation}
Another useful  identities one gets via polarizations (see \cite[p. 5]{springer_octonions_2000})
\begin{gather}
\oscal{x_1y}{x_2y} = \oscal{x_1}{x_2} N(y), \quad \oscal{xy_1}{xy_2} = N(x)\oscal{y_1}{y_2} \label{eq:scalar_product2}\\
\oscal{x_1y_1}{x_2y_2} + \oscal{x_1y_2}{x_2y_1} = 2\oscal{x_1}{x_2}\oscal{y_1}{y_2}.\label{eq:scalar_product3}
\end{gather}
Combining the second equation of \eqref{eq:scal_product} with \eqref{eq:scalar_product3} we get
\begin{equation}\label{eq:scalar_product4}
2\oscal{a}{c}\oscal{b}{d} = \oscal{a\conj{b}}{c\conj{d}} + \oscal{a\conj{d}}{c\conj{b}}.
\end{equation}
Finally, we will need
\begin{equation}\label{eq:scalar_product5}
\oscal{ab}{c} = \oscal{b}{\conj{a}c}, \quad \oscal{a}{bc} = \oscal{a\conj{c}}{b}.
%2\oscal{a,\conj{d}}\oscal{b}{\conj{c}} &= \oscal{ab}{\conj{cd}} + \oscal{a\conj{c}}{}
\end{equation}

The octonionic multiplication  can be ``decomposed'' using scalar product and cross product similarly  as in the case of quaternions. Namely, we have $\oscal{x}{y} = \Re( x\conj{y})$ and we define the cross product as \[x \times y = \Im (x\conj{y}) = \frac{1}{2}( x\conj{y} - y\conj{x}).\] It is not really a cross product, but its restriction to the space of imaginary octonions. It is easy to see that the cross product has purely imaginary values.

%% HARVEY
% The octonionic multiplication  can be ``decomposed'' using scalar product and cross product similarly  as in the case of quaternions. Namely, we have $\oscal{x}{y} = \Re( x\conj{y})$ and we define the cross product as \[x \times y = \Im (\conj{y}x) = \frac{1}{2}( \conj{y}x - \conj{x}y).\] It is not really a cross product, but its restriction to the space of imaginary octonions is (Elduque \ref{}). $x \times y = \frac{1}{2}[x,y] - (\Re x \Im y - \Re y \Im x)$ We follow conventions from Harvey \ref{harvey}.

A bit more obscure is the triple cross product 
\[
u \times v \times w = \frac{1}{2}\bigl( u(\conj{v}w) - w(\conj{v}u) \bigr)
\]
that appears in the theory of calibrations (and in fact defines what is called the Cayley calibration), see \cite{harvey_spinors_1990} for details. Finally, the associator is defined as
\[
\{ x,y,z \} = (xy)z - x(yz).
\]
The associator is completely antisymmetric. This property is actually equivalent to the alternativity of the octonionic algebra which in turn implies that any subalgebra generated by two elements is associative. (Artin theorem, \ref{}) It is easy to see that if any entry is a multiple of unit in $\octo$, then the associator is zero. Thus it descends to a map $\Lambda^3 \mathrm{Im}\, \octo \to \field$.

The projective octonionic plane $\projplane$ can be defined via the exceptional formally real Jordan algebra $\jalg[3] = \mathrm{Herm}(3,\octo)$. The points of the geometry are then idempotents of trace one. The automorphism group of this Jordan algebra is the compact real group $F_4$ whose action preserves the trace and determinant of these octonionic matrices. One can define $F_4$-invariant positive definite scalar product on $\jalg$ as
\[
 \jscal{A}{B} = \mathrm{Tr}(A \circ B),
\]
where $A \circ B$ is the Jordan product of $A$ and $B$ defined by
\[
 A \circ B = \frac{1}{2} \left( AB + BA \right).
\]
We can generalize this as follows. Take a matrix $G=\begin{pmatrix} \gamma_1 & 0 & 0 \\ 0 & \gamma_2 & 0 \\ 0 & 0 & \gamma_3\end{pmatrix}$ such that $G^2 = \mathrm{Id}$ and all $\gamma_i$ are from the ground field $\field$. Then we define $G$-Hermitian matrices as those satisfying $GA^\dagger = AG$ and on the space of those matrices we still have the same.


Let $G = \begin{pmatrix} \gamma_1 & 0 & 0\\ 0 & \gamma_2 & 0\\ 0 & 0 & \gamma_3\end{pmatrix}$ be a real (or complex?) matrix such that $G^2 = 1$. Scalar product on $\octo^3$ is defined by $\oscal{x}{y} = \frac{1}{2}(x^\dagger G y + y^\dagger G x) = \sum_{i=1}^3 \gamma_i\oscal{x_i}{y_i}$, where we use the standard scalar product on octonions for which $\oscal{x}{x} = N(x)$.

 For $x\in\octo^3$ define $\varphi (x) = \frac{1}{x^\dagger G x} x (Gx)^\dagger$, it is a trace one idempotent by direct calculation that only uses the fact that $\octo$ form a composition algebra $N(xy) = N(x)N(y)$. Moreover it is $G$-Hermitian, meaning $GA=A^\dagger G$. All $G$-symmetric matrices have the following form
\begin{equation}\label{g-matrix}
  \begin{pmatrix}
  \gamma_1 r_1 & \gamma_2 \conj{x_1} & \gamma_3 \conj{x_2} \\
  \gamma_1 x_1 & \gamma_2 r_2 & \gamma_3 \conj{x_3} \\
  \gamma_1 x_2 & \gamma_2 x_3 & \gamma_3 r_3
  \end{pmatrix}.
\end{equation}

The scalar product on such matrices is defined by $\jscal{A}{B} = \mathrm{Tr}\; (A \circ B)$ and for a general $G$-symmetric matrix this gives us the quadratic form
\[
	\jscal{A}{A} = 2\left( \gamma_1\gamma_2 N(x_1) + \gamma_1\gamma_3 N(x_2) + \gamma_2\gamma_3 N(x_3)\right) + \sum_{i=1}^{3} r_i^2
\]
and by polarization this yields
\begin{equation}\label{eq:jscal1}
\jscal{A}{B} = 2\left( \gamma_1\gamma_2 \oscal{x_1}{y_1} + \gamma_1\gamma_3 \oscal{x_2}{y_2} + \gamma_2\gamma_3\oscal{x_3}{y_3} \right) + \sum_{i=1}^3 r_i s_i.
\end{equation}
The set of all $G$-Hermitian matrices with multiplication give by
\[
A \circ B = \frac{1}{2}(AB+BA)
\]
form an exceptional Jordan algebra and from classification of Jordan algebras (see \cite{springer_octonions_2000} for details) we know that up to isomorphism that there is only one exceptional Jordan algebra over the complex numbers given by $\jalg[3][\octo_\complex]$ which corresponds to $G=\mathrm{Id}$. Over the real numbers there are actually three isomorphism classes represented by $\jalg[3]$, $\jalg[1,2]$, corresponding to $G = \begin{psmatrix} 1 & 0 & 0 \\ 0 & 1 & 0 \\ 0 & 0 & -1\end{psmatrix}$ and the Jordan algebra over the split octonions $\jalg[3][\octo']$. The automorphism group of an exceptional Jordan algebra is a group of type $F_4$. In the complex case we get of course the complex Lie group $F_4$, the automorphism group of $\jalg$ is the compact Lie group of type $F_4$, the automorphism group of $\jalg[3][\octo']$ is the split real Lie group of type $F_4$ and finally the $\jalg[1,2]$ has the only remaining real Lie group $F_4^{-20}$ as its automorphism group. 

Now we define the affine patches / coordinate charts.

\begin{align}
\projplane: & \quad U_i = \set{(x_1, x_2, x_3)^T \in \octo^3 \,\middle|\, x_i = 1} \\
\projplane[1,1]:  & \quad 
	\begin{aligned}
			U_1 &= \set{(1, x_2, x_3)^T \in \octo^3 \,\middle|\, 1 + N(x_2) - N(x_3) > 0} \\
			U_2 &= \set{(x_1, 1, x_3)^T \in \octo^3 \,\middle|\, N(x_1) + 1  - N(x_3) > 0} \\
    \end{aligned} \\
\projplane[2][H]:  & \quad U_3 = \set{(x_1, x_2, 1)^T \in \octo^3 \,\middle|\, N(x_1) + N(x_2) - 1 < 0} \\
\projplane[2][P]_s: & \quad U_i = \set{(x_1, x_2, x_3)^T \in \octo_s^3 \,\middle|\, x_i = 1 \,\&\, \sum\nolimits_i N(x_i) > 0}
\end{align}

In all cases we would like to identify lines corresponding to the different points on different patches. Over an associative algebra it is quite easy, because we can just use an equivalence relation $(x_1, x_2, x_3)^T \sim  (\lambda x_1, \lambda x_2,\lambda x_3)^T$ for any nonzero $\lambda$ from the algebra. This relation is however not transitive when we are working over the octonions. Nevertheless it is transitive on our affine coordinate patches as was shown in \cite{held_semi-riemannian_2009}. There is only one complication, in the case of the split octonions we actually demand $\lambda$ to have positive norm $N(\lambda) > 0$.


With these preliminaries behind us we can finally define the $\mathbb{R}$ (or $\mathbb{C}$) linear mapping $\varphi: \octo^3 \to \jalg $\ by
\[
	\varphi(a) = \frac{a(Ga)^\dagger}{a^\dagger Ga}.
\]
The idea for such a mapping is borrowed from \cite{allcock_identifying_1997, aslaksen_restricted_1991}. Its matrix form looks like this
\begin{equation}\label{eq:varphi}
\varphi(a) = \frac{1}{\sum\nolimits_i\gamma_i N(a_i)} 
	\begin{pmatrix}
		\gamma_1 N(a_1) & \gamma_2 a_1 \conj{a_2} & \gamma_3 a_1 \conj{a_3} \\
		\gamma_1 a_2\conj{a_1} & \gamma_2 N(a_2)  & \gamma_3 a_2 \conj{a_3} \\
		\gamma_1 a_3\conj{a_1} & \gamma_2 a_3 \conj{a_2} & \gamma_3 N(a_3) \\
	\end{pmatrix}
\end{equation}

Four octonionic planes were defined in the article \cite{held_semi-riemannian_2009} by giving maps and transition functions. We are going to show that the domains used for maps are actually just the analogs of classical affine coordinates of the projective or hyperbolic plane. We will show that these octonionic planes can be actually defined as the space of idempotent matrices of trace one in some exceptional Jordan algebra $\jalg$ which makes the $F_4$-symmetry quite manifest. Now we define our affine coordinate patches which are completely analogous to the classical picture from $\reals^{1,2}$. The only novelty is the split projective plane $\projplane[2][P]_s$.


\begin{lemma}
 The equations for the octonionic plane are 
\begin{equation}\label{eq:trace}
\gamma_1 r_1 + \gamma_2 r_2 + \gamma_3 r_3 = 1,
\end{equation}
\begin{align}
 r_1 & = \gamma_1 r_1^2 + \gamma_2 N(x_1) + \gamma_3 N(x_2) \label{eq:diag1}\\
 r_2 & = \gamma_2 r_2^2 + \gamma_1 N(x_1) + \gamma_3 N(x_3) \label{eq:diag2} \\
 r_3 & = \gamma_3 r_3^2 + \gamma_1 N(x_2) + \gamma_2 N(x_3) \label{eq:diag3}
\end{align}
\begin{align}
 r_3 x_1 &= \conj{x_3}x_2 \label{eq:antidiag1} \\
 r_2 x_2 &= x_3 x_1 \label{eq:antidiag2} \\
 r_1 x_3 &= x_2\conj{x_1}.  \label{eq:antidiag3}
\end{align}
\end{lemma}
\begin{proof}
Straightforward, we have just used the equation $\mathrm{Tr}\, A = 1$ to rewrite the off-diagonal terms of $A^2=A$:
\begin{align*}
\gamma_1 x_1 &= (\gamma_1 r_1 + \gamma_2 r_2)\gamma_1 x_1  + \gamma_1\gamma_3 \conj{x_3}x_2 \\
\gamma_1 x_2 &= (\gamma_1 r_1 + \gamma_3 r_3)\gamma_1 x_2  + \gamma_1\gamma_2 x_3 x_1 \\
\gamma_2 x_3 &= (\gamma_2 r_2 + \gamma_3 r_3)\gamma_2 x_3  + \gamma_1\gamma_2 x_2\conj{x_1}.
\end{align*}


\end{proof}
\begin{lemma}
The map $\varphi$ restricted to the affine coordinate patches has a well-defined smooth inverse on an octonionic plane, i.e. for any $A \in \jalg$ satisfying $A^2 = A$ and $\mathrm{Tr}\, A = 1$ there exists $a\in \octo^3$ whose one coordinate is 1 and such that $\varphi(a) = A$.
\end{lemma}
\begin{proof}
The equation \eqref{eq:trace} implies that at least one $r_i$ is nonzero. Without loss of generality, we will treat the case $r_1 \neq 0$ as the others follow by permuting the indices.

Comparing \eqref{g-matrix} with \eqref{eq:varphi} and imposing $a_1 = 1$ we see that we must have $r_1 =  \sfrac{1}{\oscal{a}{a}}$ and $\gamma_1 x_1 = r_1 \gamma_1 a_2$ which leads us to defining $a_2 = \sfrac{x_1}{r_1}$. Similarly, we obtain $a_3 = \sfrac{x_2}{r_1}$. Now we need to check whether these choices satisfy all the remaining equations.

First of all $\oscal{a}{a} = \gamma_1 + \gamma_2 \sfrac{N(x_1)}{r_1^2} + \gamma_3\sfrac{N(x_2)}{r_1^2}$ where the right hand side is equal to $\sfrac{ \gamma_1 r_1^2+ \gamma_2N(x_1) + \gamma_3N(x_2)}{r_1^2}$. By the equation \eqref{eq:diag1} this is $1/r_1$ as it should be. The remaining antidiagonal term of \eqref{g-matrix} = \eqref{eq:varphi} is $x_3 = \sfrac{a_3\conj{a_2}}{\oscal{a}{a}}  = \sfrac{ x_2\conj{x_1}}{r_1}$ which is exactly the equation \eqref{eq:antidiag3}.

The diagonal terms pose a slightly bigger challenge as they are equivalent to $r_1 r_2  = N(x_1)$ and $r_1 r_3 = N(x_2)$. For a nonzero $x_1, x_2$ these equations can be derived from \eqref{eq:antidiag1} and \eqref{eq:antidiag2}. In the associative case, these are two of the equations for $\varphi(a)$ to be of rank 1 and this actually remains true even in the non-associative case. However, we can obtain these equations by elementary calculations  as we will illustrate in the case $x_2 = 0$. For then the equation \eqref{eq:antidiag3} yields $x_3 = 0$ as we suppose $r_1 \neq 0$ from the beginning. The equation \eqref{eq:antidiag1} implies either $r_3 = 0$ or $x_1 = 0$. The latter case leading to $A$ being a zero everywhere except at the upper left position and $a = (1,0,0)^T$. In the former case, we first take the square of the  condition $1 = \mathrm{Tr}\, A = \gamma_1 r_1 + \gamma_2 r_2$ to obtain $r_1^2 + 2\gamma_1\gamma_2 r_1 r_2 + r_2^2 = 1$ and subsequently 
\[
r_1r_2 = \frac{1-r_1^2 - r_2^2}{2\gamma_1\gamma_2}.
\]
The equations \eqref{eq:diag1}, \eqref{eq:diag2} turn into 
\[
\gamma_1 r_1 = r_1^2 +  \gamma_1\gamma_2N(x_1), \qquad \gamma_2 r_2 = r_2^2 + \gamma_1\gamma_2 N(x_1)
\]
and their sum yields after a rearrangement 
\[
N(x_1) = \frac{1- r_1^2 - r_2^2}{2\gamma_1\gamma_2}.
\]
\end{proof}

\begin{lemma}
The  directional derivative of $\varphi$ is given by 
\[
\partial_u \varphi(x) = \frac{\psi(x,u) - 2\oscal{x}{u}\varphi(x)}{\oscal{x}{x}},
\]
where $\psi(x,u) = u(Gx)^\dagger + x(Gu)^\dagger$ or in matrix form
% \[
% x(Gu)^\dagger = 
% \begin{pmatrix}
% \gamma_1 x_1 \conj{u_1} & \gamma_2 x_1 \conj{u_2} & \gamma_3 x_1 \conj{u_3}  \\
% \gamma_1 x_2 \conj{u_1} & \gamma_2 x_2 \conj{u_2} & \gamma_3 x_2 \conj{u_3}  \\
% \gamma_1 x_3 \conj{u_1} & \gamma_2 x_3 \conj{u_2} & \gamma_3 x_3 \conj{u_3}
% \end{pmatrix}
% \]
% \[
% u(Gx)^\dagger = 
% \begin{pmatrix}
% \gamma_1 u_1 \conj{x_1} & \gamma_2 u_1 \conj{x_2} & \gamma_3 u_1 \conj{x_3}  \\
% \gamma_1 u_2 \conj{x_1} & \gamma_2 u_2 \conj{x_2} & \gamma_3 u_2 \conj{x_3}  \\
% \gamma_1 u_3 \conj{x_1} & \gamma_2 u_3 \conj{x_2} & \gamma_3 u_3 \conj{x_3}
% \end{pmatrix}
% \]
% which yields
\[
\psi(x,u) =
\begin{pmatrix}
2\gamma_1\oscal{x_1}{u_1} & \gamma_2(x_1\conj{u_2} + u_1\conj{x_2}) & \gamma_3(x_1\conj{u_3} + u_1\conj{x_3})  \\
\gamma_1(x_2\conj{u_1} + u_2\conj{x_1}) & 2\gamma_2\oscal{x_2}{u_2} & \gamma_3(x_2\conj{u_3} + u_2\conj{x_3})  \\
\gamma_1(x_3\conj{u_1} + u_3\conj{x_1}) & \gamma_2(x_3\conj{u_2} + u_3\conj{x_2}) & 2\gamma_3\oscal{x_3}{u_3}
\end{pmatrix}.
\]
\end{lemma}
\begin{proof}
The directional derivative $\partial_u \varphi(x) = \lim_{t\to 0} \frac{\mathrm{d}}{\mathrm{d}t} \varphi(x+tu)$ and using  
\[
\partial_u \left( \frac{1}{x^\dagger G x} \right) = - 2\frac{\oscal{x}{u}}{\oscal{x}{x}^2}
\]
we obtain the result after a short calculation.
\end{proof}

Now it is straightforward to calculate the pullback $\jscal{\partial_u \varphi(x)}{\partial_v \varphi(x)}$ of the Jordan algebra scalar product  and calculate the curvature using Gauss equation. The main point here is that the scalar products on the Jordan algebras are invariant with respect to their automorphism groups and so one obtains Riemannian metrics on these coordinate patches which are invariant with respect to these groups. What remains is to prove that these patches actually cover whole orbit in each case and determine the automorphism groups. But these matters are well known \cite{springer_octonions_2000, shukuzawa_orbit_2003}.

%\begin{theorem}
%The pullback along $\varphi$ is given by
%\begin{equation}
%\jscal{\partial_u \varphi(x)}{\partial_v \varphi(x)} = 2\frac{ \oscal{x}{x}\oscal{u}{v} - \oscal{x}{u}\oscal{x}{v} + \beta(x,u,v) }{\oscal{x}{x}^2}, 
%\end{equation}
%where
%\begin{equation}
%\beta(x,u,v) =  2\Bigl( \sum_{i=1}^3 \oscal{x_i\conj{v_i}}{u_i\times x_i} + \sum_{i,j} \gamma_i\gamma_j
%\oscal{x_i}{u_i(\conj{x_j}\times \conj{v_j}) + v_i(\conj{x_j}\times \conj{u_j})} \Bigr).
%\end{equation}
%\end{theorem}
%
%\begin{proof}
%
%Notice that 
%\[
%\varphi(x) = \frac{\psi(x,x)}{2\oscal{x}{x}}
%\]
%and so it is sufficient to calculate $\jscal{\psi(x,u)}{\psi(x,v)}$ in order to obtain $\jscal{\partial_u \varphi(x)}{\partial_v \varphi(x)}$. Namely we have
%\begin{equation}\label{eq:jscal2}
%\begin{aligned}
%\jscal{\partial_u \varphi(x)}{\partial_v \varphi(x)} 
%		& = \jscal{\frac{\psi(x,u) - 2\oscal{x}{u}\varphi(x)}{\oscal{x}{x}}}{\frac{\psi(x,v) - 2\oscal{x}{v}\varphi(x)}{\oscal{x}{x}}} \\
%        & = \frac{1}{\oscal{x}{x}^2} \bigl( \jscal{\psi(x,u)}{\psi(x,v)} \\
%		& \qquad \qquad \quad - 2\oscal{x}{u}\jscal{\psi(x,v)}{\varphi(x)}\\ 
%        & \qquad \qquad \quad - 2\oscal{x}{v}\jscal{\psi(x,u)}{\varphi(x)} \\
%		& \qquad \qquad \quad +  4\oscal{x}{u}\oscal{x}{v}\jscal{\varphi(x)}{\varphi(x)} \bigr) \\
%\end{aligned}
%\end{equation}
%
%The matrices look like this
%\[
%\psi(x,u) =
%\begin{pmatrix}
%2\gamma_1\oscal{x_1}{u_1} & \gamma_2(x_1\conj{u_2} + u_1\conj{x_2}) & \gamma_3(x_1\conj{u_3} + u_1\conj{x_3})  \\
%\gamma_1(x_2\conj{u_1} + u_2\conj{x_1}) & 2\gamma_2\oscal{x_2}{u_2} & \gamma_3(x_2\conj{u_3} + u_2\conj{x_3})  \\
%\gamma_1(x_3\conj{u_1} + u_3\conj{x_1}) & \gamma_2(x_3\conj{u_2} + u_3\conj{x_2}) & 2\gamma_3\oscal{x_3}{u_3}
%\end{pmatrix}
%\]
%\[
%\psi(x,v) =
%\begin{pmatrix}
%2\gamma_1\oscal{x_1}{v_1} & \gamma_2(x_1\conj{v_2} + v_1\conj{x_2}) & \gamma_3(x_1\conj{v_3} + v_1\conj{x_3})  \\
%\gamma_1(x_2\conj{v_1} + v_2\conj{x_1}) & 2\gamma_2\oscal{x_2}{v_2} & \gamma_3(x_2\conj{v_3} + v_2\conj{x_3})  \\
%\gamma_1(x_3\conj{v_1} + v_3\conj{x_1}) & \gamma_2(x_3\conj{v_2} + v_3\conj{x_2}) & 2\gamma_3\oscal{x_3}{v_3}
%\end{pmatrix}
%\]
%and their scalar product is easily computed by \eqref{eq:jscal1}
%\begin{align*}
%    \jscal{\psi(x,u)}{\psi(x,v)} & = 4\sum_{i=1}^3 \oscal{x_i}{u_i}\oscal{x_i}{v_i} \\
%            &\quad + 2\gamma_1\gamma_2 \oscal{ x_1\conj{u_2} + u_1\conj{x_2} }{ x_1\conj{v_2} + v_1\conj{x_2} } \\
%            &\quad + 2\gamma_1\gamma_3 \oscal{ x_1\conj{u_3} + u_1\conj{x_3} }{ x_1\conj{v_3} + v_1\conj{x_3} } \\
%            &\quad + 2\gamma_2\gamma_3 \oscal{ x_2\conj{u_3} + u_2\conj{x_3} }{ x_2\conj{v_3} + v_2\conj{x_3} } 
%\end{align*}
%We expand the term $\oscal{ x_1\conj{u_2} + u_1\conj{x_2} }{ x_1\conj{v_2} + v_1\conj{x_2} }$ to
%\[
%	N(x_1)\oscal{u_2}{v_2} + N(x_2)\oscal{u_1}{v_1} + \oscal{x_1\conj{u_2}}{v_1\conj{x_2}} + \oscal{u_1\conj{x_2}}{x_1\conj{v_2}}
%\]
%and we can use \eqref{eq:scalar_product4} to rewrite the second two terms as
%\[
%2\oscal{x_1}{v_1}\oscal{x_2}{u_2} - \oscal{x_1\conj{x_2}}{v_1\conj{u_2}} + 2\oscal{u_1}{x_1}\oscal{x_2}{v_2} - \oscal{u_1\conj{v_2}}{x_1\conj{x_2}}.
%\]
%To obtain a metric that resembles Fubini-Study metric in homogeneous coordinates, we want to write 
%\[\jscal{\psi(x,u)}{\psi(x,v)}  = 2\oscal{x}{u}\oscal{x}{v} + 2\oscal{x}{x}\oscal{u}{v} + \beta(x,u,v).\]
%So let's actually calculate 
%\[
%\beta(x,u,v) = \jscal{\psi(x,u)}{\psi(x,v)}  - 2\oscal{x}{u}\oscal{x}{v} - 2\oscal{x}{x}\oscal{u}{v}.
%\]
%
%First of all, the ``diagonal terms'' are
%\begin{align*}
%2\oscal{x_i}{u_i} & \oscal{x_i}{v_i}  - \oscal{x_i}{x_i}\oscal{u_i}{v_i} = & \\
%	& = \oscal{x_i\conj{x_i}}{u_i\conj{v_i}} + \oscal{x_i\conj{v_i}}{u_i\conj{x_i}} - 2\oscal{x_i}{x_i}\oscal{u_i}{v_i} & \text{according to \eqref{eq:scalar_product4}} \\
%	& = N(x_i)\oscal{1}{u_i\conj{v_i}} + \oscal{x_i\conj{v_i}}{u_i\conj{x_i}} - N(x_i)\oscal{u_i}{v_i} - N(x_i)\oscal{\conj{u_i}}{\conj{v_i}} & \\
%    & = N(x_i)\oscal{v_i}{u_i} - N(x_i)\oscal{u_i}{v_i} + \oscal{x_i\conj{v_i}}{u_i\conj{x_i}} - \oscal{x_i\conj{u_i}}{x_i\conj{v_i}} & \text{by \eqref{eq:scalar_product5}} \\
%    & = \oscal{x_i\conj{v_i}}{u_i\conj{x_i} - x_i\conj{u_i}} & \\
%    & = \oscal{x_i\conj{v_i}}{2u_i\times x_i}. & 
%\end{align*}
%The ``off-diagonal terms'' are handled similarly by using \eqref{eq:scalar_product5} and \eqref{eq:scal_product}
%\begin{align*}
%	2\oscal{x_i}{u_i}&\oscal{x_j}{v_j} + 2\oscal{x_j}{u_j}\oscal{x_i}{v_i} - 2\oscal{x_i\conj{x_j}}{u_i\conj{v_j} + v_i\conj{u_j}} = \\
%    	& = \oscal{x_i\conj{x_j}}{u_i\conj{v_j}} +  \oscal{x_i\conj{v_j}}{u_i\conj{x_j}} + \oscal{x_j\conj{x_i}}{u_j\conj{v_i}} +  \oscal{x_j\conj{v_i}}{u_j\conj{x_i}} - 2\oscal{x_i\conj{x_j}}{u_i\conj{v_j} + v_i\conj{u_j}}  \\
%       	& = \oscal{x_i\conj{x_j}}{u_i\conj{v_j}} +  \oscal{x_i\conj{v_j}}{u_i\conj{x_j}} + \oscal{x_i\conj{x_j}}{v_i\conj{u_j}} +  \oscal{x_i\conj{u_j}}{v_i\conj{x_j}} - 2\oscal{x_i\conj{x_j}}{u_i\conj{v_j} + v_i\conj{u_j}}  \\
%        & = \oscal{x_i\conj{v_j}}{u_i\conj{x_j}} + \oscal{x_i\conj{u_j}}{v_i\conj{x_j}} - \oscal{x_i\conj{x_j}}{u_i\conj{v_j} + v_i\conj{u_j}} \\
%        & = \oscal{x_i}{(u_i\conj{x_j})v_j + (v_i\conj{x_j})u_j - (u_i\conj{v_j})x_j - (v_i\conj{u_j})x_j}.
%\end{align*}
%If we look more closely on the second entry of the scalar product we see, that we can simplify it using the associator and its properties as follows
%\begin{align*}
%(u_i\conj{x_j})v_j - (u_i\conj{v_j})x_j & = u_i(\conj{x_j}v_j - \conj{v_j}x_j) + \{u_i, \conj{x_j}, v_j\} + \{u_i, \conj{v_j}, x_j\} \\
%		& = 2u_i(\conj{x_j}\times \conj{v_j}) - \{u_i, x_j, v_j \} - \{u_i, v_j, x_j\} \\
%        & = 2u_i(\conj{x_j}\times \conj{v_j})
%\end{align*}
%
%Putting it all together, we arrive at
%\[
%\beta(x,u,v) = 2\Bigl( \sum_{i=1}^3 \oscal{x_i\conj{v_i}}{u_i\times x_i} + \sum_{i,j} \oscal{x_i}{u_i(\conj{x_j}\times \conj{v_j}) + v_i(\conj{x_j}\times \conj{u_j})} \Bigr).
%\]
%An easy calculation gives that $\beta(x,x,u) = \beta(x,u,x) = 0$. %What about \beta(u,x,x)?
%
%Finally, we can express the pullback continuing from \eqref{eq:jscal2}
%\begin{align*}
%\jscal{\partial_u \varphi(x)}{\partial_v \varphi(x)} 
%        & = \frac{1}{\oscal{x}{x}^2}\bigl( 2\oscal{x}{u}\oscal{x}{v} + 2\oscal{x}{x}\oscal{u}{v} + \beta(x,u,v) \\
%		& \qquad \qquad \quad - 2\oscal{x}{u}\frac{2\oscal{x}{x}\oscal{x}{v} + 2\oscal{x}{x}\oscal{x}{v} + \beta(x,x,v)}{2\oscal{x}{x}} \\
%		& \qquad \qquad \quad - 2\oscal{x}{v}\frac{2\oscal{x}{x}\oscal{x}{u} + 2\oscal{x}{x}\oscal{x}{u} + \beta(x,u,x)}{2\oscal{x}{x}} \\
%		& \qquad \qquad \quad + 4\oscal{x}{u}\oscal{x}{v} \bigr) \\
%        & = \frac{1}{\oscal{x}{x}^2}\bigl( 2\oscal{x}{u}\oscal{x}{v} + 2\oscal{x}{x}\oscal{u}{v} + \beta(x,u,v) \\
%		& \qquad \qquad \quad  - 4\oscal{x}{u}\oscal{x}{v} \\        
%		& \qquad \qquad \quad - 4\oscal{x}{v}\oscal{x}{u}  \\
%		& \qquad \qquad \quad +  4\oscal{x}{u}\oscal{x}{v} \bigr) \\
%        & = 2\frac{ \oscal{x}{x}\oscal{u}{v} - \oscal{x}{u}\oscal{x}{v} + \beta(x,u,v) }{\oscal{x}{x}^2}.
%\end{align*}
%\end{proof}

%\section{Vanoce}
%The result is
%\begin{equation}
%\jscal{\psi(x,u)}{\psi(x,v)} = 2\oscal{x}{x}\oscal{u}{v} + 2\Re(h(x,u)h(x,v)) + \beta(x,u,v),
%\end{equation}
%where
%	\[h(x,u) = \sum_{i=1}^3 \gamma_i x_i \conj{u_i}\] 
%and
%\begin{equation}
%\beta(x,u,v) = 2\sum_{j \geq i = 1}^3 \oscal{x_i}{(u_i \times x_j)\conj{v_j} + (v_i \times x_j)\conj{u_j}  - \{u_i, x_j, \conj{v_j}\} - \{v_i, x_j, \conj{u_j} \}}
%\end{equation}
%
%\[
%\{a,b,c\} = \frac{2}{3}\text{cyklicka suma (a x b) x c}
%\]
%
%\[
%\Re (ab) = 2\Re(a)\Re(b) - \oscal{a}{b}
%\]

\section{Invariant differential operators}

There is a well known correspondence between homomorphisms of parabolic Verma modules and invariant differential operators acting on sections of associated bundles over $G/P$ \cite{cap_bernstein-gelfand-gelfand_2001}. For $\mfrak{g}$-dominant integral weights there is actually a whole complex of invariant differential operators called Bernstein-Gelfand-Gelfand resolution \cite{bernstein_differential_1975}.  These operators can be actually defined on any parabolic geometry modeled on $(G, P)$ \cite{cap_bernstein-gelfand-gelfand_2001, calderbank_differential_2001}. Using the results of \cite{huang_dirac_2006} it can be shown that construction of \cite{calderbank_differential_2001} works also for bundles associated to formal completions of unitarizable highest weight modules \cite{tucek_yamabe_2012}. For regular weights all Kostant modules admit a BGG resolution over $G/P$. See \cite{enright_diagrams_2014} and references therein. 

Connection between unitarizable highest weight modules and invariant differential operators is one of the main topcis of \cite{davidson_differential_1991}. To explain it here, we have to change notation a little bit for let $\alpha$ denote multiindices of integers.

Let $P(V,W)$ denote the space of polynomials between two complex vector spaces $V$ and $W$ which are endowed with a Hermitian inner product. For $p\in P(V,W)$ define $p(\partial):P(V,W)\to P(V,\C)$ by equality $p(\partial) e^{(s|t)_V} = p(\overline{s})e^{(s|t)_V}$. In coordinates we get that for $p(t) = \sum_{\alpha} a_{\alpha} t^{\alpha}$ the resulting linear differential operator has the expression $p(\partial)= \sum_{\alpha} \overline{a_{\alpha}} t^{\alpha}$. The \emph{Fischer inner product} on $P(V,W)$ is defined by $\langle p,q\rangle := (q(\partial)p)(0)$. Explicitly in coordinates we have for  $p(t) = \sum_{\alpha} a_{\alpha} t^{\alpha}$ and  $q(t) = \sum_{\alpha} b_{\alpha} t^{\alpha}$ that $\langle p,q\rangle = \sum_{\alpha} \alpha! (a_{\alpha},b_{\alpha})_W$.

\begin{lemma}
 Let $p,q\in P(V,W)$ and let $f\in P(V,\C)$. Then
 \[
  \langle p,fq\rangle = \langle f(\partial)p, q \rangle = \langle q(\partial)p,f \rangle.
 \]
\end{lemma}
\begin{proof}
 All three expression are euqal to $((fq)(\partial)p)(0)$.
\end{proof}

As was argued in \cite{davidson_differential_1991} this inner product can be used to define a nondegenerate pairing between $M(\lambda)$ and its conjugate dual $M(\lambda)^*$. In this duality one gets that $J(\lambda)$ is orthogonal to $L(\lambda)$.

\begin{theorem}[Theorem 2.9 of \cite{davidson_differential_1991}]
 Let $m_1,\ldots,m_k$ be any set of generators for the maximal submodule $J(\lambda)$ of the Verma module $M(\lambda)$ which we view as an $S(\mfrak{p}_+)$-module. Then the simple submodule $L(\lambda)$ of the conjugate dual $M(\lambda)^*$ is the kernel of the constant coefficient operators $m_i(\partial)$, $1\leq i\leq t.$
\end{theorem}
\begin{proof}
 A polynomial $f$ is in $L$ if and only if $\langle f, p m_i \rangle = 0$ for all $p\in P(\mfrak{p}_-,\C) = S(\mfrak{p}_+)$, $1\leq i\leq t$. According to the previous lemma this is equivalent to $p(\partial)  m_i(\partial)f)(0)$ for all relevant $p$ and $i$. From the polynomiality of $f$ follows that $m_i(\partial)f = 0$ for all $1\leq i\leq t.$
\end{proof}

Of course, the simple submodule $L(\lambda)$ is isomorphic to the simple quotient of $M(\lambda)$ by $J(\lambda)$. Reader interested in explicit form of the generators $m_i$ can consult chapter 10 of \cite{davidson_differential_1991}. The relationship between the Fischer inner product and the Shapovalov form was calculated by \cite{wachi_contravariant_1999}.

\subsection[Explicit singular vectors for su(m, n)]{Explicit singular vectors for $\mfrak{su}(m, n)$}\label{sec:invariant}

Since Verma modules are in particular highest weight modules, any homomorphism between them is determined by the image of the generating vector. This image is called singular vector it is, up to duality, precisely the invariant differential operator we seek. In this section we apply the recently developed F-method \cite{kobayashi_f-method_2014, kobayashi_branching_2015} to find (and classify) all singular vectors in Verma modules for $\mfrak{su}(m, n)$ which are induced from scalar representation.

\medskip

Let us consider the connected complex simple Lie group $G=\SL(n+m,\C)$, $n,m \in \N$, and its Lie algebra $\mfrak{g}=\mfrak{sl}(n+m,\C)$ given by
\begin{align}
  \mfrak{sl}(n+m,\C)=\{X \in M_{n+m,n+m}(\C);\, \tr X =0\}.
\end{align}
A Cartan subalgebra $\mfrak{h}$ of $\mfrak{g}$ is given by diagonal matrices
\begin{align}
  \mfrak{h} = \{\diag(a_1,a_2,\dots,a_{n+m});\, a_1,a_2,\dots,a_{n+m} \in \C,\, {\textstyle \sum_{i=1}^{n+m}} a_i=0\}.
\end{align}
For $i=1,2,\dots,n+m$ we define $\veps_i \in \mfrak{h}^*$ by $\veps_i(\diag(a_1,a_2,\dots,a_{n+m}))=a_i$. Then the root system of $\mfrak{g}$ with respect to $\mfrak{h}$ is $\Delta=\{\veps_i-\veps_j;\, 1\leq i \neq j \leq n+m\}$. A positive root system in $\Delta$ is $\Delta^+=\{\veps_i - \veps_j;\, 1 \leq i < j \leq n+m\}$ with the set of simple roots $\Pi=\{\alpha_1,\alpha_2,\dots,\alpha_{n+m-1}\}$, $\alpha_i= \veps_i - \veps_{i+1}$ for $i=1,2,\dots,n+m-1$. Finally, the fundamental weights are $\omega_i = \smash{\sum_{j=1}^i} \veps_j$ for $i=1,2,\dots,n+m-1$.

 The subset $\Sigma=\{\alpha_1,\dots,\alpha_{n-1},\alpha_{n+1},\dots,\alpha_{n+m-1}\}$ of $\Pi$ generates the root subsystem $\Delta_\Sigma$ in $\mfrak{h}^*$. We associate to $\Sigma$ the standard parabolic subalgebra $\mfrak{p}$ of $\mfrak{g}$ by $\mfrak{p}= \mfrak{l} \oplus \mfrak{u}$. By subscript $\mu$ we denote the $\mfrak{l}$-isotypical component of a module. The $\Sigma$-height $\htt_\Sigma(\alpha)$ of $\alpha \in \Delta$ is
\begin{align}
  \htt_\Sigma\!\big({\textstyle \sum_{i=1}^{n+m-1}} a_i \alpha_i\big) = a_n,
\end{align}
so $\mfrak{g}$ is a $|1|$-graded Lie algebra with respect to the grading given by $\mfrak{g}_i = \smash{\bigoplus}_{\alpha \in \Delta,\, \htt_\Sigma(\alpha)=i}\, \mfrak{g}_\alpha$ for $0 \neq i \in \Z$, and $\mfrak{g}_0= \mfrak{h} \oplus \smash{\bigoplus}_{\alpha \in \Delta,\, \htt_\Sigma(\alpha)=0}\, \mfrak{g}_\alpha$. Moreover, we have $\mfrak{l}= \mfrak{g}_0 \simeq \C \oplus \mfrak{sl}(n,\C) \oplus \mfrak{sl}(m,\C)$, $\mfrak{u}=\mfrak{g}_1 \simeq \Hom (\C^m,\C^n)$, and $\widebar{\mfrak{u}}=\mfrak{g}_{-1} \simeq \Hom (\C^n,\C^m)$.

The basis $\{f_{ij};\, i=1,2,\dots,m,\, j=1,2,\dots,n\}$ of the root spaces in the opposite nilradical $\widebar{\mfrak{u}}$ is given by
\begin{align}
  f_{ij} = \begin{pmatrix}
    0 & 0 \\
    E_{ij} & 0
  \end{pmatrix}\!,
\end{align}
where $E_{ij}$ is the $(m \times n)$-matrix having $1$ at the intersection of the $i$-th row and $j$-th column and $0$ elsewhere. Analogously, the basis $\{e_{ij};\, i=1,2,\dots,n,\, j=1,2,\dots,m\}$ of the root spaces in the nilradical $\mfrak{u}$ is given by
\begin{align}
  e_{ij} = \begin{pmatrix}
    0 & E_{ij} \\
    0 & 0
  \end{pmatrix}\!,
\end{align}
where $E_{ij}$ is the $(n \times m)$-matrix having $1$ at the intersection of the $i$-th row and $j$-th column and $0$ elsewhere. The Levi subalgebra $\mfrak{l}$ of $\mfrak{p}$ is the linear span of
\begin{align}
  h_c = \begin{pmatrix}
    {1 \over n} I_n & 0 \\
    0 & -{1 \over m} I_m
  \end{pmatrix}\!, \qquad
  h_{A,B} = \begin{pmatrix}
    A & 0 \\
    0 & B
  \end{pmatrix}\!,
\end{align}
where $A \in M_{n,n}(\C)$, $B \in M_{m,m}(\C)$ satisfy $\tr A=0$ and $\tr B=0$. Moreover,
the element $h_c$ is a basis of the center $\mfrak{z}(\mfrak{l})$ of $\mfrak{l}$.

Finally, the parabolic subgroup $P$ of $G$ with the Lie algebra $\mfrak{p}$ is defined by $P=N_G(\mfrak{p})$, where the parabolic subalgebra $\mfrak{p}$ of $\mfrak{g}$ is given by
\begin{align}
  \mfrak{p}=\left\{\begin{pmatrix}
    A & C \\
    0 & B
  \end{pmatrix}\!;
   A \in M_{n,n}(\C),\ B \in M_{m,m}(\C),\, C \in M_{n,m}(\C),\, \tr A + \tr B =0
  \right\}\!.
\end{align}
Any character $\lambda \in \Hom_P(\mfrak{p},\C)$ is given by
\begin{align}
  \lambda= \alpha \widetilde{\omega}_n
\end{align}
for some $\alpha \in \C$, where $\widetilde{\omega}_n\in \Hom_P(\mfrak{p},\C)$
is equal to $\omega_n\in \mfrak{h}^*$ regarded as trivially extended to
$\mfrak{p} =\mfrak{h} \oplus (\bigoplus_{\alpha \in \Delta_\Sigma} \mfrak{g}_\alpha) \oplus \mfrak{u}$.
The character $\rho \in \Hom_P(\mfrak{p},\C)$  is defined as
\begin{align}
  \rho={\textstyle {n+m \over 2}}\,\widetilde{\omega}_n.
\end{align}
For $\alpha \in \C$ we denote the character $\alpha \widetilde{\omega}_n$ by $\lambda_\alpha \in \Hom_P(\mfrak{p},\C)$.
\medskip

Let us introduce the notation
\begin{align}
  E^x= \sum_{k=1}^m \sum_{\ell=1}^n x_{k\ell} \partial_{x_{k\ell}}, \qquad \qquad E^y= \sum_{k=1}^m \sum_{\ell=1}^n y_{k\ell} \partial_{y_{k\ell}}
\end{align}
for the Euler homogeneity operators and
\begin{gather}
  R^x_A=  \sum_{i,j = 1}^n \sum_{k=1}^m a_{ij} x_{ki} \partial_{x_{kj}}, \qquad \qquad L^x_B= - \sum_{i,j=1}^m \sum_{k=1}^n b_{ij} x_{jk} \partial_{x_{ik}}, \label{eq:left right action x AGS} \\
  R^y_A= - \sum_{i,j=1}^n \sum_{k=1}^m a_{ij}y_{kj} \partial_{y_{ki}}, \qquad \qquad L^y_B= \sum_{i,j=1}^m \sum_{k=1}^n b_{ij}y_{ik} \partial_{y_{jk}}  \label{eq:left right action y AGS}
\end{gather}
for $A=(a_{ij}) \in M_{n,n}(\C)$, $B=(b_{ij}) \in M_{m,m}(\C)$. We set $L^y_{ij}=L^y_{E_{ij}}$ for $1 \leq i,j \leq m$ and $R^y_{ij}=R^y_{E_{ij}}$ for $1 \leq i,j \leq n$.
\medskip


\begin{theorem}\label{thm:action AGS}
Let $\lambda \in \Hom_P(\mfrak{p},\C)$ and let $(\sigma,\mathbb{V})$ be a $\mfrak{p}$-module. Let $\eus{A}_{\widebar{\mfrak{u}}}$ be the Weyl algebra of the vector space $\widebar{\mfrak{u}}$ generated by $\{x_\alpha,\partial_{x_\alpha};\, \alpha \in \Delta(\mfrak{u})\}$ and $\eus{A}_{\widebar{\mfrak{u}^*}}$ be the Weyl algebra of the vector space $\widebar{\mfrak{u}^*}$ generated by by $\{y_\alpha,\partial_{y_\alpha};\, \alpha \in \Delta(\mfrak{u})\}.$  Then the embedding of $\mfrak{g}$ into $\eus{A}_{\widebar{\mfrak{u}}} \otimes_\C \End \mathbb{V}_{\lambda+\rho}$ and into it's Fourier dual $\eus{A}_{\widebar{\mfrak{u}}^*} \otimes_\C \End \mathbb{V}_{\lambda-\rho}$ is given by
\begin{enumerate}
\item[1)]
\begin{align}
\begin{aligned}
  \pi_{\sigma_\lambda}(f_{ij})&=-\partial_{x_{ij}}, \\
  \hat{\pi}_{\sigma_\lambda}(f_{ij})&=-y_{ij}
\end{aligned}
\end{align}
for $i=1,2,\dots,m$ and $j=1,2,\dots,n$;
\item[2)]
\begin{align}
  \begin{aligned}
    \pi_{\sigma_\lambda}(h_c)&= \bigg({1 \over n} + {1 \over m}\bigg) E^x + \bigg(\lambda(h_c)+ {n+m \over 2}\bigg) + \sigma(h_c), \\[1mm]
    \pi_{\sigma_\lambda}(h_{A,B})&= R_A^x +L_B^x + \sigma(h_{A,B}), \\[1mm]
    \hat{\pi}_{\sigma_\lambda}(h_c)&= -\bigg({1 \over n} + {1 \over m}\bigg) E^y +  \bigg(\lambda(h_c) - {n+m \over 2}\bigg) + \sigma(h_c), \\[1mm]
    \hat{\pi}_{\sigma_\lambda}(h_{A,B})&= R_A^y + L_B^y + \sigma(h_{A,B})
  \end{aligned}
\end{align}
for all $A \in M_{n,n}(\C)$, $B \in M_{m,m}(\C)$ satisfying $\tr A =0$ and $\tr B = 0$;
\item[3)]
\begin{align}
  \begin{aligned}
    \pi_{\sigma_\lambda}(e_{ij})&= \sum_{k=1}^m \sum_{\ell=1}^n x_{ki} x_{j\ell} \partial_{x_{k\ell}}  + \bigg(\lambda(h_c)+ {n+m \over 2}\bigg) x_{ji} + \sigma(e_{ij}) + x_{ji}\sigma(h_c)
    \\ &\quad + \sum_{\ell=1}^n x_{j\ell} \sigma(h_{E_{i\ell}-{1\over n}I_n \delta_{i \ell},0})
     - \sum_{k=1}^m x_{ki} \sigma(h_{0,E_{kj}-{1 \over m}I_m \delta_{kj}}),  \\
    \hat{\pi}_{\sigma_\lambda}(e_{ij})&= \sum_{k=1}^m \sum_{\ell=1}^n y_{k\ell} \partial_{y_{ki}} \partial_{y_{j\ell}} - \bigg(\lambda(h_c)-  {n+m \over 2}\bigg) \partial_{y_{ji}} + \sigma(e_{ij}) - \partial_{y_{ji}} \sigma(h_c)
    \\ &\quad - \sum_{\ell=1}^n \partial_{y_{j\ell}} \sigma(h_{E_{i\ell}-{1\over n}I_n \delta_{i \ell},0})
     + \sum_{k=1}^m \partial_{y_{ki}} \sigma(h_{0,E_{kj}-{1 \over m}I_m \delta_{kj}})
  \end{aligned}
\end{align}
for $i=1,2,\dots,n$ and $j=1,2,\dots,m$.
\end{enumerate}
\end{theorem}

\begin{proof}
It follows immediately by straightforward caculations from \cite{krizka_algebraic_2017}.
\end{proof}


Let $\alpha \in \C$ and let $(\sigma,\C)$ be the trivial $\mfrak{p}$-module. Then we have an isomorphism
\begin{align}\label{eq:Phi isomorphism}
  \tau \circ \Phi_{\lambda_\alpha} \colon M(\C_{\lambda_\alpha-\rho}) \riso \C[\widebar{\mfrak{u}}^*] \otimes_\C \C_{\lambda_\alpha-\rho}
\end{align}
of $\mfrak{g}$-modules, where the action of $\mfrak{g}$ on $\C[\widebar{\mfrak{u}}^*] \otimes_\C \C_{\lambda_\alpha-\rho}$ is given by Theorem \ref{thm:action AGS}.
%Moreover, because $\C[\widebar{\mfrak{u}}^*] \otimes_\C \C_{\lambda_\alpha-\rho}$ and also $\Sol \subset \C[\widebar{\mfrak{u}}^*] \otimes_\C \C_{\lambda_\alpha-\rho}$ are semisimple $\mfrak{l}$-modules, we denote by $(\C[\widebar{\mfrak{u}}^*] \otimes_\C \C_{\lambda-\rho})_\mu$ and $\Sol(\mfrak{g},\mfrak{p};\C[\widebar{\mfrak{u}}^*] \otimes_\C \C_{\lambda-\rho})_\mu$ the $\mfrak{l}$-isotypic components of highest weight $\mu \in \mfrak{h}^*$.
\medskip

If we introduce an $\mfrak{l}$-module
\begin{align}
  \Sol = \{f \in \C[\widebar{\mfrak{u}}^*] \otimes_\C \!\mathbb{V}_{\lambda-\rho};\, \hat{\pi}_{\sigma_\lambda}(a)f=0\ \text{for all}\ a \in \mfrak{u}\},
\end{align}
then by \eqref{eq:Phi isomorphism} we obtain an isomorphism
\begin{align}
  \tau \circ \Phi_\lambda \colon M(\mathbb{V}_{\lambda-\rho})^\mfrak{u} \riso \Sol
\end{align}
of $\mfrak{l}$-modules where $\mcal{F}$ is the algebraic Fourier duality mapping uniquely determined by 
\begin{equation*}
  \mcal{F}(x_\alpha) = -\partial_{y_\alpha}, \qquad \mcal{F}(\partial_{x_\alpha})= y_\alpha.
\end{equation*}

\begin{proposition}\label{prop:C[u] decomposition AGS}
Let $\alpha \in \C$ and let $r=\min\{n,m\}$. Then we have
\begin{align}
   \C[\widebar{\mfrak{u}}^*] \otimes_\C \C_{\lambda_\alpha-\rho} \simeq \bigoplus_{\mu \in \Lambda^+_\alpha\!(\mfrak{p})} \mathbb{F}_\mu
\end{align}
as $\mfrak{l}$-modules, where $\mathbb{F}_\mu$ is the simple $\mfrak{l}$-module with highest weight $\mu \in \mfrak{h}^*$ and
\begin{align}
 \Lambda^+_\alpha(\mfrak{p}) = \bigg\{\sum_{i=1}^r a_i \omega_{n-i} + \sum_{i=1}^r a_i \omega_{n+i} + \bigg(\alpha - {n+m \over 2} - 2 \sum_{i=1}^r a_i\!\bigg) \omega_n;\, a_1,a_2,\dots,a_r \in \N_0 \bigg\}.
\end{align}
Moreover, the corresponding highest weight vector $v_\mu$ of $\mathbb{F}_\mu$ in $\C[\widebar{\mfrak{u}}^*] \otimes_\C \C_{\lambda_\alpha-\rho}$ with
\begin{align}
\mu=\sum_{i=1}^r a_i \omega_{n-i} + \sum_{i=1}^r a_i \omega_{n+i} + \bigg(\alpha - {n+m \over 2} - 2 \sum_{i=1}^r a_i\!\bigg) \omega_n
\end{align}
is $v_\mu=q_1^{a_1}q_2^{a_2} \dots q_r^{a_r}$, where
\begin{align} \label{eq:generators hwv AGS}
  q_k = \det\!
  \begin{pmatrix}
    y_{1,n-k+1} & y_{1,n-k+2} & \dots & y_{1,n}\\
    y_{2,n-k+1} & y_{2,n-k+2} & \dots & y_{2,n} \\
    \vdots & \vdots & \ddots & \vdots \\
    y_{k,n-k+1} & y_{k,n-k+2}&  \dots & y_{k,n}
  \end{pmatrix}
\end{align}
for $k=1,2,\dots,r$.
\end{proposition}

\begin{proof}First of all, we need to verify that the polynomials $q_k$ are the highest weight vectors with highest weights $\omega_{n-k}+\omega_{n+k}+\big(\alpha - {n + m \over 2} -2\big)\omega_n$ for $k=1,2,\dots,r$. Using \eqref{eq:left right action y AGS} it easily follows that $L^y_{ij}q_k=0$ for $j=k+1,k+2,\dots,m$ and $L^y_{ij}q_k$ is the polynomial which arises from $q_k$ replacing the $j$-th row $(y_{j,n-k+1},y_{j,n-k+2},\dots,y_{j,n})$  with the $i$-th row $(y_{i,n-k+1},y_{i,n-k+2},\dots,y_{i,n})$ in the corresponding matrix \eqref{eq:generators hwv AGS} for $j=1,2,\dots,k$. Analogously, we have that $-R^y_{ij}q_k=0$ for $i=1,2,\dots,n-k$ and $-R^y_{ij}q_k$ is the polynomial which arises from $q_k$ replacing the $i$-th column \smash{$(y_{1,i},y_{2,i},\dots,y_{k,i})^{\rm T}$} with the $j$-th column \smash{$(y_{1,j},y_{2,j},\dots,y_{k,j})^{\rm T}$} in the corresponding matrix \eqref{eq:generators hwv AGS} for $i=n-k+1,n-k+2,\dots,n$. Therefore, we obtain $R^y_{ij}q_k=0$ for $1 \leq i < j \leq n$ and $L^y_{ij}q_k=0$ for $1 \leq i < j \leq m$, which gives us that $q_k$ is annihilated by $\mfrak{l} \cap \mfrak{n}$. Furthermore, since we have
\begin{align*}
  \hat{\pi}_{\lambda_\alpha}\!(h_{A,B})q_k &= \sum_{i=1}^n a_i R^y_{ii}q_k + \sum_{j=1}^m b_j L^y_{jj}q_k = -\sum_{i=n-k+1}^n a_i q_k + \sum_{j=1}^k b_j q_k
  = \bigg(\sum_{i=1}^{n-k} a_i + \sum_{j=1}^k b_j \bigg)q_k \\
  &= \omega_{n-k}(h_{A,B})q_k + \omega_{n+k}(h_{A,B})q_k+ \bigg(\alpha-{n+m \over 2}-2\bigg)\omega_n(h_{A,B})q_k
\end{align*}
for all $A=\diag(a_1,a_2,\dots,a_n)$, $B=\diag(b_1,b_2,\dots,b_m)$ satisfying $\tr A=0$, $\tr B=0$, and
\begin{align*}
  \hat{\pi}_{\lambda_\alpha}\!(h_c)q_k&= -\bigg({1 \over n} + {1 \over m}\bigg)E^yq_k + \bigg(\alpha -{n+m \over 2}\bigg)q_k = -k\bigg({1 \over n} + {1 \over m}\bigg)q_k+ \bigg(\alpha -{n+m \over 2}\bigg)q_k \\
  &= \omega_{n-k}(h_c)q_k + \omega_{n+k}(h_c)q_k+\bigg(\alpha-{n+m \over 2}-2\bigg)\omega_n(h_c)q_k,
\end{align*}
we obtain that $q_k$ is a weight vector with respect to $\mfrak{l} \cap \mfrak{h}$. Therefore, the polynomial $q_k$ is the highest weight vector with highest weight $\omega_{n-k}+ \omega_{n+k} + \big(\alpha-{n+m \over 2}-2\big)\omega_n$.

As an immediate consequence of the previous result, we get that the polynomial $q_1^{a_1}q_2^{a_2}\dots q_r^{a_r}$ with $a_1,a_2,\dots,a_r \in \N_0$ is the highest weight vector with highest weight
\begin{align*}
\mu=\sum_{k=1}^r a_k \omega_{n-k} + \sum_{k=1}^r a_k \omega_{n+k} + \bigg(\alpha - {n+m \over 2} - 2\sum_{k=1}^n a_k\bigg) \omega_n
\end{align*}
and generates a finite-dimensional simple $\mfrak{l}$-submodule of $\C[\widebar{\mfrak{u}}^*] \otimes_\C \C_{\lambda_\alpha-\rho}$ isomorphic to $\mathbb{F}_\mu$, since $\C[\widebar{\mfrak{u}}^*] \otimes_\C \C_{\lambda_\alpha-\rho}$ is a semisimple $\mfrak{l}$-module decomposing into finite-dimensional simple $\mfrak{l}$-submodules. This gives us an injective homomorphism
\begin{align*}
  \bigoplus_{\mu \in \Lambda^+_\alpha\!(\mfrak{p})} \mathbb{F}_\mu \rarr \C[\widebar{\mfrak{u}}^*] \otimes_\C \C_{\lambda_\alpha-\rho}
\end{align*}
of $\mfrak{l}$-modules, which is an isomorphism as follows from \cite{goodman_symmetry_2009}.
\end{proof}

\begin{lemma}\label{lem:solution space AGS}
Let $\alpha \in \C$ and $\mu \in \Lambda^+_\alpha(\mfrak{p})$. Then we have
\begin{align*}
 \Sol_\mu \simeq
 \begin{cases}
   \mathbb{F}_\mu & \text{for $\alpha \in \C$, $\mu = \big(\alpha-{n+m \over 2}\big)\omega_n$}, \\[1mm]
   \mathbb{F}_\mu & \text{$\!\begin{gathered}\text{for $\alpha = a-k+{\textstyle {n+m \over 2}}$, $\mu=a \omega_{n-k} + a\omega_{n+k}-(a+k)\omega_n$,}\\[1mm] \text{$a \in \N$, $k \in \{1,2,\dots,r\}$} \hfill \end{gathered}$} \\[1mm]
   0 & \text{otherwise}.
 \end{cases}
\end{align*}
\end{lemma}

\begin{proof}Using Proposition \ref{prop:C[u] decomposition AGS} we know that $\C[\widebar{\mfrak{u}}^*] \otimes_\C \C_{\lambda_\alpha-\rho}$ is a multiplicity free $\mfrak{l}$-module. Moreover, since we have $\Sol_\mu \subset (\C[\widebar{\mfrak{u}}^*] \otimes_\C \C_{\lambda_\alpha-\rho})_\mu \simeq \mathbb{F}_\mu$ for $\mu \in \Lambda^+_\alpha(\mfrak{p})$, we obtain that $\smash{\Sol_\mu}$ is either the zero $\mfrak{l}$-module or isomorphic to the $\mfrak{l}$-module $\mathbb{F}_\mu$. Therefore, it is enough to verify, whether the highest weight vector $v_\mu$ of $(\C[\widebar{\mfrak{u}}^*] \otimes_\C \C_{\lambda_\alpha-\rho})_\mu$ is contained in $\Sol$ or not. Let $\mu \in \Lambda^+_\alpha(\mfrak{p})$ and let us assume that the highest weight vector
\begin{align*}
  v_\mu = q_1^{a_1}q_2^{a_2}\dots q_r^{a_r}
\end{align*}
of $(\C[\widebar{\mfrak{u}}^*] \otimes_\C \C_{\lambda_\alpha-\rho})_\mu$ with $a_1,a_2,\dots a_r \in \N_0$ belongs to $\Sol$. For greater clarity, we denote by $q^a$ the polynomial $q_1^{a_1}q_2^{a_2}\dots q_r^{a_r}$ for $a=(a_1,a_2,\dots,a_r) \in \N_0^r$ and by $1_k \in \N_0^r$ the $r$-tuple having $1$ at the $k$-th position and $0$ otherwise. Furthermore, we have
\begin{align*}
  \hat{\pi}_{\lambda_\alpha}\!(e_{ij})&= \sum_{k=1}^m \sum_{\ell=1}^n y_{k\ell} \partial_{y_{ki}} \partial_{y_{j\ell}} - \bigg(\alpha-  {n+m \over 2}\bigg) \partial_{y_{ji}}= \sum_{k=1}^m  \partial_{y_{ki}} L^y_{kj} - \bigg(\alpha - {n-m \over 2} \bigg) \partial_{y_{ji}}
\end{align*}
for $i=1,2,\dots,n$ and $j=1,2,\dots,m$.
\smallskip

If $v_\mu =1$, then $\hat{\pi}_{\lambda_\alpha}\!(e_{ij})v_\mu=0$ for all $i=1,2,\dots,n$, $j=1,2,\dots,m$ and $\alpha \in \C$. Therefore, we have $\Sol_\mu \simeq \mathbb{F}_\mu$ for $\mu=\big(\alpha-{n+m \over 2}\big)\omega_n$ and $\alpha \in \C$.
\smallskip

Now, if $v_\mu \neq 1$, then there exists an index $s \in \{1,2,\dots,r\}$ satisfying $a_s \neq 0 $ and $a_\ell=0$ for $\ell=s+1,s+2,\dots,r$. Since $\partial_{y_{k,n-s+1}}q_\ell =0$ and $L^y_{ks}q_\ell=0$ for $\ell=1,2,\dots,s-1$ and $k=1,2,\dots,m$, we may write
\begin{align*}
   \hat{\pi}_{\lambda_\alpha}\!(e_{n-s+1,s})v_\mu &= \sum_{k=1}^m  \partial_{y_{k,n-s+1}} L^y_{ks}q^a - \big(\alpha - {\textstyle {n-m \over 2}} \big) \partial_{y_{s,n-s+1}}q^a \\
  &= \sum_{k=1}^m  a_s\partial_{y_{k,n-s+1}} q^{a-1_s}L^y_{ks}q_s - \big(\alpha - {\textstyle {n-m \over 2}} \big) \partial_{y_{s,n-s+1}}q^a \\
  &= \sum_{k=s}^m  a_s\partial_{y_{k,n-s+1}} q^{a-1_s}L^y_{ks}q_s - \big(\alpha - {\textstyle {n-m \over 2}} \big) \partial_{y_{s,n-s+1}}q^a,
\end{align*}
where we used that $L^y_{ks}q_s=0$ for $k<s$ in the last equality. Further, using $\partial_{y_{k,n-s+1}}q_\ell=0$ for $k > \ell$, we obtain
\begin{align*}
 \hat{\pi}_{\lambda_\alpha}\!(e_{n-s+1,s})v_\mu&= a_s\partial_{y_{s,n-s+1}} q^{a-1_s}L^y_{ss}q_s + \sum_{k=s+1}^m  a_s q^{a-1_s}\partial_{y_{k,n-s+1}}L^y_{ks}q_s \\
 & \quad - \big(\alpha - {\textstyle {n-m \over 2}} \big) \partial_{y_{s,n-s+1}}q^a \\
 &=a_s \partial_{y_{s,n-s+1}} q^a +(m-s)a_sq^{a_s-1_s}\partial_{y_{s,n-s+1}}q_s - \big(\alpha -{\textstyle {n-m \over 2}} \big) \partial_{y_{s,n-s+1}}q^a \\
 &= \big(a_s-s-\alpha+{\textstyle {n+m \over 2}}\big)\partial_{y_{s,n-s+1}}q^a,
\end{align*}
where we used $L^y_{ss}q_s=q_s$ and $\partial_{y_{k,n-s+1}}L^y_{ks}q_s=\partial_{y_{s,n-s+1}}q_s$ for $k>s$. As $\partial_{y_{s,n-s+1}}q^a \neq 0$ and $\hat{\pi}_{\lambda_\alpha}\!(e_{n-s+1,s})v_\mu=0$, we get
\begin{align}
  \alpha=a_s-s+{n+m \over 2}. \label{eq:weight alpha AGS}
\end{align}
Our next step is to show that $a_\ell=0$ for $\ell=1,2,\dots,s-1$. Let us assume that there exists an index $t \in \{1,2,\dots,s-1\}$ satisfying $a_t \neq 0$ and $a_\ell=0$ for $\ell=t+1,t+2,\dots,s-1$. Analogously as in the previous part, we may write
\begin{align*}
  \hat{\pi}_{\lambda_\alpha}\!(e_{n-t+1,t})v_\mu&= \sum_{k=1}^m  \partial_{y_{k,n-t+1}} L^y_{kt}q^a - \big(\alpha - {\textstyle {n-m \over 2}} \big)\partial_{y_{t,n-t+1}}q^a \\
  &= \sum_{k=1}^m \big(a_s\partial_{y_{k,n-t+1}}q^{a-1_s} L^y_{kt}q_s + a_t\partial_{y_{k,n-t+1}}q^{a-1_t} L^y_{kt}q_t \big)- \big(\alpha - {\textstyle {n-m \over 2}} \big)\partial_{y_{t,n-t+1}}q^a \\
  & =\sum_{k=s+1}^m a_s\partial_{y_{k,n-t+1}}q^{a-1_s} L^y_{kt}q_s +\sum_{k=t+1}^m  a_t\partial_{y_{k,n-t+1}}q^{a-1_t} L^y_{kt}q_t \\
  & \quad +a_s\partial_{y_{t,n-t+1}}q^{a-1_s} L^y_{tt}q_s + a_t\partial_{y_{t,n-t+1}}q^{a-1_t} L^y_{tt}q_t - \big(\alpha - {\textstyle {n-m \over 2}} \big)\partial_{y_{t,n-t+1}}q^a \\
  &= \sum_{k=s+1}^m a_sq^{a-1_s} \partial_{y_{k,n-t+1}}L^y_{kt}q_s +\sum_{k=t+1}^m  a_tq^{a-1_t} \partial_{y_{k,n-t+1}}L^y_{kt}q_t \\
  & \quad + \sum_{k=t+1}^s a_t (\partial_{y_{k,n-t+1}}q^{a-1_t}) L^y_{kt}q_t +\big(a_s+a_t-\alpha+{\textstyle {n-m\over 2}}\big) \partial_{y_{t,n-t+1}}q^a.
\end{align*}
Using $\partial_{y_{k,n-t+1}}L^y_{kt}q_s=\partial_{y_{t,n-t+1}}q_s$ for $k>s$ and $\partial_{y_{k,n-t+1}}L^y_{kt}q_t=\partial_{y_{t,n-t+1}}q_t$ for $k>t$, we get
\begin{align*}
  \hat{\pi}_{\lambda_\alpha}\!(e_{n-t+1,t})v_\mu&= (m-s) a_sq^{a-1_s} \partial_{y_{t,n-t+1}}q_s +(m-t) a_tq^{a-1_t} \partial_{y_{t,n-t+1}}q_t \\
  & \quad + \sum_{k=t+1}^s a_s a_t q^{a-1_s-1_t}(\partial_{y_{k,n-t+1}}q_s) L^y_{kt}q_t +\big(a_s+a_t-\alpha+{\textstyle {n-m\over 2}}\big) \partial_{y_{t,n-t+1}}q^a  \\
  & = a_sa_t q^{a-1_s} \partial_{y_{t,n-t+1}} q_s +a_t(a_t+s-t) q^{a-1_t} \partial_{y_{t,n-t+1}} q_t \\
  & \quad + \sum_{k=t+1}^s a_s a_t q^{a-1_s-1_t}(\partial_{y_{k,n-t+1}}q_s) L^y_{kt}q_t,
\end{align*}
where we used \eqref{eq:weight alpha AGS} in the last equality. Since $\hat{\pi}_{\lambda_\alpha}\!(e_{n-t+1,t})v_\mu=0$, we acquire the equation
\begin{align*}
q^{a-1_s-1_t}\!\bigg(a_ta_s q_t \partial_{y_{t,n-t+1}} q_s + a_t(a_t+s-t)q_s \partial_{y_{t,n-t+1}} q_t + \sum_{k=t+1}^s a_sa_t (\partial_{y_{k,n-t+1}}q_s) L^y_{kt}q_t\!\bigg) = 0,
\end{align*}
which gives us
\begin{align*}
  a_ta_s q_t \partial_{y_{t,n-t+1}} q_s + a_t(a_t+s-t)q_s \partial_{y_{t,n-t+1}} q_t + \sum_{k=t+1}^s a_sa_t (\partial_{y_{k,n-t+1}}q_s) L^y_{kt}q_t = 0.
\end{align*}
Applying the operator $\partial_{y_{t,n-t+1}}$ to the previous equation, we get
\begin{align*}
  a_t(a_s+a_t+s-t) (\partial_{y_{t,n-t+1}} q_s)(\partial_{y_{t,n-t+1}} q_t) = 0,
\end{align*}
because $\partial_{y_{t,n-t+1}} (\partial_{y_{k,n-t+1}}q_s) L^y_{kt}q_t=0$ for $k=t+1,t+2,\dots,s$. Moreover, we have $\partial_{y_{t,n-t+1}} q_s \neq 0$, $\partial_{y_{t,n-t+1}} q_t \neq 0$ and $a_s +a_t +s-t >0$, hence we obtain $a_t = 0$, which is a contradiction with the choice of the index $t$. Therefore, we have $v_\mu = q_s^{a_s}$ for $\mu= a_s \omega_{n-s} +a_s \omega_{n+s} - (a_s+s)\omega_n$ and $\alpha=a_s -s + {n + m \over 2}$. It remains to show that $\hat{\pi}_{\lambda_\alpha}\!(e_{ij})v_\mu=0$ for all $i=1,2,\dots,n$, $j=1,2,\dots,m$ and $\alpha=a_s-s+{n+m \over 2}$.
We may write
\begin{align*}
  \hat{\pi}_{\lambda_\alpha}\!(e_{ij})v_\mu &= \sum_{k=1}^m  \partial_{y_{ki}} L^y_{kj}q_s^{a_s} - (a_s-s+m) \partial_{y_{ji}}q_s^{a_s} \\
  &= \sum_{k=1}^m  a_s \partial_{y_{ki}} q_s^{a_s-1}L^y_{kj}q_s - (a_s-s+m) \partial_{y_{ji}}q_s^{a_s} \\
  & =a_s \partial_{y_{ji}} q_s^{a_s-1}L^y_{jj}q_s + \sum_{k=s+1}^m  a_s \partial_{y_{ki}} q_s^{a_s-1}L^y_{kj}q_s - (a_s-s+m) \partial_{y_{ji}}q_s^{a_s} \\ &= a_s \partial_{y_{ji}} q_s^{a_s} +  \sum_{k=s+1}^m  a_s  q_s^{a_s-1} \partial_{y_{ki}}L^y_{kj}q_s - (a_s-s+m) \partial_{y_{ji}}q_s^{a_s} \\
  &=  \sum_{k=s+1}^m  a_s  q_s^{a_s-1} \partial_{y_{ji}}q_s -(m-s)\partial_{y_{ji}}q_s^{a_s} \\
  &=(m-s)\partial_{y_{ji}}q_s^{a_s}-(m-s)\partial_{y_{ji}}q_s^{a_s}=0,
\end{align*}
where we used $L^y_{kj}q_s=0$ if $j>s$ or if $k \leq s$ and $k\neq j$ in the third equality and $\partial_{y_{ki}}L^y_{kj}q_s=\partial_{y_{ji}} q_s$ for $k > s$ in the fifth equality. Hence, we have \smash{$\Sol_\mu \simeq \mathbb{F}_\mu$} for $\mu=a \omega_{n-k} + a\omega_{n+k}-(a+k)\omega_n$ and $\alpha = a-k + {n+m \over 2}$.
\end{proof}

\begin{theorem}\label{thm:singular vectors decomposition AGS}
We have
\begin{align*}
  \tau \circ \Phi_{\lambda_\alpha+\rho} \colon M(\C_{\lambda_\alpha})^\mfrak{u} \riso
  \begin{cases}
    \mathbb{F}_{\mu_{\alpha,r}} &  \text{if $\alpha+r \notin \N$}, \\[2mm]
    \mathbb{F}_{\mu_{\alpha,r}} \oplus \bigoplus\limits_{k=0}^{\min\{r-1,\alpha+r-1\}} \mathbb{F}_{\mu_{\alpha,k}} & \text{if $\alpha + r \in \N$,}
  \end{cases}
\end{align*}
where $\mu_{\alpha,k}=(\alpha+r-k)\omega_{n-r+k}+(\alpha+r-k)\omega_{n+r-k}-(\alpha+2r-2k)\omega_n$ for $k=0,1,\dots,r$.
\end{theorem}

\begin{proof}The decomposition of the space of singular vectors $M(\mathbb{C}_{\lambda_\alpha})^\mfrak{u}$ is a straightforward consequence of Lemma \ref{lem:solution space AGS}.
\end{proof}

If $\alpha+r \in \N$, then Theorem \ref{thm:singular vectors decomposition AGS} ensures the existence of a non-trivial homomorphism
\begin{align}
  \varphi \colon M(\mathbb{F}_{\mu_{\alpha,k}}) \rarr M(\C_{\lambda_\alpha})
\end{align}
of generalized Verma modules for $k=0,1,\dots,\min\{r-1,\alpha+r-1\}$, where the simple $\mfrak{l}$-module $\mathbb{F}_{\mu_{\alpha,k}}$ is extended to a $\mfrak{p}$-module by $\mfrak{u}$ acting trivially, uniquely determined by a homomorphism
\begin{align}
  \varphi_0 \colon \mathbb{F}_{\mu_{\alpha,k}} \rarr M(\C_{\lambda_\alpha})
\end{align}
of $\mfrak{p}$-modules, which is defined by
\begin{align}
  \varphi_0(v_{\mu_{\alpha,k}})= q^{\alpha+r-k}_{r-k}(f_{ij}) v_\alpha,
\end{align}
where $v_\alpha \in M(\C_{\lambda_\alpha})$ is the highest weight vector of $M(\C_{\lambda_\alpha})$ and $v_{\mu_{\alpha,k}} \in \mathbb{F}_{\mu_{\alpha,k}}$ is the highest weight vector of $\mathbb{F}_{\mu_{\alpha,k}}$. Finally, let us remark that the same approach works also in the $\mfrak{sp}$ and $\mfrak{so}^*$ cases \cite{krizka_invariant_2017}. 