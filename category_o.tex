\chapter[Category O]{Category $\mathcal{O}$}

\section{Borel and parabolic subalgebras}

\section{Generalized Verma modules and the Shapovalov form}


For any $\lambda\in\lie{h}^*$ we denote by $\C_\lambda$ the one dimensional representation of $\lie{h}$ on with character $\lambda$. We extend the action to $\lie{b}^- := \lie{n}^-\oplus\lie{h}$ by letting $\lie{n}^-$ act trivially. The \emph{Verma module} $N(\lambda)$ is then defined as $N(\lambda) := \lie{U(g)}\otimes_{\lie{U(b^-)}} \C_\lambda$. We will denote its highest weight vector by $v_\lambda$.

Let $\lambda$ be a $\Delta_c^+$ dominant and integral weight and denote by $F(\lambda)$ the finite dimensional irreducible $\lie{k}$-module.  We extend any irreducible representation of $K_\mathbb{C}$ to $P$ and to $\overline{P}$ by letting $\lie{p}_+$ and $\lie{p}_-$ act trivially. The \emph{generalized Verma module} $M(\lambda)$ is defined as $M(\lambda) = \lie{U}(\lie{g}) \otimes_{\lie{U}(\oppar)} F(\lambda)$. It is well known and easy to prove that $M(\lambda)$ contains a maximal nontrivial submodule $J(\lambda)$ and we denote by $L(\lambda)$ the corresponding irreducible quotient of $M(\lambda)$. Since the nilradical $\lie{p}_-$ of $\oppar$ is abelian and acts trivially on $F(\lambda)$, we have that $M(\lambda) \simeq S(\lie{p}_+)\otimes F(\lambda)$ as $K_\C$ representations, where $S(\lie{p}_+)$ is the symmetric algebra  over the Lie algebra $\lie{p}_+$.
%The canonical $\overline{P}$-invariant filtration of $M(\lambda) = \bigcup_{k\geq 0} M^k(\lambda)$ has filtration components $M^k(\lambda)= \{\epsilon^{i_1}\cdots\epsilon^{i_j}\otimes v\, | v \in F(\lambda), j \leq k \}$.
In the case of $M(\lambda)$, the geometric weight corresponds to the polynomial degree shifted by the weight of $F(\lambda)$.% and the canonical $\overline{P}$-invariant filtration is the same as the filtration induced by the geometric weight.

Let $\sigma$ be involutive antiautomorphism on $\lie{U(g)}$ such that it's restriction on the real form $\lie{g}_0$ is $-\id$. With our choice of Cartan subalgebra $\lie{h}$ we have that $\sigma: X_\beta \mapsto X_{-\beta}$ for $X_\beta \in \lie{g}_\beta$ and $\sigma: h_\alpha \mapsto h_\alpha$ for $h_\alpha \in \lie{h}$.

Let $P: \lie{U(g)} \to \lie{U(h)}$ be the projection defined by the splitting \[\lie{U(g)} = \lie{U(h)}\oplus \left( \lie{n_-U(g)} + \lie{U(g)n_+} \right)\]

\begin{definition}
  The \emph{universal Shapovalov form} on $\lie{U(g)}$ is defined as \[\langle u_1 , u_2 \rangle = P(\sigma(u_1)u_2).\] It is a bilinear form on $\lie{U(g)}$ with values in $\lie{U(h)} = S(\lie{h})$.

  For $\lambda\in\lie{h}^*$ we define the \emph{Shapovalov form on the Verma module} $N(\lambda)$ by \[\langle u_1 v_\lambda , u_2 v_\lambda \rangle = P(\sigma(u)v)(\lambda).\]
\end{definition}

It is easy to see that $\langle u\cdot v,v' \rangle = \langle v,\sigma(u)\cdot v' \rangle$ for all $v,v'\in M_\lambda$ and for all $u\in \lie{U(g)}$. Forms with such a property are called \emph{contravariant forms}.

The following proposition and its proof can be found e.g. in \cite{humphreys}.

\begin{proposition}
Contravariant forms have the following properties:
\begin{enumerate}
 \item If the $\lie{U(g)}$-module $M$ has a contravariant form $(v,v')_M$, then the weight spaces are orthogonal. I.e. $( M_\mu,M_\nu)=0$ whenever $\mu\neq\nu$ in $\lie{h}^*$.
 \item Suppose $M = \lie{U(g)} \cdot v$ is a highest weight module generated by a maximal vector $v$ of weight $\lambda$. If $M$ has a nonzero contravariant form, then the form is uniquely determined up to a scalar multiple by the (nonzero!) value $( v, v)_M$.
 \item  If $\lie{U(g)}$-modules $M_1, M_2$ have contravariant forms $(v, v')_{M_1}$ and $(w,w')_{M_2}$ , then $M := M_1\otimes M_2$ also has a contravariant form, given by \[(v\otimes w, v'\otimes w')_M := (v, v')_{M_1} (w,w')_{M_2}.\] In case both of the forms are nondegenerate, so is the product form.
 \item If $M$ has a contravariant form and $N$ is a submodule, the orthogonal complement $N^\perp := \{v\in M | (v, v')_M = 0 \text{ for all } v' \in N \}$ is also a submodule.
\end{enumerate}
\end{proposition}
\begin{proof}
\begin{enumerate}
 \item We use the fact that $\sigma(h) = h$ for $h\in\lie{h}$. Let $v$ be a vector of weight $\mu$ and let $v'$ be a vector of weight $\nu$. The for any $h\in\lie{h}$ we have
 \[
  \mu(h)(v,v')_M = (h\cdot v,v')_M = (v,\sigma(h)\cdot v')_M = (v,h\cdot v')_M = \nu(h)(v,v')_M.
 \]
 Since $v,v'$ were arbitrary we must have $(v,v')=0$ for $\mu\neq \nu$.
 \item In view of the already proven point it suffices to look at values of the form on a weight
space $M_\mu$. Typical vectors $v, v' \in M_\mu$ can be written as $u\cdot v, u'\cdot v$ for suitable $u, u'\in\lie{U(n)}$. Note that since $u$ takes $M_\lambda$ into $M_\mu$, the element $\sigma(u) \in \lie{U(n^-)}$ takes $M_\mu$ into $M_\lambda$ (which is spanned by $v$). Then $(v, v')_M = (u\cdot v, u'\cdot v)_M = (v, \sigma(u)u'\cdot v)_M$, which is a scalar multiple of $(v, v)_M$ depending just on the action of $\lie{U(g)}$ and not on the choice of the form.

The remaining points are elementary.
\end{enumerate}
\end{proof}

Important consequence of this is the following lemma.
\begin{lemma}
The maximal submodule of $N(\lambda)$ is the radical of the Shapovalov form.
\end{lemma}
\begin{proof}
Let $v\in J(\lambda)$  be arbitrary. Then $\langle v, v_\lambda \rangle = 0$ since clearly $v$ and $v_\lambda$ have different weights. Since $J(\lambda)^\perp$ is a submodule of $N(\lambda)$ containing the generating vector $v_\lambda$ the statement follows.
\end{proof}

The generalized Verma module $M(\lambda)$ is a quotient of the Verma module $N(\lambda)$. Hence the simple quotient $L(\lambda)$ of $M(\lambda)$ is also a quotient of $N(\lambda)$ by its maximal submodule. We get induced contravariant forms on $L(\lambda)$ and $M(\lambda)$  which we also call Shapovalov forms and we denote them by a same symbol.

\section{Equivalences of Enright and Shelton}