\chapter{Lie algebras and their modules}

In this chapter we fix some notation and review several structural results about Lie algebras. First we start with some basic definitions concerning complex Lie algebras.

\medskip

\section{Complex simple Lie algebras}

Let $\lie{g}^{1} = \lie{g}$ and define inductively the so called \emph{lower central series} of $\lie{g}$ by $\lie{g}^{k+1} = [\lie{g},\lie{g}^{k}]$.  A subalgebra $\lie{n}$ of $\lie{g}$ is called \emph{nilpotent} if $\lie{n}^{k} = 0$ for some $k\in\N$.

Let $\lie{g}^{(1)} = \lie{g}$ and define inductively the so called \emph{derived series} of $\lie{g}$ by $\lie{g}^{(k+1)} = [\lie{g}^{(k)},\lie{g}^{(k)}]$.  A subalgebra $\lie{b}$ of $\lie{g}$ is called \emph{solvable} if $\lie{b}^{(k)} = 0$ for some $k\in\N$. A \emph{Borel subalgebra} $\lie{b}$ of $\lie{g}$ is any maximal solvable subalgebra of $\lie{g}$.


We denote by $\ad(X)$ the Lie algebra homomorphism $\lie{g}\to \lie{g}$ given by $Y\mapsto [X,Y]$. This is the \emph{adjoint representation} of $\lie{g}$.

A  Lie algebra $\lie{g}$ is said to be \emph{semisimple} if it has no nonzero solvable ideal and it is called \emph{simple} if $\lie{g} = [\lie{g},\lie{g}]$ and the only ideals of $\lie{g}$ are $0$ and $\lie{g}$. Any semisimple Lie algebra is a direct sum of simple Lie algebras. An \emph{semisimple element} $X$ of $\lie{g}$ is an element of $\lie{g}$ such that $\ad X: \lie{g} \to \lie{g}$ is diagonalizable.

A \emph{Cartan subalgebra} of a complex Lie algebra $\lie{g}$ is a maximal commutative subalgebra $\lie{h}$ of $\lie{g}$ consisting of semisimple elements. For a real Lie algebra $\lie{g}_\R$ we define its Cartan subalgebra $\lie{h}_\R$ as such that its complexification $\lie{h}_\C$ is a Cartan subalgebra of the complexification of $\lie{g}_\R$.

A \emph{reductive} Lie algebra $\lie{g}$ is a Lie algebra that decomposes as $\lie{g} = \lie{z(g)} \oplus \lie{g}_{ss}$, where $\lie{z(g)}$ is the center of $\lie{g}$ and the algebra $\lie{g}_{ss} = [\lie{g},\lie{g}]$ is the \emph{semisimple part} of $\lie{g}$. Of course, as its name suggest, the algebra $\lie{g}_{ss}$ is semisimple.

Let $\lie{g}$ be a real or complex Lie algebra let $B$ be a bilinear form $\lie{g}\otimes \lie{g} \to \k$. We say that $B$ is invariant if all $\ad X$, $X\in\lie{g}$ are skew-symmetric operators relative to $B$, i.e.
\[
 B([X,Y],Z) = -B(Y,[X,Z]), \quad \forall X,Y,Z \in \lie{g}.
\]
The most important invariant form is the so called \emph{Killing form} which we will denote $B$ and which exists for any Lie algebra $\lie{g}$. It is defined by
\[
 B(X,Y) = \tr(\ad(X) \circ ad(Y)).
\]
If $\phi:\lie{g} \to \lie{g}$ is any automorphism of the Lie algebra $\lie{g}$, then by definition $\ad(X) \circ \phi = \phi \circ \ad(X)$ for any $X\in\lie{g}$. This implies
\[
 \ad(\phi(X)) \circ \ad(\phi(Y)) = \phi \circ \ad(X)\circ\ad(Y)\circ\phi^{-1}
\]
and thus $B(\phi(X),\phi(Y)) = B(X,Y)$. The Cartan criterion states that a Lie algebra $\lie{g}$ is semisimple if and only if the Killing form $B$ is nondegenerate.


Let $\lie{g}$ be a complex semisimple Lie algebra and choose a Cartan subalgebra $\lie{h} \leq \lie{g}$. Any two Cartan subalgebras of $\lie{g}$ are conjugate by an inner automorphism of $\lie{g}$. The roots $\roots$ of $(\lie{g},\lie{h})$ are linear functionals $\alpha: \lie{h} \to \C$ such that the corresponding \emph{root space} $\lie{g}_\alpha = \{ X\in G \,|\,\forall H\in \lie{h}: [H,X] = \alpha(H) X \}$ is nonempty. In other words, roots are the weights of the adjoint representation $\ad:\lie{g}\to\lie{g}$. The roots form a finite subset $\roots \subset \lie{h}^*$ and we have the \emph{root space decomposition}
\[
 \lie{g} = \lie{h} \oplus \sum_{\alpha \in \roots} \lie{g}_\alpha.
\]

If $X\in\lie{g}_\alpha$ and $Y\in\lie{g}_\beta$, then $B(X,Y) = 0$ unless $\alpha = -\beta$. Thus the Killing form induces a nondegenerate pairing on $\lie{g}_\alpha \otimes \lie{g}_{-\alpha}\to \C$. The restriction of $B$ to $\lie{h}$ is nondegenerate and, in particular, for each linear functional $\lambda \in \lie{h}^*$, there is a unique
 element $\widetilde{H_\lambda} \in \lie{h}$ such that $\lambda(H) = B(H,\widetilde{H_\lambda})$ for all $H\in\lie{h}$. We can define the bilinear form $(\, , \,)$ on $\lie{h}^*$ by
\[
 (\lambda,\mu) = B(\widetilde{H_\lambda}, \widetilde{H_\mu}).
\]
The restriction of $(\, , \, )$ to the real span of $\roots$ is positive definite (and in particular has real values).

Let us summarize the properties of $\roots$
\begin{enumerate}
 \item For any $\alpha \in \roots$ the only nontrivial complex multiple of $\alpha$ that is also a root is $-\alpha$, i.e.
	\[ z \alpha \in \roots, \quad z\in \C \Longleftrightarrow z \in \{ 1, -1 \}. \]
 \item The roots spaces $\lie{g}_{-\alpha}$ are one-dimensional and the subspace spanned by $\lie{g}_{-\alpha}, \lie{g}_\alpha$ and $[\lie{g}_{-\alpha}, \lie{g}_\alpha]$ is a Lie subalgebra isomorphic to $\lie{sl}(2,\C)$.
 \item For $\alpha, \beta \in \roots$, $\beta \neq -\alpha$ we have
 \[
  [\lie{g}_\alpha,\lie{g}_\beta] = \begin{cases}
				      \lie{g}_{\alpha + \beta} & \text{ if } \alpha + \beta \in \roots \\
				        0 & \text{otherwise}.
                                  \end{cases}
 \]
 \item Let $\alpha, \beta \in \roots$ with $\beta \neq \pm \alpha$ and $z\in\C$. A functional $\alpha + z\beta$ can be a root if and only if $z\in\Z$. The set of such roots form an unbroken chain
 \[ \beta - p\alpha, \beta -(p-1)\alpha, \ldots, \beta + (q-1)\alpha, \beta +q\alpha, \]
       where $p,q\geq 0$ and $p-q = \frac{2(\beta,\alpha)}{(\alpha,\alpha)}$. 
\end{enumerate}

For a root $\alpha \in \roots$ we define the \emph{coroot} $\alpha^\vee$ as
\[
\alpha^\vee = \frac{2}{(\alpha,\alpha)}\widetilde{H_\alpha}.
\]
Nonzero vectors in $\lie{g}_\alpha$ are called a \emph{root vectors}. From the properties of $\roots$ follows that we can always find root vectors $E_\alpha \in \lie{g}_\alpha$, $F_\alpha \in \lie{g}_{-\alpha}$ such that the triple $E_\alpha, F_\alpha, H_\alpha$ satisfies the canonical $\lie{sl}(2)$ relations
\[
 [E,F] = H, \quad [H,E] = 2E, \quad [H,F] = -2F
\]

Choose a basis $v_1, \ldots, v_r$ of $\lie{h}$ and define a linear functional $\lambda\in\lie{h}^*$ to be \emph{positive} if there is an index $j$ such that $\lambda(v_i) = 0$ for $i<j$ and $\lambda(v_j) > 0$. The \emph{positive roots} $\roots^+$ are then the roots which are positive and we obtain a disjoint union $\roots = \roots^+ \cup \roots^-$. The \emph{negative roots}  $\roots^-$ are defined as $-\roots^+$ and we will write $\alpha > 0$ or $\alpha < 0$ to indicate whether the root is positive or negative. By $\sroots$ we will denote the set of \emph{simple roots} associated with $\roots$ and the notion of positivity for $\lie{h}^*$. The simple roots are define as the set of those positive roots, which cannot be written as a sum of positive roots. The set of simple roots $\sroots$ forms a basis of $\lie{h}^*$. Alternatively, one can start with a subset $\sroots$ of $\roots$ that form a basis of $\lie{h}^*$ and declare it to be the set of simple roots. The positive and negative roots are then obtained as 
positive or negative linear combinations of elements of $\sroots$.

Let $\sroots=\{ \alpha_1, \ldots, \alpha_r \}$ be a system of simple roots for $(\lie{g},\lie{h})$. The \emph{Cartan matrix} is defined as
\[
 a_{ij} =\frac{2(\alpha_i,\alpha_j)}{(\alpha_i,\alpha_i)}.
\]
A \emph{Dynkin diagram} is defined as a graph with vertex for each simple root and $i$th and $j$th vertex are joined by $a_{ij}a_{ji}$ many edges. If a two joined vertices correspond to roots of different lengths, one orients the edges by arrow pointing from the longer root to the shorter one.

% Sometimes it is convenient for computations to choose a suitable basis. For a simple roots $\sroots = \{ \alpha_1, \ldots, \alpha_n \}$ choose elements $E_i$  in the root spaces $\lie{g}_{\alpha_i}$ and $F_i \in \lie{g}_{-\alpha_i}$ such that $[E_i,F_i] = \frac{2}{(\alpha_i,\alpha_i)}$. This means that $H_i =[E_i,F_i]$ satisfies $\alpha_i(H_i)=2$ and hence $E_i,F_i,H_i$ is a $\lie{sl}(2,\C)$-triple. The elements of $\{H_i, E_i, F_i : i=1,\ldots, n \}$ are called \emph{Chevalley-Serre} generators and they satisfy the so called \emph{Chevalley-Serre relations}:
% \begin{align*}
%  [E_i, F_j] &= \delta_{i,j}H_i \\
%  [H_i, E_j] &= a_{ij}E_j \\
%  [H_i, F_j] &= -a_{ij}F_j \\
%  \ad(E_i)^{-a_{ij}+1} E_j &= 0 \\
%  \ad(F_i)^{-a_{ij}+1} F_j &= 0 \text{ for } i\neq j.
% \end{align*}

For a root $\alpha \in \roots$ we define \emph{root reflection} $s_\alpha$ as
\[
 s_\alpha: \phi \mapsto \phi - \frac{2(\phi,\alpha)}{(\alpha,\alpha)}\alpha.
\] It is a reflection with respect to the hyperplane orthogonal to $\alpha$ on the Euclidean space formed by the real span of $\roots$ endowed with the restriction of $(\, , \,)$. The set of roots is preserved under root reflections \( s_\alpha(\roots) = \roots\) and the group $W(\roots)$ generated by these reflections is known under the name \emph{Weyl group}. In fact, it is sufficient to take reflections with respect to simple roots to generate the whole Weyl group, i.e. $W(\sroots) = W(\roots)$. The \emph{length function} $l$ on $W$ is defined via the minimal number of simple reflections needed to express $w \in W$. 

\subsection{Borel and parabolic subalgebras}

Given a complex Lie algebra $\lie{g}$ with a chosen Cartan subalgebra $\lie{h}$ and a system of positive roots $\roots^+$ we get a \emph{Cartan decomposition} (\emph{triangular decomposition}) as
\[
 \lie{g} = \lie{n}^- \oplus \lie{h} \oplus \lie{n}, \quad \text{where } \lie{n} = \bigoplus_{\alpha \in \roots^+} \lie{g}_\alpha \text{ and } \lie{n}^- = \bigoplus_{\alpha \in \roots^-} \lie{g}_\alpha.
\]
The Lie subalgebras $\lie{n}$, $\lie{n}^-$ are nilpotent and $\lie{n}^-$ is called the \emph{opposite} Lie subalgebra to $\lie{n}$. The subalgebra $\lie{b} = \lie{h}\oplus \lie{n}$ is the \emph{standard Borel subalgebra} and we will denote by $\lie{b}^- = \lie{h}\oplus \lie{n}^-$ the opposite  Borel subalgebra. It is clear from the triangular  decomposition that $\lie{b}\cap\lie{b}^- = \lie{h}$. All Borel subalgebras are conjugated by inner automorphism to the standard Borel subalgebra.

A \emph{parabolic subalgebra} $\lie{p}$ of $\lie{g}$ is a subalgebra that contains a Borel subalgebra, \emph{standard parabolic subalgebra} is then a subalgebra that contains the standard Borel subalgebra. All parabolic subalgebras are conjugated by inner automorphism to a standard parabolic subalgebra.

Any parabolic subalgebra $\lie{p}$ has a decomposition
\[
 \lie{p} = \lie{l} \oplus \lie{u}
\]
into its Levi part $\lie{l}$ and nilpotent part $\lie{u}$. The Levi part is a reductive Lie subalgebra of $\lie{g}$.

Standard parabolic subalgebras are classified by subset of simple roots (Proposition 3.2.1 of \cite{cap_parabolic_2009}). To a standard parabolic subalgebra $\lie{p}$ we assign the subset \[\Sigma_\lie{p} = \{ \alpha\in\sroots : \lie{g}_{-\alpha}\nsubseteq \lie{p} \}.\] Conversely, the standard parabolic subalgebra $\lie{p}_\Sigma$ corresponding to a subset $\Sigma \subseteq \sroots$ is the sum of the standard Borel subalgebra $\lie{b}$ and all negative root spaces corresponding to roots $\roots_\Sigma$ which can be written as a linear combination of elements of $\sroots \setminus \Sigma$
\[
 \lie{p}_\Sigma = \lie{b} \oplus \sum_{\alpha \in \roots_\Sigma} \lie{g}_{-\alpha}.
\]
In particular, given $\Sigma \subset \sroots$ we get
\[
 \lie{l} = \lie{h} \oplus \sum_{\alpha \in \roots_\Sigma} \lie{g}_\alpha, \quad \lie{u} = \sum_{\alpha \in \roots^+ \setminus \roots_\Sigma} \lie{g}_{\alpha}.
\]

For $\Sigma \subseteq \Sigma' \subseteq \sroots$ we have $\lie{p}_{\Sigma'} \leq \lie{p}_\Sigma \leq \lie{g}$ and the two extreme choices $\Sigma = \emptyset$, $\Sigma = \sroots$ lead to $\lie{p}= \lie{g}$ and $\lie{p} = \lie{b}$ respectively.

The \emph{opposite parabolic} subalgebra $\oppar$  is obtained by switching negative roots to positive and vice versa. In other words, $\lie{g}_\alpha$ is contained in $\oppar$ if and only if $\lie{g}_{-\alpha}$ is in $\lie{p}$.

It is convenient to denote parabolic subalgebras by Dynkin diagrams with crossed or otherwise marked nodes. Namely, the standard parabolic subalgebra $\lie{p}_\Sigma$ of $(\lie{g},\roots^+)$ is denoted by the Dynkin diagram of $\lie{g}$ where the nodes corresponding to $\Sigma$ are represented by crosses instead of dots. By erasing of these crossed nodes one obtains the Dynkin diagram of the semisimple part of $\lie{l}$ and the crossed nodes correspond precisely to the generators of the center of $\lie{l}$.


Elements of $\lie{h}^*$ are called weights. Weights $\lambda$ such that $(\lambda, \alpha)$ is greater (less) or equal to zero for all $\alpha \in \roots_\Sigma$ are called \emph{$\lie{p}$-dominant} ($\lie{p}$-\emph{antidominant}). In case they satisfy this condition for all roots $\alpha \in \roots$ we call them $\lie{g}$-(anti)dominant or just (anti)dominant. The weights dual to simple coroots are called fundamental weights. There is a distinguished element of $\lie{h}^*$ sometimes called the lowest form and usually denoted by $\rho$. It is defined as the sum of fundamental weights or equivalently as half the sum of positive roots. 

The following lemma (whose proof can be found in \cite{cap_parabolic_2009}) shows that the parabolic subalgebras are equivalent to gradings of the lie algebra $\lie{g}$.
\begin{lemma}
There is a bijective correspondence between parabolic subalgebras of $\lie{g}$ and gradings $\lie{g}=\oplus_{i=-k}^k \lie{g}_i$ of $\lie{g}$.

	Given $\Sigma\subset\sroots$, the set $\lie{g}_i$ ($i\neq 0)$ is defined to be $\oplus_{\phi\in A_i} \lie{g}_\phi$, where $A_i$ contains elements $\phi=\sum_{\alpha_j\in\sroots} c_j \alpha_j$ such that $\sum_{\{j:\alpha_j\in\Sigma\}} c_j=i$, and $\lie{g}_0=\lie{h}\oplus_{\phi\in A_0} \lie{g}_\phi$.

	Given a grading $\oplus_j \lie{g}_j$, the parabolic subalgebra is then $\lie{p}=\oplus_{j\geq 0}\lie{g}_j$.
\end{lemma}

So we can see that given a parabolic Lie algebra $\lie{p} = \lie{l} \oplus \lie{u}$ there is a grading on $\lie{g}$ such that $\lie{g}_0 = \lie{l}$ and $\lie{u} = \bigoplus_{i \geq 1} g_i$. The associated filtration to a grading $\lie{g}=\oplus_{i=-k}^k \lie{g}_i$ of $\lie{g}$ is defined by $\lie{g}^i = \bigoplus_{j \geq i} \lie{g}_j.$ Sometimes it's convenient to use the following notation as well $\lie{g}_+ = \bigoplus_{i > 0} \lie{g}_i$ and $\lie{g}_- = \bigoplus_{i < 0} \lie{g}_i.$ The next lemma shows structural properties of this grading as well as relationship with the Killing form.

\begin{lemma}[Proposition 3.1.2 of \cite{cap_parabolic_2009}]
	Let $\lie{g}=\lie{g}_{-k}\oplus\cdots \lie{g}_k$ be a $|k|$-graded semisimple Lie algebra over $\k=\R$ or $\C$ and let $B:\lie{g}\otimes \lie{g}\to \k$ be a nondegenerate invariant bilinear form. Then we have:
	\begin{enumerate}
	 \item There is a unique element $E \in \lie{g}$, called the \emph{grading element}, such that \([E,X] = jX\) for all $X \in \lie{g}_j$. The element $E$ lies in the center of the subalgebra $\lie{g}_0 \leq \lie{g}$.
	 \item The $|k|$-grading on $\lie{g}$ induces a $|k_i|$-grading for some $k_i\leq k$ on each ideal $s\subseteq \lie{g}$. In particular, $\lie{g}$ is a direct sum of $|k_i|$-graded simple Lie algebras, where $k_i\leq k$ for all $i$ and $k_i=k$ for at least one $i$.
	 \item The isomorphism $\lie{g} \to \lie{g}^*$ provided by $B$ is compatible with the filtration and the grading of $\lie{g}$. In particular, $B$ induces dualities of $\lie{g}_0$-modules between $\lie{g}_i$ and $\lie{g}_{-i}$ and the filtration component $\lie{g}^i$ is exactly the annihilator (with respect to $B$) of $\lie{g}^{-i+1}$. Hence, $B$ induces a duality of $\lie{p}$-modules between $\lie{g}/\lie{g}^{-i+1}$ and $\lie{g}^i$, and in particular between $\lie{g}/\lie{p}$ and $\lie{p}_+$.  
	 \item For $i<0$ we have $[\lie{g}_{i+1},\lie{g}_{-1}] = \lie{g}_i$. If no simple ideal of $\lie{g}$ is contained in $\lie{g}_)$, then this also holds for $i=0$.
	 \item Let $A\in\lie{g}_i$ with $i>0$ be an element such that $[A,X]=0$ for all $X\in\lie{g}_{-1}$. Then $A=0$. If no simple ideal of $\lie{g}$ is contained in $\lie{g}_)$, then this also holds for $i=0$.
	\end{enumerate}
\end{lemma}

In particular, the Killing form $B$ has the following anti-diagonal block matrix form with respect to the decomposition $\lie{g}=\lie{g}_{-k}\oplus\cdots \lie{g}_k$
\[
 B = \begin{pmatrix}
      0 & \cdots & B_{k,-k} \\
      \vdots &  \ddots & \vdots \\
      B_{-k,k} & \cdots&0
     \end{pmatrix},
\]
where $B_{i,j}$ denote the restriction of $B$ to $\lie{g}_i\otimes \lie{g}_j$.

Since the grading element $E$ is in the center of $\lie{g}_0 = \lie{l}$ it acts by scalar on any irreducible $\lie{l}$-module $\mathbb{V}$. We will call this scalar \emph{geometric weight} of $\mathbb{V}$.

The Weyl group has natural action on weights by orthogonal transformations and we call a weight singular (regular) if $(\lambda + \rho, \alpha)$ is (not) zero. The reason for this $\rho$ shift is that in applications one has to consider not the defining acting of the Weyl group $W$ but rather it's \emph{affine action} 
\[
 w \cdot \lambda = w(\lambda + \rho) - \rho.
\]

Let $w$ be an element of $W$. A \emph{reduced expression} for $w$ is $w = s_{i_1} \cdots s_{i_k}$ where $k$ is as small as possible and all $s_{i_j}$ are simple reflections. In particular $l(w) = k$. The \emph{Bruhat order} on $W$ is defined as follows: $w' \leq w$ if and only if a reduced expression for $w'$ is subexpression of some reduced expression of $w.$ For any standard parabolic subalgebra $\lie{p}_\Sigma$ one can form the parabolic Weyl group $W_\Sigma$ which is generated by simple reflections belonging to roots of $\lie{l}_\Sigma$. It is a theorem of Kostant \cite{kostant_lie_1961} that for any $w \in W$ there is a unique decomposition $w = w_\Sigma w^\Sigma$ such that $l(w) = l(w_\Sigma) + l(w^\Sigma)$. The element $w^\Sigma$ is called \emph{minimal length representative} of $w$ and one of it's fundamental properties is that it maps $\lie{g}$-dominant weights to $\lie{p}$-dominant weights. The set of all minimal length representative is denoted by $W^\Sigma$ and it inherits a partial order from $W$.


\section[BGG category O]{BGG category $\mathcal{O}$}

This chapter contains the description of Bernstein-Bernstein-Gelfand category $\mathcal{O}$ and a small recapitulation of Enright-Shelton equivalences which will be needed later. In this section we will denote the Levi lie group / algebra by $K$ / $\lie{k}$ since later we will consider only the cases when the Levi subalgebra is actually the complexification of maximal compact subalgebra of certain real form of $\lie{g}$.

For any $\lambda\in\lie{h}^*$ we denote by $\C_\lambda$ the one dimensional representation of $\lie{h}$ on with character $\lambda$. We extend the action to $\lie{b} := \lie{n}\oplus\lie{h}$ by letting $\lie{n}$ act trivially. The \emph{Verma module} $N(\lambda)$ is then defined as $N(\lambda) := \lie{U(g)}\otimes_{\lie{U(b)}} \C_\lambda$. We will denote its highest weight vector by $v_\lambda$.

Let $\lambda$ be a $\lie{k}$-dominant and integral weight and denote by $F(\lambda)$ the finite dimensional irreducible $\lie{k}$-module.  We can extend any irreducible representation of $K_\mathbb{C}$ to $P$ and to $\overline{P}$ by letting $\lie{p}_+$ and $\lie{p}_-$ act trivially. The \emph{generalized} or \emph{parabolic Verma module} $M(\lambda)$ is defined as $M(\lambda) = \lie{U}(\lie{g}) \otimes_{\lie{U}(\lie{p})} F(\lambda)$. In cases where we will need to deal with Verma modules for different algebras we will denote them by $M_\Sigma$ where $\Sigma$ is the subset of $\sroots$ defining the parabolic subalgebra $\lie{p}$. 

 It is well known and easy to prove that $M(\lambda)$ contains a maximal nontrivial submodule $J(\lambda)$ and we denote by $L(\lambda)$ the corresponding irreducible quotient of $M(\lambda)$. We can easily see that $M(\lambda) \simeq S(\lie{p}_-)\otimes F(\lambda)$ as $K_\C$ representations, where $S(\lie{p}_-)$ is the symmetric algebra  over the Lie algebra $\lie{p}_-$.
%The canonical $\overline{P}$-invariant filtration of $M(\lambda) = \bigcup_{k\geq 0} M^k(\lambda)$ has filtration components $M^k(\lambda)= \{\epsilon^{i_1}\cdots\epsilon^{i_j}\otimes v\, | v \in F(\lambda), j \leq k \}$.
In the case of $M(\lambda)$, the geometric weight corresponds to the polynomial degree shifted by the weight of $F(\lambda)$.% and the canonical $\overline{P}$-invariant filtration is the same as the filtration induced by the geometric weight.

The \emph{(parabolic) category $\mathcal{O}_\lie{p}$} is the full subcategory of $\lie{U(g)}$ modules whose objects $M$ satisfy:
\begin{enumerate}
\item $M$ is a finitely generated $\lie{U(g)}$ module
\item Viewed as a $\lie{U(l)}$ module, $M$ is a direct sum of finite dimensional simple modules.
\item $M$ is locally $\lie{u}$-finite
\end{enumerate}

\subsection{Translation principle}\label{sec:translation}
The whole category $\mathcal{O}^ \lie{p}$ decomposes into so called \emph{infinitesimal blocks} $\mathcal{O}^\lie{p}_\mu$, where $\mu$ is an antidominant weight of $\roots$. These blocks are full subcategories consisting of modules on which the center of the universal enveloping algebra acts by a character induced from $\mu$.  One can move between different blocks using so called translation functors which in favorable cases provide equivalence of these subcategories. These functors map Verma modules to Verma modules and simple modules to simple modules. The favorable cases are determined by so called facets which are in turn defined via root hyperplanes. The bottom line is that once two weights $\mu, \mu'$ have the same signs of scalar products with positive roots the infinitesimal blocks $\mathcal{O}^\lie{p}_\mu$ and $\mathcal{O}^\lie{p}_{\mu'}$ are equivalent. See chapter 7 of \cite{humphreys_representations_2008} for details.

%\subsection[Category O and translation functors]{Category $\mathcal{O}$ and translation functors}
%
%We present a short overview of translation functors as is presented in \cite{humphreys_representations_2008}. Since we deal with integral weights only, we omit all the notations that concerns nonintegral weights.
%
%\begin{definition}
%  Let $\lambda, \mu \in \weights$ be such that $\nu = \mu - \lambda \in \weights^+$ and consider a finite dimensional representation $F(\nu)$ of highest weight $\nu$. Then the translation functor $T_\lambda^\mu: \mathcal{O}_\mu \to \mathcal{O}_\lambda$ is defined by
%  \[
%   T_\lambda^\mu M := \proj(F(\nu)\otimes M),
%  \]
%  where $\proj$ is the projection from $\mathcal{O}$ to $\mathcal{O}_\mu$.
%\end{definition}
%
%The vector space $\lie{h}^*$ decomposes into so called facets for the affine Weyl group action according to decompositions of the set of positive roots $\roots^+$ as follows. The facet $F$ corresponding to $\roots^+ = \roots^+_- \cup \roots^+_0 \cup \roots^+_+$ is defined as the set of those $\lambda \in \lie{h}$ such that
%\[
%  \begin{cases}
%   (\lambda + \rho, \alpha) < 0, & \forall \alpha \in \roots^+_- \\
%   (\lambda + \rho, \alpha) = 0, & \forall \alpha \in \roots^+_0 \\
%   (\lambda + \rho, \alpha) > 0, & \forall \alpha \in \roots^+_+.
%  \end{cases}
%\]
%
%\begin{theorem}[Theorem 7.?? of \cite{humphreys_representations_2008}]\label{thm:translation}
% If two weights $\mu$ and $\lambda$ lie in the same facet, then the translation functors $T_\lambda^\mu$ and $T_\mu^\lambda$ implement an equivalence of categories between $\mathcal{O}_\mu$ and $\mathcal{O}_\lambda$.
%\end{theorem}

\subsection{Shapovalov form}

Let $\sigma$ be involutive antiautomorphism on $\lie{U(g)}$ such that it's restriction on the real form $\lie{g}_0$ is $-\id$. With our choice of Cartan subalgebra $\lie{h}$ we have that $\sigma: X_\beta \mapsto X_{-\beta}$ for $X_\beta \in \lie{g}_\beta$ and $\sigma: h_\alpha \mapsto h_\alpha$ for $h_\alpha \in \lie{h}$.

Let $P: \lie{U(g)} \to \lie{U(h)}$ be the projection defined by the splitting \[\lie{U(g)} = \lie{U(h)}\oplus \left( \lie{n_-U(g)} + \lie{U(g)n_+} \right)\]

\begin{definition}
  The \emph{universal Shapovalov form} on $\lie{U(g)}$ is defined as \[\langle u_1 , u_2 \rangle = P(\sigma(u_1)u_2).\] It is a bilinear form on $\lie{U(g)}$ with values in $\lie{U(h)} = S(\lie{h})$.

  For $\lambda\in\lie{h}^*$ we define the \emph{Shapovalov form on the Verma module} $N(\lambda)$ by \[\langle u_1 v_\lambda , u_2 v_\lambda \rangle = P(\sigma(u)v)(\lambda).\]
\end{definition}

It is easy to see that $\langle u\cdot v,v' \rangle = \langle v,\sigma(u)\cdot v' \rangle$ for all $v,v'\in M_\lambda$ and for all $u\in \lie{U(g)}$. Forms with such a property are called \emph{contravariant forms}.

The following proposition and its proof can be found e.g. in \cite{humphreys_representations_2008}.

\begin{proposition}
Contravariant forms have the following properties:
\begin{enumerate}
 \item If the $\lie{U(g)}$-module $M$ has a contravariant form $(v,v')_M$, then the weight spaces are orthogonal. I.e. $( M_\mu,M_\nu)=0$ whenever $\mu\neq\nu$ in $\lie{h}^*$.
 \item Suppose $M = \lie{U(g)} \cdot v$ is a highest weight module generated by a maximal vector $v$ of weight $\lambda$. If $M$ has a nonzero contravariant form, then the form is uniquely determined up to a scalar multiple by the (nonzero!) value $( v, v)_M$.
 \item  If $\lie{U(g)}$-modules $M_1, M_2$ have contravariant forms $(v, v')_{M_1}$ and $(w,w')_{M_2}$ , then $M := M_1\otimes M_2$ also has a contravariant form, given by \[(v\otimes w, v'\otimes w')_M := (v, v')_{M_1} (w,w')_{M_2}.\] In case both of the forms are nondegenerate, so is the product form.
 \item If $M$ has a contravariant form and $N$ is a submodule, the orthogonal complement $N^\perp := \{v\in M | (v, v')_M = 0 \text{ for all } v' \in N \}$ is also a submodule.
\end{enumerate}
\end{proposition}
\begin{proof}
\begin{enumerate}
 \item We use the fact that $\sigma(h) = h$ for $h\in\lie{h}$. Let $v$ be a vector of weight $\mu$ and let $v'$ be a vector of weight $\nu$. The for any $h\in\lie{h}$ we have
 \[
  \mu(h)(v,v')_M = (h\cdot v,v')_M = (v,\sigma(h)\cdot v')_M = (v,h\cdot v')_M = \nu(h)(v,v')_M.
 \]
 Since $v,v'$ were arbitrary we must have $(v,v')=0$ for $\mu\neq \nu$.
 \item In view of the already proven point it suffices to look at values of the form on a weight
space $M_\mu$. Typical vectors $v, v' \in M_\mu$ can be written as $u\cdot v, u'\cdot v$ for suitable $u, u'\in\lie{U(n)}$. Note that since $u$ takes $M_\lambda$ into $M_\mu$, the element $\sigma(u) \in \lie{U(n^-)}$ takes $M_\mu$ into $M_\lambda$ (which is spanned by $v$). Then $(v, v')_M = (u\cdot v, u'\cdot v)_M = (v, \sigma(u)u'\cdot v)_M$, which is a scalar multiple of $(v, v)_M$ depending just on the action of $\lie{U(g)}$ and not on the choice of the form.

The remaining points are elementary.
\end{enumerate}
\end{proof}

Important consequence of this is the following lemma.
\begin{lemma}
The maximal submodule of $N(\lambda)$ is the radical of the Shapovalov form.
\end{lemma}
\begin{proof}
Let $v\in J(\lambda)$  be arbitrary. Then $\langle v, v_\lambda \rangle = 0$ since clearly $v$ and $v_\lambda$ have different weights. Since $J(\lambda)^\perp$ is a submodule of $N(\lambda)$ containing the generating vector $v_\lambda$ the statement follows.
\end{proof}

The generalized Verma module $M(\lambda)$ is a quotient of the Verma module $N(\lambda)$. Hence the simple quotient $L(\lambda)$ of $M(\lambda)$ is also a quotient of $N(\lambda)$ by its maximal submodule. We get induced contravariant forms on $L(\lambda)$ and $M(\lambda)$  which we also call Shapovalov forms.


\subsection{Equivalences of Enright and Shelton}\label{sec:es_equivalence}

%Let $\lie{q}$ be a maximal parabolic subalgebra of $\lie{g}$ with Levi decomposition $\lie{q} = \lie{l_q} \oplus \lie{u_q}$ and suppose that $\lie{q}$ does not contain $\lie{p}$ and $\lie{l_q}$ contains $\lie{h}$ and $\lie{q}$ contains $\lie{b}$. Let $\lie{c}$ equal the center of $\lie{l_q}$ and fix $\lambda \in \lie{h}^*.$ Put $\mathcal{O} = \mathcal{O}(\lie{g}, \lie{p}, \lambda)$ and $\mathcal{O}_{\lie{l_q}} = \mathcal{O}_t( \lie{l_q}, \lie{l_q \cap p}, \lambda).$  

Let $\lie{p}$ be a parabolic subalgebra of $\lie{g}$ of Hermitian type. (I.e. giving rise to a $|1|$-grading on $\lie{g}.$) Let $\Sigma$ be the set of singular simple roots for an antidominant weight $\mu$:
\[
J = \{ \alpha \in \sroots \,|\, (\mu + \rho, \alpha^\vee) = 0 \}. 
\]
\begin{theorem}[8, Prop 2.3 of \cite{boe_representation_2005}]\label{thm:es_equivalence}
Let
\[
W^{\Sigma, J} = \{ w \in W^\Sigma \, | \,  \forall \alpha \in \Sigma  :  w < ws_\alpha \text{ and } ws_\alpha \in W^\Sigma\}
\]
and define $L(w)$ to be the simple quotient of the parabolic Verma module $M_\Sigma(w) = M(w_\Sigma w \cdot \mu)$ where $w_\Sigma$ is the longest element of the parabolic Weyl group $W_\Sigma$. Then 
\[
W^{\Sigma, J} \to \{ \text{simple modules in } \mathcal{O}^\lie{p}_\mu \}
\]
given by $w \mapsto L(w)$ is a bijection.
\end{theorem}

\begin{theorem}[\cite{enright_categories_1987}, \cite{enright_highest_1989}] 
\hspace{1em} \\[-1em]
\begin{enumerate}
	\item Suppose that either $\roots$ has one root length or that it has roots of two lengths and all the roots in $J$ are short. Then there exists an equivalence of categories
	\[
	\mathcal{E}: \mathcal{O}^\lie{p}_\mu \to \mathcal{O}^\lie{p'}_{\mathrm{reg}},
	\]
	where $\lie{p}'$ is a parabolic subalgebra of Hermitian type of a complex simple Lie algebra $\lie{g}'$ of rank smaller or equal to the rank of $\lie{g}.$ Moreover, there is an isomorphism of posets $W^{\Sigma, J} \to W'^{\Sigma'}$ such that the functor $\mathcal{E}$ sends parabolic Verma modules $M_\Sigma(w)$ to parabolic Verma modules $M_{\Sigma'} (w')$ and similarly for simple modules.
	
	\item Suppose that $\roots$ has two root lengths and $J$ contains a long root. Then there exists an equivalence of categories
	\[
	\mathcal{E}: \mathcal{O}^\lie{p}_\mu \to \mathcal{O}^\lie{p'}_{\mathrm{reg}} \oplus  \mathcal{O}^\lie{p'}_{\mathrm{reg}},
	\]
	where $\lie{p}'$ is a parabolic subalgebra of Hermitian type of a complex simple Lie algebra $\lie{g}'$ of rank smaller or equal to the rank of $\lie{g}.$ More precisely the poset $W^{\Sigma, J}$ is a disjoint union $W^{\Sigma, J}_\mathrm{odd} \cup W^{\Sigma, J}_{\mathrm{even}}$ of two poset and the category $\mathcal{O} ^\lie{p}_\mu$ has a decomposition into a direct sum $\mathcal{O} ^\lie{p}_\mu = \mathcal{O} ^\lie{p}_{\mu, \mathrm{even}} \oplus \mathcal{O} ^\lie{p}_{\mu, \mathrm{odd}}$ such that all extensions between modules in different summands is zero. There exists isomorphisms of posets $ W^{\Sigma, J}_{\mathrm{odd}} \to W'^{\Sigma'}$ and $ W^{\Sigma, J}_{\mathrm{even}} \to W'^{\Sigma'}$ and corresponding equivalences of categories $\mathcal{E}_{\mathrm{odd}}: \mathcal{O} ^\lie{p}_{\mu, \mathrm{odd}} \to  \mathcal{O}^\lie{p'}_{\mathrm{reg}}$ and $\mathcal{E}_{\mathrm{even}}: \mathcal{O} ^\lie{p}_{\mu, \mathrm{even}} \to  \mathcal{O}^\lie{p'}_{\mathrm{reg}}$ such that $M_\Sigma(w) \to M_{\Sigma'} (w')$ and similarly for their simple quotients.
\end{enumerate}
\end{theorem}

Let's illustrate this equivalence with an example. 
\begin{example}[\cite{boe_kostant_2009}]
Consider the complex simple Lie algebra $\lie{sl}(6)$ and let $\Sigma = \{\alpha_3\}$. Since the Weyl group is a permutation group acting on $\epsilon$-coordinates, the weights which are singular have repeating values. Pick $\mu = (0,1,2\, |\, 3,3,4)$. The permutations of $\mu + \rho$ which are highest weights plus $\rho$ of simple modules in $\mathcal{O}^{\Sigma, J}$ have their first three and last three entries strictly decreasing.  There are only six possible cases, namely
\[
(4, 3, 2\, |\, 3, 1, 0) \quad (4, 3, 1\, |\,3, 2, 0) \quad (4, 3, 0\, |\,3, 2, 1)
\]
and their counterparts with first three and last three entries swapped. Now these three weights are equivalent to 
\[
(4,  2\, |\,  1, 0) \quad (4,  1\, |\, 2, 0) \quad (4,  0\, |\, 2, 1).
\]
If we impose that the weight for $\lie{sl}(6)$ has to have entries increasing by $1$ this mapping on weights has inverse since we know that we have to put the number $3$ into two places and there's only one way to do it in each case so that one obtains $\lie{l}$-dominant weight. This directly generalizes to all other $\lie{sl}$ cases as any weight can be brought by translation functors to a weight that starts with $0$, has entries which are nondecreasing and which increase only by $1$.
\end{example}

\begin{remark}
The proof of these equivalences is not entirely satisfactory. The functors implementing the Enright-Shelton equivalences are combinations of derived Zuckermann functors, extension functor $M \to \lie{U(g)} \otimes_{\lie{U(q)}} (M \otimes F \otimes L)$ where $L$ is the one-dimensional module of weight $\rho(\lie{u_q})$ of a certain parabolic subalgebra $\lie{q}$. Determining the effect of these functors on modules of singular character is hard. Independent proof of one version of the Enright-Shelton equivalence was given by \cite{soergel_mathscr_1988} using Beilinson-Bernstein localization. Another independent proof for the $A$ series was given by \cite{bernstein_categorification_1999}. See also \cite{pandzic_bgg_2016}.
\end{remark}

\section{Cohomology and homology of Lie algebras}

In this section we denote by $\lie{m}$ an arbitrary Lie algebra and by $\rep{V}$ its representation. We will denote the action of $\lie{m}$ on $\rep{V}$ by a dot. We follow \cite{kostant_lie_1961} and \cite{cap_parabolic_2009}.

The chain spaces of \emph{Lie algebra homology} $C_k(m,\rep{V})$ of the algebra $\lie{m}$ with values in $\rep{V}$ are defined as $\Lambda^k\lie{m}\otimes\rep{V}$. The Lie algebra \emph{homology differential} \[\lhd_\lie{m} : C_{k+1}(\lie{m},\rep{V}) \to C_k(\lie{m},\rep{V})\] is defined by the following formula
\begin{multline*}
\lhd_\lie{m} (Z_0\wedge\cdots Z_k\otimes v) = \sum_{i=0}^k (-1)^{i+1} Z_0\wedge\cdots\wedge\widehat{Z_i}\wedge\cdots\wedge Z_k\otimes Z_i\cdot v\,  + \\
				      +\sum_{i<j} (-1)^{i+j} [Z_i,Z_j]\wedge Z_0\wedge\cdots\wedge\widehat{Z_i}\wedge\cdots\wedge\widehat{Z_j}\wedge\cdots\wedge Z_k\otimes v,
\end{multline*}
where $Z_i\in\lie{m}$ for $i=0,\ldots,k$. If the algebra $\lie{m}$ is abelian, then the second term in the sum is zero. The cochain spaces of \emph{Lie algebra cohomology} $C^k(\lie{m},\rep{V})$ are defined as $\Hom(\Lambda^k \lie{m},\rep{V})$. The Lie algebra \emph{cohomology differential} \[\lcd_\lie{m} :  C^k(\lie{m},\rep{V}) \to C^{k+1}(\lie{m},\rep{V})\] is defined as
\begin{multline*}
 (\lcd_\lie{m} \phi)(X_0,\ldots, X_n) = \sum_{i=0}^n (-1)^i X_i \cdot \phi(X_0,\ldots,\widehat{X_i},\ldots, X_n)\, + \\
				+\sum_{i<j} (-1)^{i+j} \phi([X_i,X_j],X_0,\ldots,\widehat{X_i},\ldots,\widehat{X_j},\ldots,X_n),
\end{multline*}
where $X_0,\ldots, X_n \in \lie{m}$ and $\phi:\Lambda^k \lie{m} \to \rep{V}$. Again, we can forget the second term if the algebra $\lie{m}$ is commutative.

There is a natural identification of $C_k(\lie{m}^*,\rep{V})$ and $C^k(\lie{m},\rep{V})$ coming from the natural isomorphism $\Lambda^k \lie{m}^* \otimes \rep{V} \simeq \Hom(\lie{m},\rep{V})$.

In the special case when we consider $\lie{m}$ to be a nilradical $\lie{p}_-$ of some opposite parabolic subalgebra $\lie{p}$, we have the Killing form that induces an isomorphism ${\lie{p}_-}^* \simeq \lie{p}_+.$ Since we can identify $C^k(\lie{p}_-,\rep{V})=\Hom (\Lambda^k\lie{p}_-,\rep{V})$ with $\Lambda^k {\lie{p}_-}^*\otimes\rep{V}$, we see we can consider the Lie algebra cohomology differential $\lcd$ as an operator on the chain spaces of Lie algebra homology $\lcd: \Lambda^k {\lie{p}_-}^* \otimes V \to \Lambda^{k+1}{\lie{p}_-}^*\otimes V$. After these identifications we get the formula
\[
 \lcd  (Z_1\wedge\cdots\wedge Z_k\otimes v) = \sum_{i=1}^{2p} \epsilon^i \wedge Z_1\wedge\cdots\wedge Z_k\otimes e_i\cdot v.
\] The Lie algebra cohomology differential is $\overline{P}$-equivariant. In particular, both $\lcd$ and $\lhd$ are $L$-equivariant and consequently they preserve the geometric weight.

The following two propositions are well known.

\begin{proposition}
Suppose $M$ is a $\mathfrak{g-}$module. Let $\mathfrak{p}\subseteq \mathfrak{%
g}$ be a parabolic subalgebra with nilradical $\mathfrak{n}$ and Levi factor
$\mathfrak{l}$. \newline
(a) Let $M^{\ast }$ denote the $\mathfrak{g}-$module dual to $M.$ Then there
is a natural isomorphism
\begin{equation*}
H_{\text{p}}(\mathfrak{n},M^{\ast })\cong H^{\text{p}}(\mathfrak{n},M)^{\ast
}
\end{equation*}%
where $H^{\text{p}}(\mathfrak{n},M)^{\ast }$ denotes the $\mathfrak{l-}$%
module dual to $H^{\text{p}}(\mathfrak{n},M)$.\newline
(b) Let d denote the dimension of $\mathfrak{n}$. Then there is a natural
isomorphism
\begin{equation*}
H_{\text{p}}(\mathfrak{n},M)\cong H^{\text{d-p}}(\mathfrak{n},M)\otimes
\Lambda ^{\text{d}}\mathfrak{n}\newline
\end{equation*}
\end{proposition}



%\subsection{Real versus complex}

\begin{proposition}[Proposition 3.3.6 of \cite{cap_parabolic_2009}]
 Let $\lie{g}$ be a $|k|$-graded semisimple Lie algebra with complexification $\lie{g}^\C$ and let $\rep{V}$ be a complex representation of $\lie{g}$. Then the real cohomology spaces $H^*(\lie{g}_-,\rep{V})$ are 	naturally complex vector spaces and we have a natural isomorphism of $\lie{g}_0$-modules
 \[
  H_\R^*(\lie{g}_-,\rep{V}) \simeq H_\C^*(\lie{g}_-^\C,\rep{V}).
 \]

\end{proposition}



%\subsection{Invariant and coinvariant functors and their derivations}

Finally let us recall interpretation of the Lie algebra (co)homology as derived functors which plays a crucial role in the proof of existence of the BGG resolution for regular Kostant modules. See \cite{enright_diagrams_2014, boe_kostant_2009} and references therein. 

\emph{The right standard resolution of }$\mathbb{C}$ is the
complex of free right $U(\mathfrak{n})-$modules given by
\begin{equation*}
\cdots \rightarrow \Lambda ^{p+1}\mathfrak{n}\otimes U(\mathfrak{n}%
)\rightarrow \Lambda ^{p}\mathfrak{n}\otimes U(\mathfrak{n})\rightarrow
\cdots \rightarrow \mathfrak{n}\otimes U(\mathfrak{n})\rightarrow U(%
\mathfrak{n})\rightarrow 0\text{.}
\end{equation*}%
Applying the functor
\begin{equation*}
-\otimes _{U(\mathfrak{n})}M\text{ }
\end{equation*}%
to the standard resolution we obtain a complex
\begin{equation*}
\cdots \rightarrow \Lambda ^{p+1}\mathfrak{n}\otimes M\rightarrow \Lambda
^{p}\mathfrak{n}\otimes M\rightarrow \cdots \rightarrow \mathfrak{n}\otimes
M\rightarrow M\rightarrow 0
\end{equation*}%
of left $\mathfrak{l}-$modules called \emph{the standard} $\mathfrak{n}-$%
\emph{homology complex}. Here $\mathfrak{l}$ acts via the tensor product of
the adjoint action on $\Lambda ^{p}\mathfrak{n}$ with the given action on $M$%
. Since $U(\mathfrak{g})$ is free as $U(\mathfrak{n})-$module, one can show that the $p$th-homology of the standard complex is in fact the $p$th $\mathfrak{n}-$homology group $H_{\text{p}}(\mathfrak{n},M).$

The \emph{zero} $\mathfrak{n}-$\emph{cohomology }of a $\mathfrak{g}%
-$module $M$ is the $\mathfrak{l}-$module
\begin{equation*}
H^{0}(\mathfrak{n},M)=\text{Hom}_{U(\mathfrak{n})}(\mathbb{C},M).
\end{equation*}%
This suggest that applying the left exact functor from the category of $\mathfrak{g}-$%
modules to the category of $\mathfrak{l}-$modules
\begin{equation*}
\text{Hom}_{U(\mathfrak{n})}(-,M)
\end{equation*}%
to the standard resolution of $\mathbb{C}$, this time by free left $U(%
\mathfrak{n})-$modules, will yield the Lie algebra cohomology groups which is indeed the case. 

Hence we see that the we can see the Lie algebra (co)homology as the $\mathrm{Tor}$ and $\mathrm{Ext}$ functors.

\subsection{Disjoint operators and Hodge decomposition}

Let $V$ be a topological vector space and let $\lcd$ and $\lhd$ be two linear operators on $V$ such that $\lcd^2 = \lhd^2 = 0$. Such operators are called \emph{disjoint} if
\begin{gather*}
 \lcd\lhd x= 0 \text{ implies } \lhd x =0 \\
 \intertext{and}
 \lhd \lcd x = 0 \text{ implies } \lcd x = 0
\end{gather*}
for all $x\in V$. In other words, $\lhd$ and $\lcd$ are disjoint if and only if
\begin{equation}\label{eq:disjointness}
 \ker \lcd \cap \im \lhd = 0 \quad \& \quad \ker \lhd \cap \im \lcd = 0.
\end{equation}

%TODO Co zkusit pozadovat, aby $V=\im\lhd \oplus \ker \lcd$ a vice versa?

The \emph{Laplacian} is then defined as
\[\lap = \lcd \lhd + \lhd \lcd.\] We note that $\lap  = (\lhd + \lcd)^2$ and hence one can consider $\lhd + \lcd$ as the Dirac operator for the Laplacian $\lap$. %See the book \cite{huang} for investigations of this Dirac operator.

\begin{proposition}%[Proposition 2.1 of \cite{kostant_lie_1961}]
 Let the notation be as above and assume $\lcd$ and $\lhd$ are disjoint and $\dim V < \infty$. Then
 \begin{equation}\label{eq:hodge_decomp_aux}
  \ker \lap = \ker \lcd \cap \ker\lhd \quad \& \quad  \im \lap = \im \lhd \oplus \im \lcd.
 \end{equation}
 Also one has a direct sum decomposition (\emph{Hodge decomposition}),
 \begin{equation}\label{eq:hodge_decomp}
    V = \im \lcd \oplus \ker \lap \oplus \im \lhd.
 \end{equation}
\end{proposition}
\begin{proof}
To prove the first equality of \eqref{eq:hodge_decomp_aux} we just use the definition of disjointness \eqref{eq:disjointness} of $\lhd$ and $\lcd$. The inclusion $\ker \lhd \cap \ker \lcd \subseteq \ker\lap$ is trivial. Let $x \in \ker \lap$ and put $y=-\lhd\lcd x$. Then $\lhd y =0$ and also $y =\lcd\lhd x$. Thus $\lhd\lcd\lhd x = 0$ and by disjointness this implies that $\lcd\lhd x = 0$ which, for the same reason, implies $\lhd x = 0$. Similarly $\lcd x = 0$. The remaining two decompositions of \eqref{eq:hodge_decomp_aux} are direct consequence of $\ker\lap = \ker\lhd\cap\ker\lcd$ and of disjointness of $\lhd$ and $\lcd$.

 It is obvious that $\im \lap \subseteq \im \lcd + \im \lhd$ and  by \eqref{eq:hodge_decomp_aux} and disjointness \eqref{eq:disjointness} we have $(\im \lhd + \im \lcd) \cap \ker \lap = 0$. If follows that $\im \lap \cap \ker \lap =0$. Since $V$ is finite-dimensional, we get $\dim \ker\lap + \dim \im \lap = \dim V$ and \eqref{eq:hodge_decomp}  and the second equality of \eqref{eq:hodge_decomp_aux} follows.
\end{proof}

\begin{corollary}
 Under the assumptions of the previous proposition, there are decompositions
  \begin{equation}
    \quad \ker \lhd = \im \lhd \oplus \ker \lap, \quad \ker \lcd = \im\lcd\oplus\ker\lap.
  \end{equation}
  Consequently we have isomorphisms
  \begin{equation}\label{eq:kerlap_iso}
    \ker \lap \simeq \ker \lcd /\im \lcd \quad\&\quad \ker \lap \simeq \ker \lhd / \im \lhd
  \end{equation}
  given by restriction to $\ker \lap$ of the canonical mappings $\ker \lcd \to \ker \lcd / \im \lcd$ and $\ker \lhd \to \ker \lhd / \im \lhd$.
\end{corollary}
\begin{proof}
 This is a direct consequence of the Hodge decomposition \eqref{eq:hodge_decomp}.
\end{proof}

\begin{remark}
 For any two linear mappings $\lhd$ and $\lcd$ on a vector space $V$ such that $\im\lhd\cap\im\lcd = 0$ we have that \[ \ker (\lhd + \lcd)  = \ker \lhd \cap \ker \lcd.\] Under our assumptions we get that $\ker (\lhd +\lcd) = \ker \lap$, since $\im \lhd \subseteq \ker \lhd$ and by disjointness \eqref{eq:disjointness} $\im \lhd \cap \im \lcd = 0$.
\end{remark}


To better understand the situation we can represent these operators in a block matrix form with respect to the Hodge decomposition \eqref{eq:hodge_decomp}
\[
 \lcd = \begin{pmatrix} 0 & 0 & A\\0&0&0\\0&0&0\end{pmatrix} \quad \lhd = \begin{pmatrix} 0 & 0 &0\\0&0&0\\ B&0&0\end{pmatrix} \quad
 \lap = \begin{pmatrix}
	  AB & 0 & 0 \\
	  0 & 0 & 0 \\
	  0 & 0 & BA
	\end{pmatrix},
\]
where $A$ is the restriction of $\lcd$ to $\im\lhd$ and $B$ is the restriction of $\lhd$ to $\im \lcd$. Disjointness of $\lcd$ and $\lhd$ then amounts to nondegeneracy of $A$ and $B$.

There is a convenient way to construct pairs of disjoint operators.
\begin{proposition} %TODO jak je to s im adjungovaneho operatoru? v konecne dimenzi by ta podminka na nedegenerovanost na im adjunktu mela byt nadbytecna. v nekonecne?
 Let $\lhd$ be a differential on a vector space $V$ endowed with a non-degenerate bilinear (or sesquilinear) form $\langle \, , \, \rangle$ and let $\lcd$ be the adjoint of $\lhd$ with respect to $\langle \, , \, \rangle$. If the restrictions of $\langle \, , \, \rangle$ to $\im \lhd$ and $\im \lcd$ are non-degenerate, then $\lhd$ and $\lcd$ are disjoint and the Hodge decomposition \eqref{eq:hodge_decomp} is orthogonal with respect to $\langle \, , \, \rangle$.
\end{proposition}
\begin{proof}
 It is trivial to see that $\lcd$ is a differential, for we have 
 \[ 0 = \langle \lhd ^2 x, y \rangle = \langle x, \lcd^2 y \rangle\]
 for all $x,y\in V$.

 To prove \eqref{eq:disjointness}, let $x\in V$ be such that $\lhd \lcd x = 0$. We get
 \[ 0 = \langle \lhd\lcd x , y \rangle = \langle \lcd x, \lcd y \rangle \]
 for all $y\in V$. By our assumptions is the bilinear (sesquilinear) map $(x,y) \to \langle \lcd x, \lcd y \rangle$ non-degenerate and hence we conclude that $\lcd x = 0$. Exchanging the roles of $\lhd$ and $\lcd$ in this computation we get that also $\lcd \lhd x =0$ implies $\lhd x=0$.

 Orthogonality of the Hodge resolution follows from $\ker \lhd = (\im\lcd)^\perp$ and $\ker \lcd = (\im \lhd)^\perp$.
\end{proof}
\begin{corollary}
 Let $V$ be endowed with a scalar product $\langle \, , \, \rangle$ and a differential $\lhd$. Let $\lcd$ be the adjoint of $\lhd$. Then $\lhd$ and $\lcd$ are disjoint operators. 
\end{corollary}
\begin{proof}
 This is trivial as the restriction of $\langle \, , \, \rangle$ to any subspace is non-degenerate.
\end{proof}

 If $V$ is infinite-dimensional and $L$ is any closed subspace of finite codimension, then any algebraic complementary subspace to $L$ in $V$ is also a topological complement (\cite{schaefer_topological_1999}). In these cases we demand that $\lhd$ and $\lcd$ are continuous with closed image\footnote{We note that for a finite-dimensional vector space $V$ are these topological conditions automatically satisfied for all linear operators.}

The authors of \cite{huang_dirac_2006} proved that if $\rep{V}$ is a unitarizable $(\lie{g},K)$-module and $\lie{u}_+$ is abelian nilradical of a $\theta$-stable parabolic subalgebra\footnote{See the definition \ref{def:thetastableparabolic}.}, then the Hodge decomposition of $C_\bullet(\lie{u}_+,\rep{V})$ is still valid. 

\subsection{Kostant theorem and Kostant modules}

%\subsection{Hodge decompositions and Kostant's theorem}
%TODO konecne rozmerny pripad

%\begin{definition}
%A real parabolic subalgebra (of some real form of $\lie{g}$, possibly different from $\lie{g}_0$) $\lie{q}_0$ is  \emph{admissible}, if it's complexification $\lie{q}$ is  $\theta$-stable,\footnote{Parabolic subalgebra $\lie{q}$ is $\theta$-stable iff $\lie{q}\cap\overline{\lie{q}}=\lie{l}$ and $\theta \lie{q}=\lie{q}$.} where $\theta$ is the Cartan involution associated to the real form $\lie{g}_0$ and for $\lie{q}=\lie{l}\oplus\lie{u}_+$ it holds that $\lie{l}\subseteq \lie{k}$ and $\lie{p}_+\subseteq\lie{u}_+$.
%\end{definition}

It was proven in \cite{kostant_lie_1961} that for a finite-dimensional $\lie{g}$-representation $\rep{V}$ and a parabolic subalgebra $\lie{p} = \lie{l} \oplus \lie{p}_+$ there is a positive definite scalar product on $C_\bullet(\lie{p}_+,\rep{V})$ with respect to which are the differentials $\lcd$ and $\lhd$ adjoint. It follows that there is a direct sum \emph{Hodge decomposition} of $\lie{l}$-modules $C_\bullet(\lie{p}_+,\rep{V})=\im \lcd \oplus  \ker \lap \oplus \im \lhd$ and moreover $\ker \lhd = \ker \lap \oplus \im \lhd$ and $\ker \lcd = \ker \lap \oplus \im \lcd$. It follows that $H_\bullet(\lie{p}_+,\rep{V})\simeq \ker \lap \simeq H^\bullet(\lie{p}_-,\rep{V})$.

\begin{theorem}[\cite{kostant_lie_1961}]
	Let $\lie{g}$ be a complex simple Lie algebra and let $\lie{p}_\Sigma$ be a standard parabolic subalgebra with Levi decomposition $\lie{p}_\Sigma = \lie{l}_\Sigma \oplus \lie{u}_\Sigma$. Let $W^\Sigma$ be the poset of minimal coset representatives. For every $\lie{g}$-integral and $\lie{g}$-dominant weight $\lambda$ there is isomorphism of $\lie{l}_\Sigma$ modules
	\begin{equation}
		 H^i(\lie{u}_\Sigma, L(\lambda))\simeq \bigoplus_{\substack{w\in W^{\Sigma}\\ l(w) = i}} F(w \cdot \lambda),
	\end{equation}
	where $L(\lambda)$ is the finite dimensional $\lie{g}$-module with highest weight $\lambda$ and $F(w\cdot \lambda)$ are finite dimensional $\lie{l}_\Sigma$ modules with highest weights $w \cdot \lambda$ where $w \cdot \lambda = w(\lambda + \rho) - \rho$ is the affine action of $w \in W.$
\end{theorem}

This theorem gave rise to a definition of \emph{Kostant modules} which are (not necessarily finite-dimensional) modules for which there is a similar formula for cohomology. It started with \cite{collingwood_mathfrakn-homology_1985} and was later developed in \cite{boe_kostant_2009}. The definition roughly says that a simple highest weight module $L$ is Kostant module if there is a graded poset $P$ with a length function in $W$ such that the $i$th cohomology is given by the affine action of elements of $P$ of length $i$. For simple modules with regular weight this poset $P$ is given by (sub)poset of minimal coset representatives $W^\Sigma$. For singular weights the situation is much more complicated, the ``working definition'' is that the poset $P$ is given by $W^{\Sigma, J}$ from theorem \ref{thm:es_equivalence}. We refer to \cite{boe_kostant_2009} and \cite{enright_diagrams_2014} for further details and examples. For our applications we only need to know that the unitarizable highest weight modules of chapter \ref{ch:unitarizable} are examples of Kostant modules.

The article \cite{boe_kostant_2009} classifies Kostant modules for arbitrary blocks of category $\mathcal{O}^\lie{p}$ for $|1|$-graded parabolic $\lie{p} \leq \lie{g}$. The classification is given by certain subdiagrams of the Dynkin diagram. Given subdiagram of the Dynkin diagram one can use the Enright-Shelton equivalence functor to simple modules of the Lie algebra defined by the smaller diagram and one obtains Kostant modules for the bigger Lie algebra. Moreover all Kostant modules arise in this way. Let $J$ denote the set of singular simple roots, i.e. \(J= \{ \alpha \in \sroots \,|\, (\lambda+\rho,\alpha^\vee) = 0 \}.\) The columns $\alpha$ and $\alpha'$ give the crossed roots in the Dynkin diagrams. Each Kostant module corresponds to a connected subdiagram of $\mathcal{D}'$ containing $\alpha'$. Each such subdiagram gives a parabolic pair and the Kostant modules correspond to finite-dimensional modules for the semisimple part of the Levi factor of $\mathcal{D}'$.

\begin{table}[H] 
%\begin{center}
\begin{tabular}{cccccc}
$\mathcal{D}$ &  $\alpha$ & $|J|$ & $\mathcal{D}'$ & $\alpha'$    \\[2pt] \hline 
$(A_n,A_{r-1}\times A_{n-r})$ & $\a_r$   & $t$ & $(A_{n-2t},A_{r-t-1}\times A_{n-r-t})$ & $\a_{r-t}$ \\
$(B_n,B_{n-1})$ & $\alpha_1$   & 1 (short) & $\varnothing$  & --  \\
$(B_n,B_{n-1})$ & $\a_1$   & 1 (long) & $\varnothing$ & --  \\
$(C_n,A_{n-1})$ & $\a_n$   & $t$ (all short) & $(D_{n+1-2t},A_{n-2t})$  & $\a_{n+1-2t}$  \\
$(C_n,A_{n-1})$ & $\a_n$   & $t$ (1 long) & $(D_{n+1-2t},A_{n-2t})$  & $\a_{n+1-2t}$ \\
$(D_n,D_{n-1})$ & $\a_1$   & 1 & $(A_1,\varnothing)$  & $\a_1$  \\
$(D_n,D_{n-1})$ & $\a_1$   & 2 ($\{n-1,n\}$) & $\varnothing$  & --   \\
$(D_n,A_{n-1})$ & $\a_n$   & $t$ & $(D_{n-2t},A_{n-2t-1})$  & $\a_{n-2t}$  \\
$(E_6,D_5)$ & $\a_6$   & 1 & $(A_5,A_4)$  & $\a_5$ \\
$(E_6,D_5)$ & $\a_6$   & 2 & $\varnothing$  & --  \\
$(E_7,E_6)$ & $\a_7$   & 1 & $(D_6,D_5)$  & $\a_1$   \\
$(E_7,E_6)$ & $\a_7$   & 2 & $(A_1,\varnothing)$  & $\a_1$  \\
$(E_7,E_6)$ & $\a_7$   & 3 ($\{2,5,7\}$) & $\varnothing$  & --   \\[2pt] \hline
\end{tabular}
\medskip
\caption{Data for singular Hermitian symmetric categories}\label{tbl:sing}
%\end{center}
\end{table}


\section{Harish-Chandra modules and their globalizations}

Our ultimate goal is to use unitarizable highest weight modules in geometric applications as in \cite{tucek_yamabe_2012}. For these applications we actually need modules which are naturally $\overline{P}$-modules but not $P$-modules which poses a serious problem as the structure group of our Cartan geometry is $P$ and thus we need some way to ``globalize'' the unitarizable $\overline{P}$-modules into $P$-modules. There is a well known theory of globalizations which we review here following \cite{vogan_unitary_2008}. For technical reasons, we will actually need to use so called \emph{formal globalization}\footnote{Denoted by $\mathbb{V}^{-K}$ in \cite{vogan_unitary_2008}.} which is not strictly speaking globalization as it is not representation of $G$. It is, however, representation of $P.$

Fix a maximal compact subgroup $K_{0}$ of a real Lie group $G_{0}$. Suppose we have
a linear action of $K_{0}$ on a complex vector space $M$. A vector $m\in M$
is called $K_{0}-$\emph{finite} if the span of the $K_{0}-$orbit of $m$ is
finite-dimensional and if the action of $K_{0}$ on this subspace is
continuous. The linear action of $K_{0}$ on $M$ is called $K_{0}-$\emph{%
finite} when every vector is $K_{0}-$finite. By definition, \emph{Harish-Chandra module} for $G_{0}$ is a finite length $\mathfrak{g}-$module equipped with a compatible, $K_{0}-$finite, linear action. One knows that
an irreducible $K_{0}-$module has finite multiplicity in a Harish-Chandra
module. For our purposes, it will also be useful to refer to a category of
\emph{good }$K_{0}-$modules. A \emph{good }$K_{0}-$module is a
locally finite module such that each irreducible $K_{0}-$module has finite
multiplicity therein.

 A representation of $G_{0}$ on a complete locally convex
topological vector space $V$ is called \emph{admissible} if $V$ has finite
length (with respect to closed invariant subspaces) and if each irreducible $%
K_{0}-$module has finite multiplicity in $V$. When $V$ is admissible, then
each $K_{0}-$finite vector in $V$ is differentiable and the subspace of $%
K_{0}-$finite vectors is a Harish-Chandra module. The representation is
called \emph{smooth} if every vector in $V$ is differentiable. In this case,
$V$ is a $\mathfrak{g}-$module. For example, it is known that an admissible
representation in a Banach space is smooth if and only if the representation
is finite-dimensional.

Given a Harish-Chandra module $M$, \emph{a globalization} $M_{%
\text{glob}}$ \emph{of }$M$ is an admissible representation of $G_{0}$ whose
underlying $(\mathfrak{g},K_{0})-$module of $K_{0}-$finite vectors is
isomorphic to $M$. By now, four canonical globalizations of Harish-Chandra
modules are known to exist. These are: the smooth globalization of Casselman
and Wallach, its dual (called: the distribution globalization),
Schmid's minimal globalization and its dual (the maximal
globalization). All four globalizations are smooth and functorial.


The \emph{minimal globalization }$M_{\text{min}}$ of a
Harish-Chandra module $M$ is uniquely characterized by the property that any
$(\mathfrak{g},K_{0})-$equivariant linear map of $M$ onto the $K_{0}-$finite
vectors of an admissible representation $V$ lifts to a unique, continuous $%
G_{0}-$equivariant linear map of $M_{\text{min}}$ into $V$. In particular, $%
M_{\text{min}}$ embeds $G_{0}-$equivariantly and continuously into any
globalization of $M$. The construction of the minimal globalization shows
that it's realized on a \emph{DNF space}. This means that its continuous
dual, in the strong topology, is a nuclear Frech\'{e}t space. One knows that
$M_{\text{min}}$ consists of analytic vectors and that it surjects onto the
analytic vectors in any Banach space globalization. Like each of the canonical
globalizations, the minimal globalization is functorially exact. In
particular, a closed $G_{0}-$invariant subspace of a minimal globalization
is the minimal globalization of its underlying Harish-Chandra module and a
continuous $G_{0}-$equivariant linear map between minimal globalizations has
closed range.

To characterize the maximal globalization, we introduce the $%
K_{0}- $finite dual on the category of Harish-Chandra modules. In
particular, let $M $ be a Harish-Chandra module. Then the algebraic dual $%
M^{\ast }$ of $M$ is a $\mathfrak{g}-$module and a $K_{0}-$module, but in
general not $K_{0}-$finite. We define $M^{\vee }$, \emph{the} $K_{0}-$\emph{%
finite (or Harish-Chandra) dual to} $M$, to be the subspace of $K_{0}-$%
finite vectors in $M^{\ast }$. Thus $M^{\vee }$ is a Harish-Chandra module.
In fact, the functor $M\mapsto M^{\vee }$ is exact on the category of good $%
K_{0}-$modules. We also have the formula
\begin{equation*}
\left( M^{\vee }\right) ^{\vee }\cong M\text{.}
\end{equation*}%
The maximal globalization $M_{\text{max}}$ of $M$ can be defined by the
equation%
\begin{equation*}
M_{\text{max}}=\left( \left( M^{\vee }\right) _{\text{min}}\right) ^{\prime }
\end{equation*}%
where the last prime denotes the continuous dual equipped with the strong
topology. In particular, $M_{\text{max}}$ is a globalization of $M$. Observe
that the maximal globalization is an exact functor, since all functors used
in the definition are exact. Because of the minimal property of $M_{\text{min%
}}$, it follows that any globalization of $M$ embeds continuously and
equivariantly into $M_{\text{max}}$. Note that the continuous dual of a
maximal globalization is the minimal globalization of the dual
Harish-Chandra module.


Writing the elements of a Harish-Chandra module as almost everywhere zero sequences we can actually identify these globalizations as vector spaces of sequences with infinitely many nonzero elements which moreover satisfy certain growth conditions. Namely, fixing an $K$-invariant scalar product on each $K$-representation, the minimal globalization consists of those sequences whose $K$-norms decrease exponentially fast and the maximal globalization consists of sequences whose $K$-norms are less than exponentially increasing. 


%%%%%%%%%%%%%%%%%%%%%%%%%%%%%%%%%%%%%%%%%%%%%%%
From now on consider the Hermitian symmetric situation, where $\lie{k}$ is actually the Levi subalgebra of a parabolic subalgebra with abelian radical. Consider the simple quotient $L(\lambda)$ of the parabolic Verma module $M(\lambda)$ induced from the opposite parabolic algebra $\oppar$. It is not a $P$-representation but we would like induce from it a $P$-representation in order to be able to use it in geometrical applications. Let $L(\lambda)=\bigoplus_{\mu\in \widehat{K}_\C} L_\mu$ be the decomposition of $L(\lambda)$ into $K_\C$-types. Each $L_\mu$ is contained in some $S^k(\lie{p}_+,F(\lambda))$ modulo the maximal submodule of $M(\lambda)$ and the algebra $\lie{p}_+$ acts  as multiplication by a variable, while $\lie{p}_-$ acts basically as differentiation. To formalize this write $L(\lambda) = \bigoplus_{\mu\in\widehat{K_\C}, k\in\mathbb{N}} L_{\mu,k}$ where $L_{\mu,k}$ is the $K_\C$-type contained in $S^k(\lie{p}_+,F(\lambda))$ and note that $\lie{p}_+ (L_{\mu,k}) \subset L_{\mu,k+1}$.


The \emph{formal globalization} or rather \emph{formal completion} of $L(\lambda)$ is defined as   $\overline{L(\lambda)}=\prod_{\mu\in \widehat{K}_\C} L_\mu$ (product of topological vector spaces).  Since each $K_\C$-type is finite-dimensional and each $S^k(\lie{p}_+,F(\lambda))$ contains only finitely many irreducible $K_\C$-representations, we can write it as
\[\overline{L(\lambda)} = \prod_{k\in\mathbb{N}} L_k,\] %TODO predelat do lemmatu s dukazem?
where $L_k = \bigoplus_{\mu\in\widehat{K_c}} L_{\mu,k}$ is a finite sum. The action of $\lie{p}_+$ works as a right shift: $\lie{p}_+(L_k)\subset L_{k+1}$. The $(\lie{g},K)$-module $L(\lambda)$ can be realized as a space of polynomials with values in a finite dimensional $K$-representation $F_\lambda$ and it's formal completion can be thought of as a space of formal power series with values in $F_\lambda$. %TODO posledni vetu predelat do lematu s dukazem?

Now it is easy to see that the formal globalization is representation of $P$, since the action of $\lie{p}_+$ on $\overline{L(\lambda)}$ integrates without any problems. Let $X$ be an arbitrary element of $\lie{p}_+.$ The exponential $u:=\exp(tX)v$ is defined for $v\in\overline{L(\lambda)}$ as usual by $\sum_{i=0}^\infty \frac{t^i}{i!} X^iv$. This is well defined for $v$ if and only if each component $u_\mu$ is well defined. If $u_\mu$ is contained in some $S^k(\lie{p}_+,F(\lambda))$, then it is a sum of at most $k$ elements of lower or equal geometric weight.


\begin{proposition}
 Suppose that the Lie algebra homology and cohomology differentials are disjoint for $\mathbb{V}$, i.e. $\ker \lhd \cap \im \lcd = 0$ and $\ker \lcd \cap \im \lhd = 0$. The differentials are disjoint for the formal completion of $\mathbb{V}$ as well. Moreover, if the homology (or equivalently cohomology) of $\mathbb{V}$ is finite-dimensional, then it is equal to the cohomology of the formal completion.
\end{proposition}

\begin{proof}

Let $\mathbb{W}$ be the direct sum of chain spaces of $\lhd$. Elements of the formal completion (of an infinite-dimensional module) $\mathbb{W}$ can be written as infinite tuples of elements of $\mathbb{W}$
$$
u\in\overline{\mathbb{W}} \leftrightarrow u = (u_i)_{i\in I},
$$
where the index set $I$ runs over all eigenvalues of the action of $E$ on $\mathbb{W}$. 

For a contradiction, suppose that there is nonzero $u$ in $\ker \lhd \cap \im \lcd$. Without a loss of generality, we can consider $u$ to be an element of the $k$-th chain space. Since $u$ is nonzero, there must exists $i$ such that $u_i\neq 0$. Both differentials are natural, which means that they commute with restriction from $\overline{\mathbb{W}}$ to $\mathbb{W}$. They are moreover $K_\mathbb{C}$-invariant, which in particular means that they commute with the action of the grading element, which means that they act component-wise on $(u_i)_{i \in I}$. It follows that if we take $\widetilde{u}= (\widetilde{u_j})\in \mathbb{W} \subseteq \overline{\mathbb{W}}$ to be zero  for $j\neq i$ and $\widetilde{u_i} = u_i$, then we have $\lhd \widetilde{u} = 0$.  Since $u\in \im \lcd$, there must exists a $v\in\overline{\mathbb{W}}$ such that $\lcd v = u$ and we can define in the same way as before $\widetilde{v}\in\mathbb{W}$
such that $\lcd \widetilde{v} = \widetilde{u}$. Thus we see that there is a non-zero element in $\ker \lhd \cap \im \lcd\cap \Lambda^k\otimes \mathbb{W}$. This is a contradiction with the assumption that $\lhd$ and $\lcd$ are disjoint for $\mathbb{V}$.


Basically, the same argument proves that the cohomology must stay the same if it is finite-dimensional. From the Hodge theory (which is implied by the disjointness) we know that any element of the cohomology group can be uniquely represented by an element in $\ker (\lhd + \lcd)$. Let $D=\lhd + \lcd$ and suppose for a contradiction that there is a nonzero element $(u_i)$ in $\ker D \setminus \ker D_{|\mathbb{W}}$, in the $k$-th chain space of $\mathbb{V}$. Pick $i_0$ such that $u_{i_0} \neq 0$ and $i_0$ is bigger than any eigenvalue of $E$ on $\ker D_{|\mathbb{W}}$.\footnote{This is well defined because of the finite-dimensionality of $\ker D_{|\mathbb{W}}$.} Since $D$ is again $K_\mathbb{C}$-invariant we proceed as before and construct $\widetilde{u}\in\mathbb{W}\cap\ker D$. The choice of $i_0$ now leads to the desired contradiction.

\end{proof}


