\chapter*{Conclusion}
\addcontentsline{toc}{chapter}{Conclusion}

We have calculated cohomology of all unitarziable weights in the conformal Hermitian symmetric case and obtained partial results in the remaining ones. We found explicit singular vectors in some cases and we have shown in general that these modules give rise to sequences of differential operators similarly to finite-dimensional $\lie{g}$ representations. The formula for cohomology of unitarizable weights is based on certain equivalence of categories that transports the results from the finite-dimensional representations to the unitarizable ones. What does one obtain by transporting unitarizable modules from the smaller rank? 

The class of modules for which there exists the BGG resolution (in the flat case) is strictly bigger than the class of unitarizable highest weight modules. It is not clear whether there are curved analogues for these nonunitarizable Kostant modules. Another natural question is whether our modification of the Calderbank--Diemer construction works also for some other globalization apart from the formal one. And of course, having explicit expression for these operators is also highly desirable. 

We have made some progress towards elementary description of the smaller of the two exceptional Hermitian symmetric cases. Description of this space using octonions in a similar vein to \cite{pazourek_hyperplane_2011} based on \cite{baez_supersymmetry} is current work in progress. 

