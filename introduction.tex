\chapter*{Introduction}
\addcontentsline{toc}{chapter}{Introduction}

This thesis started with investigation of the higher symmetries of the Laplacian \cite{eastwood_higher_2005, tucek_construction_2011}. These are differential operators on $\mathbb{R}^n$ that preserve the space of harmonic functions and one can construct them using the so called ambient construction. The explicit construction of invariant operators is in general difficult problem. For parabolic geometries (of which is the conformal geometry prominent example) this is equivalent to finding homomorphism of parabolic Verma modules. These homoromphisms are in turn completely determined by so called singular vectors. Recently there has been new development which uses isomorphism of parabolic Verma modules with differential operators with constant coefficients where the action of the Lie algebra in question is rather complicated \cite{krizka_invariant_2017} but which is rather straightforward to calculate for the so called $|1|$-graded parabolic geometries \cite{kobayashi_branching_2015}. One can then apply algebraic Fourier transform to reinterpret the set of singular vectors as polynomial solutions of system of PDEs with polynomial coefficients. Since these PDEs come from the Lie algebra action one can use representation theory to considerably simplify this problem. We do so in the section \ref{sec:invariant}. 

Not all these invariant differential operators are natural. Meaning that not all invariant operators acting between sections of associated bundles over the homogeneous spaces $G/P$ admit curved analogues which would act on sections of bundles associated to a Cartan geometry modeled on the pair $(G, P)$. The counterexample is a conformally invariant power of the Laplace operator \cite{graham_conformally_1992, gover_conformally_2004}. There is however a big class of operators that always admits curved analogues and moreover with quite favourable properties. For each finite-dimensional irreducible representation of $\lie{g}$ there is a sequence of such operators which even form a complex in the flat case. This is the famous resolution of Bernstein, Gelfand and Gelfand \cite{bernstein_differential_1975}. The existence of curved analogues was proved in \cite{cap_bernstein-gelfand-gelfand_2001} and this construction was simplified in \cite{calderbank_differential_2001}. Among these so called BGG operators are various interesting `geometric' operators whose kernels give e.g. projective Killing tensor fields or conformal Killing spinors. All these operators have the property that the dimension of their solution space is bounded above by the dimension of the so called tractor bundle associated to the representation one has started with. In other words, these operators are all overdetermined as this property holds even locally and thus for any small subset of the manifold the dimension of solution space is finite-dimensional. Thus if one wants to obtain operators such as conformally invariant modification of the Laplace operator, one has to consider infinite-dimensional representations. Moreover, the article \cite{shaynkman_unfolded_2006} shows that in all cases the operator is determined by the nilpotent cohomology as in the finite dimensional case. It turns out that the construction of \cite{calderbank_differential_2001} with little modification goes through also for a certain infinite-dimensional representation. Basically the same proof works for any unitarizable module of highest weight. These modules occur only for so called Hermitian symmetric spaces which are recalled in chapter \ref{ch:hermitian}. There are five classical series of such spaces and two exceptional ones. Thus one is lead to investigate the nilpotent cohomology of this class of modules. This is investigated in chapter \ref{ch:unitarizable}. In contrast with the finite-dimensional representations where the structure of the cohomology for abelian nilradical is much simpler than for nilradical of a general parabolic subalgebra, the cohomology of unitarizable highest weight modules contains considerable combinatorial obstacle. Namely, one has to consider intersection of the cone(s) of unitarizable weights with the cones given by root hyperplanes. This can get pretty complicated. In the end, a computer program, whose listing is in appendix \ref{app:source}, was written and can be used to compute not only cohomologies of unitarizable weights but also the finite-dimensional ones even in the relative case. A table of results for low ranks is compiled in appendix \ref{app:cohomology}.


Going back to the starting example of the ambient construction for $\mathbb{R}^n$ we can motivate the last original part of this thesis. The ambient construction in the flat case amounts to working on $\mathbb{R}^{n+1, 1}$ or rather it's projectivization where the $\mathbb{R}^n$ is conformally compactified as the sphere $S^n.$ The isometry group $\mathrm{SO}(n+1,1)$ of the ambient space induces conformal transformation of this embedded $S^n$. The projectivization of the two-sheeted hyperboloid of the defining quadric provides a model for the hyperbolic space. All points of the projectivization of $\mathbb{R}^{n+1, 1}$ are lines and as such can be identified with orthogonal projectors of rank 1. There is a well known scalar product on matrices and one can wonder what kind of Riemannian strucutre it induces on subvariety of rank 1 idempotents. This leads to uniform description of octonionic planes which were usually defined as homogeneous spaces of their isometry group. These groups are of type $\mathrm{F}_4$ and as such are a bit tricky to define. The usual way is to define them as automorphism groups of certain Jordan algebras. The article \cite{held_semi-riemannian_2009} provided elementary definition of these octonionic planes on case by case basis. The caveat being that the identification with the homogeneous spaces uses classification of so called Osserman manifolds. We provide uniform and much simpler way to obtain this  elementary description of octonionic planes in section \ref{sec:octo}. Complexification of the projective octonionic plane then gives one of the two exceptional Hermitian symmetric spaces.


