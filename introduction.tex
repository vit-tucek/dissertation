\chapter*{Introduction}
\addcontentsline{toc}{chapter}{Introduction}


The general problem is to find all natural differential operators  between sections of natural vector bundles on some geometric category (e.g. on conformal or projective manifolds). The naturality implies that on an open subset of a homogeneous space (which is the canonical model for the `geometry' under consideration) the natural vector bundles are the homogeneous vector bundles and natural differential operators are just the invariant differential operators. Not all invariant differential operators however are natural. The counterexample is a power of the Laplace operator (see \cite{graham_conformally_1992} and \cite{gover_conformally_2004}).

It was proven in \cite{cap_bernstein-gelfand-gelfand_2001} that there is a large class of invariant operators which are in fact natural.  Among the BGG operators are various interesting `geometric' operators whose kernels give e.g. conformal Killing tensor fields. Consequently, the BGG operators were much studied --- see e.g. \cite{cap_projective_2010} for a recent applications. The construction of these operators was much simplified in \cite{calderbank_differential_2001}.

The construction of BGG operators starts with a general parabolic geometry modelled on a parabolic pair $(G,P)$ and a finite-dimensional $(\lie{g},P)$-representation. In this work we show that this construction goes through also for a certain infinite-dimensional representation. The representation we will consider is a formal globalization of a unitarizable highest weight module. That is a module which is both a $(\lie{g},K)$-module and a highest weight module for $\lie{g}$. In this article we show that the BGG construction by Calderbank -- Diemer for this module goes through and  yields the conformally invariant Yamabe operator which is a conformally invariant analogou of Laplace -- Beltrami operator.