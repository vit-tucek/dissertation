\chapter{Source code}
The Bruhat graphs in this work were produced with the help of the following code written for the mathematical software called sage \cite{sage}. Most of the code was submitted in a form of patch with respect to the version 5.12.

\begin{minted}[fontsize=\footnotesize]{python}
######################################################################
##############             HELPER FUNCTIONS             ##############
######################################################################

from sage.graphs.graph_latex import setup_latex_preamble
setup_latex_preamble()

def one_graded_from_dynkin(D, crossed_node):
    ct = D.cartan_type() # D might be relabeled, we need to get rid of that
    ct = CartanType([ct.type(), ct.rank()])
    index_set = list(ct.index_set())
    index_set.remove(crossed_node)
    return ct, index_set

def setup(CT, index_set):
    W = WeylGroup(CT, prefix="s")
    AS = W.domain()
    FW = AS.fundamental_weights()
    vFW = FW.map(lambda v: v.to_vector())
    RP = AS.root_poset(facade=True)
    nonparabolic_roots = [x.to_ambient() for x in AS.root_system.root_lattice().positive_roots_nonparabolic(index_set=index_set)]
    nRP = RP.subposet(nonparabolic_roots)
    rho = AS.rho()
    return W, AS, FW, vFW, rho, nRP

def inject_positive_integer_variables(names, p=0, q=0):
    """
    If names is a string (e.g. "A") this function injects into the workspace n variables named Ap till Aq inclusive.
    Otherwise it is assumed that names is a list of variables that will be injected.
    They are assumed to take only nonnegative integral values.
    """
    def inject_name(name, i=None):
        if i is not None:
            name = name + str(i)
        var(name)
        eval("assume({s} >= 0, ({s}, 'integer'))".format(s=name))

    if isinstance(names, basestring):
        for i in range(p, q+1):
            inject_name(names, i)
    else:
        for name in names:
            inject_name(name)

def setup_cone(variable_list, cone_str):
    """
    Deletes all assumptions, defines variables and declares them to be integral and nonnegative.

    variable_list is either a list of strings or 3-tuple containing name and range for the variables (see inject_positive_integer_variables)
    """
    forget()
    global vFW
    print("\n Initial assumptions:")
    print(assumptions())
    if isinstance(variable_list, list):
        inject_positive_integer_variables(variable_list)
    else:
        inject_positive_integer_variables(*variable_list)
    print("Cone: %s with assumptions:" % cone_str)
    print(assumptions())
    print("\n")
    exec(cone_str, globals())
    #exec(cone_str, globals(), locals())
    #exec(cone_str, locals(), globals())

class RootWithScalarProduct:
    """
    Use this to relabel graphs of positive roots with scalar product with given weight v.
    """
    def __init__(self, r, v):
        self.root = r
        self.scalarproduct = v.dot_product(r.associated_coroot().to_vector())

    def _latex_(self):
        return "(%s, %s)" % (latex(self.scalarproduct), latex(self.root))

    def __str__(self):
        return str(self.scalarproduct)

    def __repr__(self):
        return repr(self.scalarproduct)

def poset_scalar_product(poset, v, only_nonnegative=True):
    """
    Returns LaTeX code of poset of roots whose nodes were labeled by inner product of those roots with give weight v.
    """
    p = poset.relabel(lambda r: RootWithScalarProduct(r, v))
    if only_nonnegative:
        p = p.subposet([x for x in p if not(x.scalarproduct < 0)])
    if p.is_empty():
        print("Poset of scalar products is empty.")
    return p

def _fix_basis_latex(string):
    """
    Basis of Ambient spaces are indexed by e_0, ..., e_{n-1} instead of conventional \epsilon_1, ..., \epsilon_n.
    This functions returns string with fixed LaTeX source. You can render its output in notebook by calling latex.eval(...)
    """
    import re
    def shift_number(matchobj):
        return "e_{%d}" % (int(matchobj.group(1)) + 1)

    index_re = re.compile("e_{(\d+)}")
    return index_re.sub(shift_number, string).replace("e_{", "\epsilon_{")

def fix_basis_latex(obj):
    return _fix_basis_latex(str(latex(obj))).replace("0000000000000", "")

def get_poset_latex(poset, orientation="up"):
    hd = poset.hasse_diagram()
    if orientation != "up":
        hd.set_latex_options(rankdir=orientation)

    return fix_basis_latex(latex(hd))

def save_diagram(name, poset, orientation="up"):
    import os
    with open(os.path.join("diagrams", name + ".tikz"), 'w') as f:
        f.write(get_poset_latex(poset, orientation=orientation))

######################################################################
##############  Parabolic enhancements for Weyl groups  ##############
######################################################################

import sage.combinat.root_system.weyl_group as wg

def parabolic_bruhat_graph(self, index_set = None, side="right"):
    """
    Returns the Hasse graph of the poset ``self.bruhat_poset(index_set,side)`` with edges labeled by the cover relation
    """
    elements = self.minimal_representatives(index_set, side)
    covers =[(x,y)  for y in elements for x in y.bruhat_lower_covers() if x in elements]
    res = DiGraph()
    for u,v in covers:
        res.add_edge(u,v,v.inverse()*u)
    return res

def parabolic_weight_graph(self, weight, index_set=None,side="right"):
    elements = self.minimal_representatives(index_set,side)
    wl0 = self.long_element(index_set)
    #covers =[(wl0*x,wl0*y)  for y in elements for x in y.bruhat_lower_covers() if x in elements] # funguje jen pro "right"
    covers =[(x,y)  for y in elements for x in y.bruhat_lower_covers() if x in elements]
    res = DiGraph()
    rho = weight.parent().rho()
    v = weight + rho
    def act_on_weight(v,x):
        return str((v.weyl_action(x) - rho).to_dominant_chamber(index_set).to_vector())
    for x,y in covers:
        #a = v.weyl_action(x) - rho
        #b = v.weyl_action(y) - rho
        #res.add_edge(str(a.to_vector()),str(b.to_vector()))
        a = act_on_weight(v,x)
        b = act_on_weight(v,y)
        res.add_edge(a,b)
    return res

def parabolic_weight_graph_enum(self, weight, index_set=None, side="right"):
    elements = [x for x in enumerate(self.minimal_representatives(index_set,side))]
    covers =[(x,y)  for y in elements for x in elements if x[1] in y[1].bruhat_lower_covers()]
    res = DiGraph()
    rho = weight.parent().rho()
    v = weight + rho
    def act_on_weight(v,x):
        return str((v.weyl_action(x) - rho).to_dominant_chamber(index_set).to_vector())
    for x,y in covers:
        a = str(x[0]) + ":" + act_on_weight(v,x[1])
        b = str(y[0]) + ":" + act_on_weight(v,y[1])
        res.add_edge(a,b)
    return res

def parabolic_poset(self, Levi_indices, side="right"):
    # returns a poset of minimal representatives of W_S \ W
    # self is a finite-dimensional Weyl group
    # first we compute orbit of the characteristic vector of our parabolic subalgebra
    # this is for representatives of left cosets; to obtain representatives for right cosets just take the inverse
    elements = self.minimal_representatives(Levi_indices, side)
    #since our Weyl elements should be already reduced (?), we could optimize this step by constructing the cover relations directly thus reducing quadratic complexity to linear
    covers = tuple([x,y]  for y in elements for x in y.bruhat_lower_covers() if x in elements)
    return Poset( (elements, covers), cover_relations = True)

def parabolic_weight_poset(self, weight, Levi_indices, side="right", relative_index_set=None):
    rho = weight.parent().rho()
    v = weight + rho
    elements = self.minimal_representatives(Levi_indices, side, relative_index_set=relative_index_set)
    covers = tuple([x,y]  for y in elements for x in y.bruhat_lower_covers() if x in elements)
    labels = {}
    for x in elements:
        labels[x] = str((v.weyl_action(x) - rho).to_dominant_chamber(Levi_indices).to_vector())
    return Poset( (elements, covers), cover_relations = True, element_labels=labels)

def minimal_representatives(self, index_set=None, side="right", relative_index_set=None):
    """
    Returns the set of minimal coset representatives of ``self`` by a parabolic subgroup.

    INPUT:

    - ``index_set`` - a subset (or iterable) of the nodes of the Dynkin diagram, empty by default, denotes the generators of the Levi part
    - ``side`` - 'left' or 'right' (default)
    - ``relative_index_set`` - superset of index_set for the relative Case, again determines the Levi part

    See documentation of ``self.bruhat_poset`` for more details.

    The output is equivalent to ``set(w.coset_representative(index_set,side))``

    but this routine is much faster. For explanation of the algorithm see e.g. Cap, Slovak:
    Parabolic geometries, p. 332

    EXAMPLES::

        sage: G = WeylGroup(CartanType("A4"),prefix="s")
        sage: index_set = [1,3,4]
        sage: side = "left"
        sage: a = set(w for w in G.minimal_representatives(index_set,side))
        sage: b = set(w.coset_representative(index_set,side) for w in G)
        sage: print a.difference(b)
        set([])
    """
    from sage.combinat.root_system.root_system import RootSystem
    from copy import copy

    if side != 'right' and side != 'left':
        raise ValueError, "%s is neither 'right' nor 'left'"%(side)

   #TODO check for relative_index_set being a superset of index_set

    weight_space = RootSystem(self.cartan_type()).weight_space()
    if index_set == None:
        crossed_nodes = set(self.index_set())
        relative_crossed_nodes = set()
    else:
        crossed_nodes = set(self.index_set()).difference(index_set)
        if not relative_index_set:
            relative_index_set = self.index_set()
        relative_crossed_nodes = set(self.index_set()).difference(relative_index_set)
    # the characteristic vector
    rhop = sum([weight_space.fundamental_weight(i) for i in crossed_nodes if not(i in relative_crossed_nodes)])
    '''

    The variable "todo" serves for traversing the orbit of rhop, while the directory "known" serves
elements in the orbit of rhop while known[vec] are paths of simple reflections from rhop to vec.

    '''
    todo = [rhop]
    known = dict()
    known[rhop] = []
    if rhop == 0:
        return set( [self.one()] )
    else:
        while len(todo) > 0:
            vec = todo.pop()
            nonzero_coeffs = [i for i in self.index_set() if (vec.coefficient(i) > 0) and (i in relative_index_set)]
            for i in nonzero_coeffs:
                new_vec = vec.simple_reflection(i)
                new_reflections = copy(known[vec])
                new_reflections.append(i)
                todo.append(new_vec)
                known[new_vec] = new_reflections
        if side =='left':
            return set(self.from_reduced_word(w) for w in known.values())
        else:
            #here we could just take the inverses of w but reversing the list of simple reflections
            return set(self.from_reduced_word(w[::-1]) for w in known.values())

def bruhat_poset(self, index_set = None, side="right", facade = False):
    from sage.combinat.posets.posets import Poset
    elements = self.minimal_representatives(index_set, side)
    # Since our Weyl elements should be already reduced (?), we could
    # optimize this step by constructing the cover relations directly (see Cap, Slovak: Parabolic
    # thus reducing quadratic complexity of the next step to linear. On the other hand, we would
    covers = tuple([x,y] for y in elements for x in  y.bruhat_lower_covers()
                    if x in elements)
    return Poset((self, covers), cover_relations = True, facade=facade)


wg.WeylGroup_gens.minimal_representatives = minimal_representatives
#wg.WeylGroup_gens.bruhat_poset = bruhat_poset
wg.WeylGroup_gens.parabolic_poset = parabolic_poset
wg.WeylGroup_gens.parabolic_bruhat_graph = parabolic_bruhat_graph
wg.WeylGroup_gens.parabolic_weight_graph = parabolic_weight_graph
wg.WeylGroup_gens.parabolic_weight_graph_enum = parabolic_weight_graph_enum
wg.WeylGroup_gens.parabolic_weight_poset = parabolic_weight_poset

######################################################################
##############   Cohomology given by root embeddings    ##############
######################################################################

def get_P_lambda(CT, index_set, side):
    """
    Returns parabolic poset for CT with Levi part given by index_set that consists of minimal representative of `side' cosets.
    """
    W = WeylGroup(CT, prefix="s")
    return W.parabolic_poset(index_set, side)

def W_lambda_weight_poset(P_lambda, embedding):
    """
    Returns poset P_lambda embedded to a bigger Weyl group through embedding
    :embedding:  tuple (simple_roots, W) where simple_roots are embeddings of simple_roots into reflections in W.
    """
    simple_roots, W = embedding
    AS = W.domain()
    for i in simple_roots.keys():
        simple_roots[i] = AS.from_vector(vector(simple_roots[i]))

#       reflections = w.parent().reflections()
#       embedding = {}
#       for i  in simple_roots:
#           embedding[i] = reflections()[simple_roots[i]]
    def embedd(w):
        reflections_word = map(lambda i: simple_roots[i], w.reduced_word())
        return W.from_reduced_word(reflections_word, word_type="all")
    print "Embedding: ", simple_roots
    return P_lambda.relabel(embedd)

def to_fundamental_weights(AS, u):
    """
    u is a dense vector over symbolic ring epsilon basis and this will return its coordinates wrt basis of fundamental weights
    should work better than matrix M as some Cartan Types (i.e. E_6) are implemented in Ambient space of higher dimension than the rank
    """
    return vector([u.dot_product(AS.simple_coroot(i).to_vector()) for i in AS.index_set()])


def get_dominant(AS, u, index_set):
    """
    Returns as provably dominant element in the orbit of u as possible. The orbit is taken with respect to Weyl subgroup generated by indices from index_set. If all coefficients of u are numbers it will return the dominant element.

    If there is no dominant element ends up in infinite loop! TODO BUG?
    """
    negative = True
    while negative:
        for i in index_set:
            c = u.dot_product(AS.simple_coroot(i).to_vector())
            if c < 0:
                break
        else: # we haven't found provably negative coefficient wrt fundamental weights
            negative = False
        if negative:
            u = AS.weyl_group().simple_reflection(i)*u
    return u

def get_W_action(AS, index_set, side, v, node_dist=1.5, with_dynkins=False):
    """
    index_set is index_set determines the Levi part of the space we are embedding into
    """
    class W_action:
        ct = AS.cartan_type()

        #M = Matrix([AS.fundamental_weight(i).to_vector() for i in AS.index_set()]).transpose().inverse() # change of basis from "epsilons" to fundamental weights

        def to_fundamental_weights(self, u):
            """
            u is a dense vector over symbolic ring epsilon basis and this will return its coordinates wrt basis of fundamental weights
            should work better than matrix M as some Cartan Types (i.e. E_6) are implemented in Ambient space of higher dimension than the rank
            """
            return vector([u.dot_product(AS.simple_coroot(i).to_vector()) for i in AS.index_set()])

        def __init__(self, w):
            #AS = w.domain()
            #nw = w.coset_representative(index_set, side) # Enright is wrong!
            nw = w

            #self.result = nw.reduced_word()
            #self.result = M*(2*nw.matrix()*AS.rho().to_vector() - AS.rho().to_vector())
            #self.result = M*(nw.matrix()*v - AS.rho().to_vector())
            self.result = self.to_fundamental_weights(get_dominant(AS, nw.matrix()*v - AS.rho().to_vector(), index_set))
            #self.result = self.to_fundamental_weights(nw.matrix()*v - 1*AS.rho().to_vector())
            #self.result = (AS.from_vector(nw.matrix()*v) - AS.rho()).to_dominant_chamber(index_set).to_weight_space()
            # u, direction = (AS.from_vector(nw.matrix()*v) - AS.rho()).to_dominant_chamber(index_set, reduced_word=True)
#             wl = w.coset_representative(index_set, "left")
#             wr = w.coset_representative(index_set, "right")
#           self.result = u.to_weight_space(), w.reduced_word(), direction, wl.reduced_word(), wr.reduced_word(), (w*wl.inverse()).reduced_word(), (wr.inverse()*w).reduced_word(), (w*wl.inverse())*wl
#             self.result = u.to_weight_space(), w.reduced_word(), direction

        def __repr__(self):
            return repr(w)

        def _latex_(self):
            if with_dynkins:
                global parabolic_index_set
                parabolic_index_set = map(lambda i: latex(self.result[i-1]), index_set) # TODO can give wrong node fill in case there are repeating labels
                def labeling(i):
                    return latex(self.result[i-1])
                dynkin_latex = "\n\n \\begin{tikzpicture}\n" + _fix_basis_latex(self.ct._latex_dynkin_diagram(label=labeling, node_dist=node_dist)) + "\\end{tikzpicture}\n\n "
                return dynkin_latex
            else:
                return latex(self.result)

    return W_action

def to_fundamental_weights(AS, u):
            """
            u is a dense vector over symbolic ring epsilon basis and this will return its coordinates wrt basis of fundamental weights
            should work better than matrix M as some Cartan Types (i.e. E_6) are implemented in Ambient space of higher dimension than the rank
            """
            return vector([u.dot_product(AS.simple_coroot(i).to_vector()) for i in AS.index_set()])

def cohomology_poset(small_CT, small_index_set, simple_roots_embedding, W, big_index_set, v, with_dynkins=False):
    P = get_P_lambda(small_CT, small_index_set, "left")
    eP = W_lambda_weight_poset(P, (simple_roots_embedding, W))
    AS = W.domain()
    action = get_W_action(AS, big_index_set, "left", v, with_dynkins=with_dynkins)
    return eP.relabel(action)


######################################################################
##############    Cohomology of unitarizable modules    ##############
######################################################################

def WG_action(w, v):
    """
    Action of weyl group element w on vector v.
    Workaround for subgroups not containing elements of the supergroup.
    """
    AS = v.parent()
    return AS.from_vector(w.matrix()*v.to_vector())

def get_length_function(positive_roots):
    positive_roots = set(positive_roots)
    @cached_function
    def l(w):
        return len([a for a in positive_roots if WG_action(w.inverse(), -a) in positive_roots])
    return l

def get_generating_roots(weight, index_set):
    """
    Returns a list of roots that generate the reflection subgroup which governs cohomology of unitarizable hihgest weight modules.
    First part of Enright's formula from his paper on u-cohomology.
    The convention is that Verma modules are induced from lambda (i.e. no rho-shift)
    """
    AS = weight.parent() # the ambient space
    rho = AS.rho()
    Psi = [r for r in AS.positive_roots() if r.scalar(rho + weight) == 0]

    if AS.cartan_type()[0] in "BCG":
        print [x.is_short_root() for x in Psi]
        is_there_long_root = any(not(x.is_short_root()) for x in Psi)
    else:
        is_there_long_root = False
    print "Is there long root:", is_there_long_root
    def test_root(r):
        n = r.associated_coroot().scalar(weight + rho) # TODO check coroot calculations
        if is_there_long_root:
            short = r.is_short_root()
        else:
            short = True
        #print r, long_root, short
        return n.is_integer() and n > 0 and short

    nonparabolic_roots = [x.to_ambient() for x in AS.root_system.root_lattice().positive_roots_nonparabolic(index_set=index_set)]
    parabolic_roots = [x.to_ambient() for x in AS.root_system.root_lattice().positive_roots_parabolic(index_set=index_set)]
    #print("Nonparabolic roots: %s" % sorted(nonparabolic_roots))
    Phi = [r for r in nonparabolic_roots if test_root(r) and all(r.scalar(s) == 0 for s in Psi)]
    return Phi, Psi, parabolic_roots, nonparabolic_roots

def generate_subgroup(generators):
    """
    Keep multiplying and taking inverses as long as new elements are constructed.
    Unfortunately, this routine takes too much time in practice.
    """
    new = set(a*b for (a,b) in cartesian_product([generators, generators])).union(set(g.inverse() for g in generators))
    if new == generators:
        return new
    else:
        return generate_subgroup(new)

def DyerN(w):
    W = w.parent()
    return [t for t in W.reflections() if (t*w).length() < w.length()]

def DyerCoxeterGenerators(H):
    return [w for w in H if set(DyerN(w)) == set([w])]

def get_subsystem_data(weight, index_set):
    AS = weight.parent()
    W = AS.weyl_group()
    generating_roots, Psi, parabolic_roots, nonparabolic_roots =  get_generating_roots(weight, index_set)
    reflections = W.reflections()
    generators = set(reflections[r] for r in generating_roots)

    #W_lambda = list(generate_subgroup(generators)) # subgroup generates H as a matrix group and we lose all the WeylGroupElement methods # too slow
    #W_lambda = [W.element_class(W, h) for h in W.subgroup(generators)] # too slow
    W_lambda = W.subgroup(generators)
    W_lambda_reflections = []
    for x in W_lambda:
        g = W.element_class(W, x)
        if g in reflections:
            W_lambda_reflections.append(g)

    # calculate Coxeter generators of the reflection subgroup
    # see [Deodhar] or [Dyer] for proof
    def DyerCoxeterGenerators(H_reflections):
        # optimized version
        #H_reflections = [W.element_class(W, x) for x in reflections if x in H] # WARNING switching H and reflections leads to empty set!
        # H_reflections = [W.element_class(W, x) for x in H if W.element_class(W, x) in reflections] # the previous stopped working in Sage 7.6 # refactored shortly thereafter to assume that we have only reflections at input
        W_length = get_length_function(AS.positive_roots())
        def DyerN(w):
            w = W.element_class(W, w)
            return set(t for t in H_reflections if  W_length(t*w) < W_length(w))

        return [w for w in H_reflections if DyerN(w) == set([w])]

    coxeter_generators = DyerCoxeterGenerators(W_lambda_reflections)

    lambda_positive_roots = [r for r in reflections.keys() if reflections[r] in W_lambda_reflections]
    lambda_simple_roots = [r for r in reflections.keys() if reflections[r] in coxeter_generators]
    lambda_parabolic_roots = [r for r in lambda_positive_roots if r in parabolic_roots]
    lambda_nonparabolic_roots = [r for r in lambda_positive_roots if r in nonparabolic_roots]

    # decompose coset representative according to their length
    from collections import defaultdict
    def is_dominant(v, positive_roots):
        return all(v.scalar(r) > 0 for r in positive_roots)
    lambda_W_c = defaultdict(list)
    rho = AS.rho()
    lambda_length = get_length_function(lambda_positive_roots)
    for w in W_lambda:
        if is_dominant(WG_action(w, rho), lambda_parabolic_roots):
            lambda_W_c[lambda_length(w)].append(w)

    return Psi, generating_roots, lambda_simple_roots, lambda_positive_roots, lambda_parabolic_roots, lambda_nonparabolic_roots, W_lambda, lambda_W_c

# small hack for LaTeXing DynkinDiagrams of generalized flag manifolds
from sage.combinat.root_system.cartan_type import CartanType_abstract as cta
def _my_latex_draw_node(self, x, y, label, position="below=4pt", fill='white'):
        r"""
        Draw (possibly marked [crossed out]) circular node ``i`` at the
        position ``(x,y)`` with node label ``label`` .
        - ``position`` -- position of the label relative to the node
        - ``anchor`` -- (optional) the anchor point for the label
        EXAMPLES::
            sage: CartanType(['A',3])._latex_draw_node(0, 0, 1)
            '\\draw[fill=white] (0 cm, 0 cm) circle (.25cm) node[below=4pt]{$1$};\n'
        """
        global parabolic_index_set
        #print parabolic_index_set, label
        fill = "black" if label in parabolic_index_set else "white"
        return "\\draw[fill={}] ({} cm, {} cm) circle (.1cm) node[{}]{{${}$}};\n".format(fill, x, y, position, label)
cta._latex_draw_node =  _my_latex_draw_node

def examine(weight, index_set, cone=None, only_nonnegative=True, show_diagrams=False, show_latex=False, orientation="up", cartan_type="", with_dynkins=False, **kwargs):
    Psi, generating_roots, lambda_simple_roots, lambda_positive_roots, lambda_parabolic_roots, lambda_nonparabolic_roots, H, lambda_W_c = get_subsystem_data(weight, index_set)
    print("Weight: %s" % weight)
    print("Singular roots: %s" % sorted(Psi))
    print("Set of generating roots: %s" % sorted(generating_roots))
    print("Set of generated roots: %s" % sorted(lambda_positive_roots))
    #show("Set of generated roots:", lambda_positive_roots)
    print("Simple roots: %s" % sorted(lambda_simple_roots))
    #print("Scalar products of pairs of distinct simple roots: %s" % set(u.scalar(v) for (u,v) in cartesian_product([lambda_simple_roots, lambda_simple_roots]) if u != v))

    print("Noncompact lambda-roots: %s" % lambda_nonparabolic_roots)

    AS = weight.parent()
    weight = weight
    if cone is not None:
        print("Translated cone: %s" % cone)
        v = weight.to_vector() + cone + AS.rho().to_vector() # we induce Verma modules from lambda and hence we need to test scalar products with weight shifted by rho
    else:
        v = weight.to_vector() + AS.rho().to_vector() # we induce Verma modules from lambda and hence we need to test scalar products with weight shifted by rho

    nonparabolic_roots = [x.to_ambient() for x in AS.root_system.root_lattice().positive_roots_nonparabolic(index_set=index_set)]
    nRP = AS.root_poset(facade=True).subposet(nonparabolic_roots)
    sP = poset_scalar_product(nRP, v, only_nonnegative=only_nonnegative)
    if sP.is_empty():
        print "Poset of nonnegative scalar products is empty."
        sP_latex = ""
    else:
        sP_latex = get_poset_latex(sP, orientation)
    if  sP_latex != "":
        if show_diagrams:
            print("Scalar products (possibly only the nonnegative ones) of the weight (in the cone) with noncompact roots:")
            _ = latex.eval(sP_latex)
        if show_latex:
            print sP_latex

### BUG
# DynkinDiagram calls CartanMatrix with all its arguments (*args) see line 181 in dynkin_diagram.py
    # this causes error with relabeling for A1, i.e. DynkinDiagram(CartanMatrix([[2]]), ["ahoj"])
    # workaround here is to use Matrix instead of CartanMatrix, but note that the label in Dynkin diagram is wrong!
    # also, DynkinDiagram(CM, index_set=...) doesn't work as intended and produces labeling range(order(CM))
    # print("Cartan matrix:\n%s" % CM)
    # if len(lambda_simple_roots) > 0:
#         D = DynkinDiagram(CM, lambda_simple_roots)
#     else:
#         D = DynkinDiagram(CM, index_set=lambda_simple_roots)

# in the end, we relabel explicitely in case we have rank 1
    CM = Matrix([[rj.scalar(ri.associated_coroot()) for rj in lambda_simple_roots] for ri in lambda_simple_roots])
    D = DynkinDiagram(CM, lambda_simple_roots)
    if len(lambda_simple_roots) == 1:
        D = D.relabel({1: lambda_simple_roots[0]})


    # let's calculate the cohomology of the "shape" given by vertex of the cone
    e = set(D.index_set()).intersection(lambda_nonparabolic_roots).pop() # we know that there is only one simple noncompact root in the lambda-subsystem
    print("Noncompact root in the lambda-subsystem: %s" % e)
    crossed_node = D.index_set().index(e) + 1
    # simple_roots_embedding = dict(enumerate(D.index_set(), 1)) #BUG returns sorted index_set
    simple_roots_embedding = D.cartan_type()._relabelling

    ct, small_index_set = one_graded_from_dynkin(D, crossed_node)
    # print ct, small_index_set, simple_roots_embedding, AS.weyl_group(),index_set, v

    # There is a BUG in DynkinDiagram._latex_ which causes not very nice labeling of nodes in the Dynkin diagram
    #latex.eval(fix_basis_latex(latex(D)))
    #print fix_basis_latex("\n" + latex(D) + "\n")
    # This is a workaround
    labeling = lambda i: latex(i)

    global parabolic_index_set
    parabolic_index_set = [latex(simple_roots_embedding[s]) for s in small_index_set]

    dynkin_latex = "\\begin{tikzpicture}\n" + _fix_basis_latex(D.cartan_type()._latex_dynkin_diagram(labeling)) + "\\end{tikzpicture}"
    #dynkin_latex = dynkin_latex.replace(".25cm", ".15cm").replace("fill=white", "fill=black") # make nodes smaller and black TODO automatically make noncompact root white

    if show_diagrams:
        _ =latex.eval(dynkin_latex)
    if show_latex:
        print dynkin_latex

    cP = cohomology_poset(
            ct, # to get rid of relabeling, we want this to be labeled by integers
            small_index_set,
            simple_roots_embedding,
            AS.weyl_group(prefix="s"),
            index_set, # not used right now, uniform "projection" on minimal representatives doesn't seem to work (i.e. Enright has a mistake in his paper, we take k-dominant weight in the orbit)
            v,
            with_dynkins=with_dynkins
            )
    bgg_poset = get_poset_latex(cP, orientation)
    if show_diagrams:
        _ = latex.eval(bgg_poset)
    if show_latex:
        print bgg_poset
        #print(latex(cP))

    from string import Template

    scalar_poset_template = Template(r"""
\begin{figure}[H]
  \centering
      $scalar_poset
  \caption{Nonnegative scalar products with noncompact roots}
\end{figure}
    """)
    if sP_latex == "":
        scalar_poset = ""
    else:
        scalar_poset = scalar_poset_template.substitute(scalar_poset=sP_latex)

    template = Template(r"""

\subsubsection{$cartan_type}

Cone of unitarizable weights: $weight \\

$scalar_poset

%\noindent $$\lambda = $$ $weight \\
\noindent Set of singular roots: $singular_roots \\

\begin{figure}[H]
  \centering
  $reduced_dynkin
  \caption{The reduced hermitian symmetric pair $$(\mathfrak{g}_\lambda, \mathfrak{k}_\lambda)$$}
\end{figure}

\begin{figure}[H]
  \centering
      $bgg_poset
  \caption{Nilpotent cohomology / BGG resolution}
\end{figure}

        """)

    if Psi:
        singular_roots_latex = "$\{" + ", ".join(_fix_basis_latex(latex(r)) for r in Psi) + "$\}"
    else:
        singular_roots_latex = "$\emptyset$"
    coefficients = [c for c in to_fundamental_weights(AS, v - AS.rho().to_vector())]
    F = CombinatorialFreeModule(SR, ["omega_{%d}" % i for i in range(1, len(coefficients)+1)], prefix="omega", latex_prefix="\\omega")
    terms = latex(sum(c*F.monomial(i) for (i,c) in enumerate(coefficients, 1)))
    #terms = " + ".join("{c}\\omega_{i}".format(c=c, i=i) for (i, c) in enumerate())
    return  template.substitute(weight="$" + terms.strip() + "$",
                   singular_roots=singular_roots_latex,
                   reduced_dynkin=dynkin_latex,
                   scalar_poset=scalar_poset,
                   bgg_poset=bgg_poset,
                   cartan_type=cartan_type
                  )

\end{minted}