\documentclass[final]{birkmult}
\usepackage{inputenc}
\usepackage{amsmath}
\usepackage{url}
%\usepackage{microtype}

 \newtheorem{theorem}{Theorem}[section]
 \newtheorem{lemma}[theorem]{Lemma} 
 \theoremstyle{definition}
 \newtheorem{definition}[theorem]{Definition}
 \theoremstyle{remark}
 \newtheorem{remark}[theorem]{Remark}
 \numberwithin{equation}{section}

\newcommand{\p}[1]{\partial_{#1}}
\newcommand{\pd}[2]{\frac{\partial #1}{\partial #2}}
\newcommand{\lap}{\Delta}
\newcommand{\alap}{\widetilde{\Delta}}
\DeclareMathOperator{\D}{D}
\DeclareMathOperator{\aD}{\mathcal{D}}
\DeclareMathOperator{\euler}{\mathbb{E}}
\newcommand{\R}{\mathbb{R}^{p,q}}
\newcommand{\aR}{\mathbb{R}^{p+1,q+1}}
\newcommand{\bi}[3][c]{X_{#2} Z^{#3} + Z_{#2} X^{#3} + Y_{#2}^{#1} Y^{#3}_{#1}}
\newcommand{\di}[2]{{\delta_{#1}}^{#2}}

\newcommand{\aSO}{\mathrm{SO}(p+1,q+1)}
\newcommand{\SO}{\mathrm{SO}(p,q)}

\newcommand{\topten}{\circledcirc}
\newcommand{\ffrac}[2]{{\mbox{\large$\frac{#1}{#2}$}}}

\begin{document}
%opening
\title{Construction of conformally invariant differential operators}
\author{V\'it Tu\v{c}ek}
\address{M\'UUK\\ Sokolovsk\'a 49/83\\Praha\\186 00}
\email{vit.tucek@gmail.com}

\begin{abstract}
We present a method of computation of the explicit form of conformally invariant differential operators on $\mathbb{R}^n$ defined using the ambient metric construction. The action of the conformal group on the conformal compactification of $\mathbb{R}^n$ is realised as the action of $\mathrm{SO}(n+1,1)$ on the projectivisation of the null cone in the ambient space $\mathbb{R}^{n+1,1}$. We first review a class of differential operators on the ambient space, which give rise to the conformally invariant differential operators on $\mathbb{R}^n$, and then we show a method how to write down the explicit coefficients of the induced operator by means of a suitable adapted frame on the ambient space. The procedure gives an alternative and direct method how to compute the so called higher symmetry operators for the Laplace equation introduced by M. Eastwood.
\end{abstract}

\thanks{The author was supported by GA\v{C}R 201/09/H012.}
\subjclass{Primary 53A30; Secondary 58J70}
\keywords{symmetry operators, Laplace, conformally invariant operators}

\maketitle

%content

\section{Introduction}
  It is easily checked that the Laplace operator is invariant with respect to the group of Euclidean transformations (rotations \& translations) in the sense that
  \[ \lap (f\circ A) = (\lap f)\circ A,\ A\in \mathrm{Iso}(\mathbb{R}^n) \]
  holds for any smooth function $f$. Another classical result is, that one can use the spherical inversion to `translate' solutions of the Dirichlet problem from the inside of the unit ball to the outside and vice versa. This is based on the fact that for $x' = x/\|x\|^2$ we have 
  \[ \lap_{(x)} u(x) = \|x\|^{2-n}\lap_{(x')}v(x') \]
  where $u(x) =  \|x\|^{2-n}v(x'(x))$ and $\lap_{(x)}$ denotes the Laplacian in coordinates $x$. %CHECK
  The spherical inversion of course doesn't preserve lengths; however it preserves angles which makes it a conformal transformation. The problem is that it is not defined on the whole $\mathbb{R}^n$. To remedy this situation one can either work with pseudogroups or take a suitable conformal compactification. The latter approach, which we will follow in this article, goes in a similar spirit as adding the point at infinity to the complex plane to get the Riemann sphere. 

  Another example of an operator with similar nice properties with respect to the group of conformal transformations is the Dirac operator. Thus one is led naturally to the study of the  conformally invariant (also called covariant) differential operators. 
There is a complete classification of conformally invariant operators on a sphere given by \cite{er} and \cite{Baston}. The articles \cite{be} and \cite{slovk_invariant_1992} contain an extensive summary of the results.  Generalisation to geometrical structures other than conformal is possible and desirable. For example the operators arising in a resolution of the Dirac operator in $k$-variables are invariant with respect to the group $\mathrm{Spin}(n+k,k)$ (see \cite{pf,pf2}).  Another example is the symplectic analogue of the Dirac operator, which belongs to a class of operators treated in \cite{ks} using the representation theoretical results of \cite{ks2}. The classification of invariant operators usually boils down to decompositions of various tensor products of representations into irreducibles under the appropriate structure group. The construction of majority of these operators has been carried out also in the curved setting (\cite{css,cd,es}), but their coefficients are difficult to determine.

In the paper \cite{Eastwood}, M. Eastwood studied higher symmetries of the conformally invariant Laplace operator on the sphere and he constructed them using the ambient construction. In the paper, he also showed that they have curved analogues
and computed their explicit form using various additional tools. The main aim of the article
is to develop methods how to compute the family of these conformally invariant operators
directly from their definition on the ambient space.

  We use the so called abstract index notation introduced by Penrose, which is extremely convenient for performing coordinate-free computations with tensors. In this notation the indices represent the kind of object they're attached to, rather than the coordinates with respect to some basis. The upper indices denote vectors (or vector fields) while lower ones represent one-forms. Repetition of indices denotes contraction and thus one writes the natural pairing between vectors and forms as $x^ay_a$. The round and square brackets around indices stand for the symmetrisation and antisymmetrization respectively. A hat over an index or a symbol implies it's omission in an expression.  In a presence of a metric tensor $g_{ab}$ we lower and raise the indices as usual $x^a = x_b g^{ab}$. For vectors and forms on $\mathbb{R}^n$ we use lower case indices while for the ambient space $\mathbb{R}^{n+2}$ we use the upper case.

  The symbol $\p{a}$ denotes the operator which to any smooth function $f$ assigns the one-form $\p{a} f$ for which $x^a\p{a} f$ is the derivation of $f$ in the direction $x^a$. The Leibniz rule applies and consequently we have $x^a\p{a} (y^b\p{b}f) =x^a y^b \p{a} \p{b} f + x^a(\p{a}y^b) \p{b} f$. In long expressions we use $\p{a_1\cdots a_s}$ as a shorthand for $\p{a_1}\cdots \p{a_s}$.

  In the next section we review the classical  description of the group of conformal transformations on $\R$ and we will use this description in the third section to provide a method for an explicit construction of the conformally invariant differential operators. In the last section we compute as an example the coefficients of the so called higher symmetry operators of the Laplace operator. 

\section{Conformal geometry and the ambient construction}

  There exists two equivalent approaches to conformal geometry in the setting of Riemannian manifolds. One of them uses the rather advanced notion of Cartan geometry modelled on a parabolic pair of Lie groups $(SO(n+1,1),P),$ while the other approach defines the conformal transformation as those diffeomorphisms $\varphi$ of a Riemannian manifold $(M,g)$ which preserve the metric up to a scalar multiple -- i.e. $\varphi^* g = \Omega^2 g$ for some $\Omega\in\mathcal{C}^\infty(M)$ such that $\forall m\in M: \Omega(m) \neq 0$. The conformal class $[g]$ determined by a metric $g$ is then the equivalence class of the relation $(g \simeq \tilde{g} \leftrightarrow \exists \Omega: \tilde{g} = \Omega^2 g)$. The ambient model provides a nice way to easily identify these approaches in the case of the Euclidean space.

  In what follows we will work with a pseudoeuclidean space $\R$ equipped with a symmetric nondegenerate bilinear form $g_{ab}$ of signature $(p,q)$, $p+q=n$. The local conformal transformations are those diffeomorphisms of open sets of $\R$ which preserve angles of curves. The Liouville theorem states (see e.g. \cite{slovk_invariant_1992}) that, in the case of $n \geq 3$,  every local conformal transformation on $\mathbb{R}^n$ is a composition of translations, rotations, dilatations or special conformal transformations\footnote{Special conformal transformation is a generalisation of the circle inversion. It is given by a map $x \mapsto (x-x_0)/{\lVert x-x_0\rVert}^2$.} As a consequence, the conformal group is generated by these four kinds of mappings. The situation for $n=2$ is quite different since one has uncountably many local conformal transformations -- the group generated by these four mappings is then sometimes called the M\"obius group.

  The \emph{ambient space} of $\R$ is the direct sum $\aR = \mathbb{R}\oplus\R \oplus \mathbb{R}$ with non-degenerate symmetric bilinear form $g_{AB}$ defined by
  \begin{equation*}
	 g_{AB}x^Ay^B = x^0y^\infty + x^\infty y^0 + g_{ab} x^a y^b
  \end{equation*}
  for 
  \(
  x^A = \begin{pmatrix}
	x^0, x^a, x^\infty
	\end{pmatrix}^\mathrm{T}\) and \(y^A = \begin{pmatrix}
	y^0,y^a,y^\infty
	\end{pmatrix}^\mathrm{T}\).


  The term ambient will be used when referring to the objects defined on some open subset of $\mathbb{R}^{n+2}$ and  ambient objects will be distinguished by a tilde. For example the ambient Laplace operator is $\alap = g^{AB}\p{A}\p{B}$.

  If we want to lower the index of $x^A,$ we have 
  \[
  x_A = g_{AB}x^B =  \begin{pmatrix}
  x_\infty,
  x_a,
  x_0
  \end{pmatrix} 
  \]
  and as a block matrix the ambient metric takes the form
  \(  g_{AB} = \left(\begin{smallmatrix}
		    0 & 0 & 1 \\ 
		    0 & g_{ab} & 0 \\ 
		    1 & 0 & 0
	      \end{smallmatrix} \right).  \)
  It is a matter of an elementary calculation to show that the signature of the ambient metric is indeed $(p+1,q+1)$.

  Let  $ r = g_{AB}x^Ax^B$ be the quadratic form associated to the ambient metric $g_{AB}$. The \emph{null cone} $\mathcal{N} = \left\{ x \in \aR\, |\, r(x) = 0 \right\}$ is the zero set of $r$. Consider the mapping $\phi: \R \rightarrow \mathcal{N} \subset \aR$ given by %\todo{obrazek}
  \[
  x^a \mapsto \begin{pmatrix}
								  1 \\
								  x^a \\
								  -x^a x_a / 2
						  \end{pmatrix} =: \phi^A.
  \]
  The rays of the null cone intersect the embedded $\mathbb{R}^n$ at exactly one point and we have the following dense subset of $\mathcal{N}$
  \[\mathcal{N}_0= \left\{ \begin{pmatrix} t\\tx^a\\-t\frac{x^ax_a}{2}\end{pmatrix} : t \in \mathbb{R} \setminus\{ 0\}  \right\} \cong \mathbb{R}\setminus\{0\} \times \phi(\mathbb{R}^n) \]
  with an associated projection map $\pi: \mathcal{N}_0 \to \R$ defined as 
  \(\pi: \begin{pmatrix} t\\tx^a\\-t\frac{x^ax_a}{2}\end{pmatrix} \mapsto  x^a.\)

  \begin{lemma}
    The triple $(\phi,\mathcal{N},g_{AB})$ determines the conformal class of $\R$.
   \end{lemma}
  \begin{proof}
    For any smooth nowhere zero function $\Omega$ on $\mathbb{R}^n$ consider the subset of $\mathcal{N}$ given by $\Omega(x^a)\phi(x^a)$. This can be viewed as another embedding of $\mathbb{R}^n$ into the ambient space. The claim is that the ambient metric induces the metric $\Omega^2g_{ab}$ on this embedded $\mathbb{R}^n$.

    The tangent map of this embedding is
    \[
      \partial_a \phi^B = \begin{pmatrix} \partial_a \Omega(x^b)\\(\partial_a \Omega(x^b))x^b + \Omega(x^b){\delta_a}^b\\ -(\partial_a\Omega(x^b))\frac{x^bx_b}{2} - \Omega(x^b)x_a\\ \end{pmatrix}
    \]
    and, denoting $\Omega_a=\p{a}\Omega(x^b)$, the pullback of the ambient metric at a point $x\in \mathbb{R}^n$ is computed as follows
    \begin{align*}
	    g_{AB}\partial_c \phi^A \partial_d \phi^B & = \partial_c \phi^{(0}\partial_d \phi^{\infty)} + g_{ab} \partial_c \phi^a \partial_d \phi^b&\\
						      & = -\Omega_c(\Omega_d \tfrac{x^bx_b}{2} + \Omega x_d) - \Omega_d(\Omega_c \tfrac{x^bx_b}{2} + \Omega x_c) +\\
							    &\quad\; + g_{ab}(\Omega_c x^a + \Omega {\delta_c}^a)(\Omega_d x^b + \Omega {\delta_d}^b)\\
						      & = -\Omega_c\Omega_dx^bx_b - 2\Omega \Omega_{(c}x_{d)} + \Omega^2 g_{cd} + \Omega_c\Omega_dg_{ab}x^ax^b +\\
							    & \quad\; + g_{ab}(\Omega_cx^a\Omega{\delta_d}^b + \Omega{\delta_c}^a\Omega_dx^b)\\
						      & = \Omega^2 g_{cd} - 2\Omega\Omega_{(c}x_{d)} + \Omega_c\Omega g_{ad}x^a + \Omega \Omega_d g_{cb}x^b\\
						      & = \Omega^2 g_{cd}.
	  \end{align*}
    We can conclude that $\Omega\phi$ is isometrical embedding of $(\mathbb{R}^n,\Omega^2 g_{ab})$ into the ambient space and that the ambient metric induces $g_{ab}$ in the case of $\Omega = 1$. 
  \end{proof}
  
  It is obvious that $\mathcal{N}$ is preserved by the defining action of $\aSO$. The fact that is of the most importance to us is that every local conformal transformation of $\R$ is given by this action -- one just multiplies the vector $\phi(x^a)$ by a $\aSO$ matrix and takes the projection $\pi$ of the result. The projectivisation of $\mathcal{N}$ is the conformal compactification of $\R$ and the stabiliser of a line in $\mathcal{N}$ is a parabolic subgroup $P$ of $\aSO$. This identifies the conformal compactification as a flat Cartan geometry of type $(\aSO,P)$. For details see \cite{slovk_invariant_1992}. %TODO zkontrolovat

  Following the idea from \cite{cg} we introduce \emph{the adapted frame} on $\aR$ in order to be able to perform efficient calculations.
  \begin{definition} For $t,\rho \in \mathbb{R}$ and $x^a \in \R$ define three vectors 
  \[
  X^A = \begin{pmatrix}
	  t \\
	  tx^a \\
	  t(\rho-\frac{x^ax_a}{2}) \\
	\end{pmatrix} \quad
  Y_b^A = \partial_b X^A = \begin{pmatrix}
			      0_b \\
			      t\delta_b^a \\
			      -t x_b
			\end{pmatrix} \quad
  Z^A = - \frac{1}{n} \partial^b Y_b^A = \begin{pmatrix}
					      0 \\
					      0^a \\
					      t
				      \end{pmatrix}.
  \]
  \end{definition}
  If we lower the indices via the ambient metric we have
  \[
  X_A = (	t(\rho-\frac{x^ax_a}{2}), tx_a, 1)\quad 
  Y_{bA} =  (-tx_b, tg_{ab}, 0_b)\quad 
  Z_A = (t,0_a,0).
  \]

  One immediately sees that $X^A$ is the embedding $\phi(x^a)$ when $t=1$ and $\rho=0$. 

  \begin{lemma}For $X^A$, $Y^A_b$ and $Z^A$ as above we have
	  \begin{equation}\label{eq:bi}
	  t^2({\delta_A}^B + 2\rho Z_AZ^B)= \bi{A}{B}.
	  \end{equation}
  \end{lemma}

  \begin{proof}
    The equation \eqref{eq:bi} is an analogue of the standard decomposition of the identity mapping on $\mathbb{R}^n$ into the projectors to some orthonormal basis. As such it follows from the following straightforward computations in coordinates.
    \begin{equation} \label{dotproducts}
	  \begin{split}
	      g^{AB} X_A X_B     & = 2 \cdot t^2(\rho-x^a x_a/2) + t^2 g^{ab}x_a x_b = -t^2r + t^2r +2t^2\rho = 2t^2\rho \\
	      g^{AB} X_A Z_B     & = t \cdot t + 0 \cdot (-tx^a x_a/2) + tg^{ab}x_a 0_b = t^2 \\
	      g^{AB} Y_A^c Y_B^d & = 0^d \otimes (-tx^c) + 0^c \otimes (-tx^d) + g^{ab} t\delta_a^c t\delta_b^d = t^2g^{cd} \\
	      g^{AB} X_A Y_B^c   & = (-tx^a x_a/2)\cdot 0^c + (-tx^c)\cdot t + g^{ab}tx_a t \delta_b^c = -t^2x^c + t^2x^c = 0^c \\
	      g^{AB} Z_A Y_B^c   & = t \cdot 0^c + 0 \cdot (-tx^c) + g^{ab} t0_a t\delta_b^c = 0^c\\
	      g^{AB} Z_A Z_B     & = 2 \cdot 0 + g^{ab} 0_a 0_b = 0,
	  \end{split}
    \end{equation}
  \end{proof}

  If we differentiate vector field $X^A$ with respect to the real parameter $t$ we get $\partial_t X^A = \phi(x^a)$. Because $Y_c^BY_B^d = t^2 \delta_c^d$, we only need to compute $\partial_\rho X^A = Z^A$ to see that the formula for $X^A$ defines in fact a change of coordinates on the open half-space $\{t > 0\}$ of $\aR$. %Consequently we can interpret the symbol $X^A$ as another notation for $x^A$. 
  Consequently, the symbols $X^A$ and $x^A$ represent the same object -- the identity vector field on $\aR$. Let's explicitly define the mapping of the coordinate change:
  \begin{equation*}
	  \Phi(t,x^a,\rho) = \begin{pmatrix}t \\ tx^a \\	t(\rho-\frac{x^ax_a}{2}) \\ \end{pmatrix} = \begin{pmatrix} y^0\\ y^a\\ y^\infty \\ \end{pmatrix}.
  \end{equation*}
  We see that  $\phi(x^a) = \Phi(1,x^a,0)$ and the identity \eqref{eq:bi} simplifies on the image of $\phi$ to 
  \[
  {\delta_A}^B = \bi{A}{B}.
  \]
  This identity will be of a great use later on.
\pagebreak[2]
  \begin{lemma}\label{lem:euler_change}
    The Euler operator in the new coordinates is equal to $\mathbb{E} = t\p{t}$.
  \end{lemma}

  \begin{proof} For $f(y^A) \in \mathcal{C}^\infty(\mathbb{R}^n)$ we have
    \begin{align*}
      t \p{t} f (y^A(t,x^a,\rho)) & = t( \pd{f}{y^0} + x^a \pd{f}{y^a} +
      (\rho-\frac{x^bx_b}{2})\pd{f}{y^\infty}  ) \\
	      & = y^0( \pd{f}{y^0} + \frac{y^a}{y^0} \pd{f}{y^a} +
      \frac{y^\infty}{y^0}\pd{f}{y^\infty}  )\\
	      & = (y^A\p{A} f) \circ \Phi(t,x^a,\rho)
    \end{align*}
  \end{proof}

Since we want to deal with differential operators we need to incorporate smooth functions on $\R$ into the picture. Suppose that $f$ is a smooth function defined on the neighbourhood of  origin in $\R$. Then for any $w \in \mathbb{C}$
\begin{equation}
	\tilde{f}(\Phi(t,x^a,0)) = t^w f(x^a)  
	\label{eq:extenze}
\end{equation}
defines a smooth function on a `conical neighbourhood' of $(1,0,0)$ inside the null cone $\mathcal{N}$. Moreover it is a homogeneous function of degree $w$ because $\tilde{f}(\lambda y^A) = \lambda^w \tilde{f}(y^A)$ for $\lambda > 0$. Conversely $f$ may be recovered from $\tilde{f}$ by setting $t = 1$. In order to be able to apply ambient differential operators to $\tilde{f}$ we need to extend it from the null cone to the whole space or at least to some open (in $\aR$) neighbourhood of $(1,0,0)$. We will call any such extension \emph{ambient extension}. There are infinitely many choices for such an extension even if we stick to the homogeneous ones. Nevertheless, any two such extensions will differ by a very convenient factor.

\begin{lemma}
	Let $\tilde{f}$ and $\hat{f}$ be two smooth $w$-homogeneous extensions of $f$ on some open neighbourhood of $(1,0,0)$ not containing zero. Then there exist a smooth $(w-2)$-homogeneous function $h$ such that $(\tilde{f}-\hat{f})(y^A) = r(y^A)h(y^A)$ where $r$ is the defining quadric of the null cone.
\end{lemma}
\begin{proof}
  For any smooth function $k$ on $\aR$ holds 
	\begin{equation*}
		k(t,x^a,\rho) = k(t,x^a,0) + \int_0^1 \frac{d}{ds} k(t,x^a,s\rho) \mathrm{d}s
				     = k(t,x^a,0) + \rho \int_0^1 \frac{\partial k}{\partial \rho}(t,x^a,s\rho) \mathrm{d}s.	
	\end{equation*}
  According to the first equation of \eqref{dotproducts} we have $\rho =r/2ti^2$. For $k=(\tilde{f}-\hat{f})\circ\Phi$ we have $k(t,x^a,0) = 0$ and the result follows by substitution with $\Phi^{-1}$. 
\end{proof}

\begin{remark}
	The classical chain rule formula, with regard to \eqref{eq:extenze}, gives 
	\begin{equation}\label{eq:chain}
		\p{a} f = \p{a} (\tilde{f} \circ \phi) = (\p{a}\phi^B)(\p{B}\tilde{f})\circ \phi = (Y_{a}^{B}\p{B}\tilde{f})\circ \phi
	\end{equation}
	for any ambient extension $\tilde{f}$ of $f$ because $\p{a}\phi^B$ equals $Y_{a}^{B}$ on the image of $\phi$.
  Using this expression for $f = Y_b^A\circ\phi$ we get 
\begin{equation}\label{eq:derivace_ypsilonu}
 (-tg_{cb}Z^A) \circ \phi =\p{c}(Y_b^A\circ \phi) =  (Y_{c}^{D} \p{D}Y^A_b)\circ \phi.
\end{equation}
\end{remark}

\section{Construction of conformally invariant differential operators}

 Let $(\mathbb{V}_1,\varrho_1), (\mathbb{V}_2,\varrho_2)$ be two representations of a Lie group $G$ and let $G$ have a smooth action on a manifold $M$. One defines the induced action of $G$ on smooth functions $\mathcal{C}^\infty(M,\mathbb{V}_i)$ by $(g\cdot f)(x) = \varrho_i(g)(f(g^{-1})x)$. The $G$-invariant differential operators are defined as those differential operators which are equivariant with respect to the induced action of $G$. 

 In our case, the Lie group in question is $\aSO$ and the underlying manifold is $\mathcal{N}$. As a consequence, we may find the conformally invariant operators among the orthogonally invariant ones. An ambient differential operator $\tilde{D}$ induces an operator on $\mathbb{R}^n$ if and only if the value of $(\tilde{D}\tilde{f})\circ \phi$ doesn't depend on the ambient extension $\tilde{f}$ of $f$ (i.e. $\tilde{D}\tilde{f} = \tilde{D}\hat{f})$. For a linear operator, this condition is equivalent to $\tilde{D}(\tilde{f}-\hat{f}) = \tilde{D} (rh) = 0$. Of course such a condition can hold only for some weights $w$ of the extension and not for the other weights. The conclusion is that a conformally invariant linear differential operator is induced by a $\aSO$-invariant operator $\tilde{D}$ for which there exists a weight $w\in\mathbb{C}$ such that
\[[\tilde{D},r]\tilde{f}= \tilde{D}r\tilde{f} - r \tilde{D}\tilde{f} = 0.\] %TODO dostanu tak vsechny?

 Suppose we are given such an operator $\tilde{D}$ with leading term $S^{A_1\cdots A_i}(x^A) \p{A_1\cdots A_i}$. In order to find the coefficients of the induced operator we use the adapted frame \eqref{eq:bi} and write $\tilde{D}=S^{A_1\cdots A_i}(x^A) \di{A_1}{B_1}\cdots\di{A_i}{B_i}\p{B_1}\cdots\p{B_i} + \text{LOT}$ which on the image of $\phi$ simplifies to 
 \[S^{A_1\cdots A_i}(x^a) (\bi{A_1}{B_1})\cdots(\bi{A_i}{B_i})\p{B_1 \cdots B_i},\]
 since we already assume that the result doesn't depend on the ambient extension and hence we can drop all the terms containing $\varrho$. Also the terms containing $Z^{B_i}$ can be omitted as well, because they represent derivatives in the direction transversal to the null cone and the result doesn't depend on the ambient extension.

 Let us compute the expression for the operator induced by the ambient Laplace operator as an illustration of this method,
  \begin{align*}
  \left(g^{AB} \p{A}\p{B} \tilde{f}\right)\circ \phi  & =  \left(g^{AB} {\delta_A}^C {\delta_B}^D \p{C} \p{D} \tilde{f}\right)\circ \phi  &\\
		  & =  \left[g^{AB} \frac{1}{t^2}(\bi[q]{A}{C}-\rho {U_A}^C)\right.\cdot&\\ 
		  &\quad \cdot \left.\frac{1}{t^2}(\bi[r]{B}{D}-\rho {U_B}^D) \p{C} \p{D}\tilde{f}\right]\circ \phi &\\
		  &	=  [(Z^C X^D + X^C Z^D) \p{C} \p{D}f + g^{qr} Y_q^C Y_r^D \p{C} \p{D}\tilde{f}]\circ \phi. &\text{by \eqref{dotproducts}}
  \end{align*}
 Because we have $[\p{A},\p{B}]=0$, the first term in the last expression equals
	\begin{align*}
		2Z^CX^D\p{C}\p{D} & = 2\left(Z^C\p{C}X^{D}\p{D}-Z^{C}(\p{C}X^{D})\p{D}\right) \\
				& = 2\left(Z^C\p{C}X^{D}\p{D}-Z^{C}(\di{C}{D})\p{D}\right) \\
				& = 2Z^C\p{C}(\mathbb{E} - 1)
	\end{align*}
  applied to $\tilde{f}$ and evaluated on the image of $\phi$. The second term is
	\begin{align*}
		\big(g^{qr} Y_q^C Y_r^D \p{C} \p{D} \tilde{f}\big)\circ \phi 
			& = \big(g^{qr} [Y_q^C \p{C} Y_r^D  \p{D} - Y_q^C( \p{C} Y_r^D)  \p{D}]\tilde{f}\big)\circ \phi  &\\
			& = \big(g^{qr} Y_q^C \p{C} (Y_r^D\p{D}\tilde{f})\big)\circ \phi  + \big(g^{qr}g_{qr}Z^D\p{D}\tilde{f}\big)\circ \phi & \text{by \eqref{eq:derivace_ypsilonu}}\\
			& = \lap f + n(Z^D\p{D}\tilde{f})\circ \phi.&
	\end{align*}
  Hence 
    \[ (\alap \tilde{f}) \circ \phi = \lap f + \left(Z^D\p{D} (n+2\mathbb{E} - 2)\tilde{f}\right)\circ \phi\]
  and we see that, for $w=1-n/2$, the ambient Laplace operator on $\aR$ induces the Laplace operator on $\R$.

  We see that the computation of the coefficients boils down to two parts -- calculation of the contractions with the symbol of the operator and calculation of the contractions with the differentials. For the latter part we can record here the following lemma.  

  \begin{lemma}\label{lem:derivacni_cast}
      Let $D(k,s)$ be an operator defined as
    \[D(k,s) =X^{D_1}\cdots X^{D_k}Y_{c_{k+1}}^{D_{k+1}}\cdots Y_{c_s}^{D_s} \p{D_1} \cdots \p{D_s} .\]
    Then modulo the terms depending on the ambient extension we have for $w$-homogeneous functions
 \[(D(k,s) \tilde{f} )\circ \phi = \left[ \prod_{i=1}^{k}(w-s+i) \right] \p{c_{k+1}}\cdots \p{c_s} f.\]
  \end{lemma}
  \begin{proof}
    Let $T(k)$ denote the operator $X^{D_1}\cdots X^{D_k}\p{D_1}\cdots \p{D_k}$. We have	
    \begin{align*}
	    T(k) & = X^{D_1} \cdots X^{D_{k-2}} X^{D_{k-1}}X^{D_k}\p{D_1}\cdots \p{D_k}\\
		& = X^{D_1} \cdots X^{D_{k-2}} X^{D_{k-1}} (\p{D_1} X^{D_k} - \di{D_1}{D_k} ) \p{D_2} \cdots \p{D_k}  \\
		& = X^{D_1} \cdots X^{D_{k-2}} X^{D_{k-1}} \p{D_1} X^{D_k} \p{D_2} \cdots \p{D_k} f - T(k-1) \\
		& = X^{D_1} \cdots X^{D_{k-2}} (\p{D_1} X^{D_{k-1}} - \di{D_1}{D_{k-1}}) X^{D_k} \p{D_2} \cdots \p{D_k} - T(k-1) \\
		& = X^{D_1} \cdots X^{D_{k-2}} \p{D_1} X^{D_{k-1}} X^{D_k} \p{D_2} \cdots \p{D_k}- 2 T(k-1) \\
		%& \mspace{54mu} \vdots \\
		& \qquad \  \vdots \\
		& = X^{D_1} \p{D_1} X^{D_2}\cdots X^{D_k} \p{D_2} \cdots \p{D_k} - (k-1) T(k-1) \\
		& = \euler T(k-1) - (k-1)T(k-1) = (\euler -k+1)T(k-1)
    \end{align*}
    Since $T(1) = X^{D_1}\p{D_1} = \euler$ we see that
    \[ T(k)= X^{D_1} \cdots X^{D_k}\p{D_1}\cdots \p{D_k}  = (\euler -k+1)(\euler - k +2)\cdots (\euler -1) \euler .\]

    We can view $Y_b^A$ as a $1$-homogeneous function on $\mathbb{R}^n$ with values in $\mathbb{R}^{n+2}$ because the homogeneity in the standard coordinates translates to homogeneity in $t$ by the lemma \eqref{lem:euler_change}. Therefore the Euler operator acts as the identity on $Y_{c}^{D}$ which implies $[\euler,Y_{c_i}^{D_i}] = 0$. Using this fact we can write 
\( D(k,s) =T(k) Y_{c_{k+1}}^{D_{k+1}}\cdots Y_{c_s}^{D_s}\p{D_{k+1} \cdots  D_s}\).

Iterating the formula \eqref{eq:chain} we get 
    \begin{equation*}
	\p{c_1} \cdots \p{c_k} f = (Y_{c_1}^{D_1}\p{D_1}(Y_{c_2}^{D_2}\p{D_2}(\cdots \p{D_{k-1}}(Y_{c_k}^{D_k}\p{D_k}\tilde{f}))\cdots)\circ\phi.
    \end{equation*}
  Since $Y_{c}^{D} \p{D}Y^A_b =  -tg_{cb}Z^A$ by  \eqref{eq:derivace_ypsilonu}, we see that the difference between the above expression and the formula $Y_{c_1}^{D_1}\cdots Y_{c_k}^{D_k} \p{c_1}\cdots\p{c_k}$ yields a differentiation of $f$ in the direction transversal to the embedding $\phi$ that clearly depends on the choice of an ambient extension.

Since each differentiation lowers the homogeneity by one, $T(k)$ acts in the expression for $(D(k,s)\tilde{f})\circ\phi$ on $w-(s-k)$-homogeneous function and the result follows.
  \end{proof}

\section{Symmetry operators of the Laplace equation}
  
  As an application of the just presented method we compute the so called higher symmetry operators for the Laplace equation. We say that a linear differential operator $D$ is symmetry operator of the Laplace equation if there exists another linear differential operator $\delta$ such that $\lap D=\delta\lap$. It is easy to see that these operators preserve the space of harmonic functions. It was shown in \cite{Eastwood} that, modulo the trivial symmetry operators of the form $D\lap$, all the symmetry operators are induced from the ambient operators of the form $\aD_V := V^{A_1 B_1 \cdots  A_s B_s} X_{A_1} \cdots X_{A_s} \p{B_1} \cdots \p{B_s}$ where
 \[ \textstyle V^{A_1 B_1 \cdots  A_s B_s}\in\bigotimes^{2s}{\mathbb R}^{n+2} \]
  is a tensor that is skew in each pair of indices $A_iB_i$, is totally trace-free, and such that skewing over any three indices gives zero. It follows that $V^{A_1 B_1 \cdots  A_s B_s}$ is symmetric with respect to a change of the form $A_iB_i \leftrightarrow A_jB_j$ and that symmetrising over any $s+1$ indices gives zero. These symmetries can be expressed by a Young tableau
  \[
    \underbrace{\begin{picture}(100,30)
    \put(0,5){\line(1,0){100}}
    \put(0,15){\line(1,0){100}}
    \put(0,25){\line(1,0){100}}
    \put(0,5){\line(0,1){20}}
    \put(10,5){\line(0,1){20}}
    \put(20,5){\line(0,1){20}}
    \put(30,5){\line(0,1){20}}
    \put(40,5){\line(0,1){20}}
    \put(70,5){\line(0,1){20}}
    \put(80,5){\line(0,1){20}}
    \put(90,5){\line(0,1){20}}
    \put(100,5){\line(0,1){20}}
    \put(55,10){\makebox(0,0){$\cdots$}}
    \put(55,20){\makebox(0,0){$\cdots$}}
    \put(105,7){\makebox(0,0)[l]{\scriptsize trace-free part.}}
    \end{picture}}_{\text{$s$ boxes in each row}}
  \]
  
  From these symmetry properties it is easy to prove that $\aD_V$ commutes with $r$ and with $\alap$. Since it also preserves homogeneity it follows that the operator induced on $(1-n)/2$-homogeneous functions is a symmetry operator with $\delta$ being the operator induced on $(-3-n)/2$-homogeneous functions. 

\begin{theorem}
        Let $V^{A_1 B_1 \cdots  A_s B_s}$ be a tensor with the aforementioned symmetries and let \[V^{c_1 \cdots c_s} = \left ( V^{A_1 B_1 \cdots  A_s B_s} X_{A_1} \cdots X_{A_s} Y_{B_1}^{c_1} \cdots Y_{B_s}^{c_s}\right ) \circ \phi.\] Let $f$ be a smooth function on $\mathbb{R}^n$ and let $\tilde{f}$ be its $w$\nobreakdash-homogeneous extension on some open neighbourhood of $\phi(\mathbb{R}^n)$. The operator on $\mathbb{R}^n$ defined by
\(\D^w_V f = (\aD_V \tilde{f})\circ \phi \)
equals to
        \begin{equation}\label{eq:res}
                \D^w_V f = \sum_{k=0}^s  (-1)^k{\textstyle \binom{s}{k}\left( \prod_{i=1}^k\frac{w-s+i}{n+2s-1-i}\right)} (\partial_{c_1}\cdots\partial_{c_k}V^{c_1 \cdots c_s})\partial_{c_{k+1}}\cdots \partial_{c_s}  f.
        \end{equation}
\end{theorem}
\begin{proof}
  We apply the method described in the previous section and discard the terms containing $\varrho$ and $Z^{D_i}$. We arrive at the following equality on the image of $\phi$
 \[\aD_V \tilde{f} = V^{A_1 B_1 \cdots  A_s B_s} X_{A_1} \cdots X_{A_s} \sum_{\substack{I,J \subseteq \{1,\ldots s\}\\ I\cap J=\emptyset}} \left(\prod_{i\in I}Z_{B_i}X^{D_i} \prod_{j \in J} Y_{B_j}^{c_j}Y_{c_j}^{D_j}\right) \p{D_1 \cdots D_s} \tilde{f}.\]

  Because the tensor $X_{A_1} \cdots X_{A_s}$ is symmetric and $V^{A_1 B_1 \cdots  A_s B_s}$ is symmetric in pairs $A_iB_i$, we can write
	\begin{multline*}
		V^{A_1 B_1 \cdots  A_s B_s} X_{A_1} \cdots X_{A_s} \left(\prod_{i\in I}Z_{B_i}X^{D_i} \prod_{j \in J} Y_{B_j}^{c_j}Y_{c_j}^{D_j}\right) = \\
		= V^{A_1 B_1 \cdots  A_s B_s} X_{A_1} \cdots X_{A_s} \left(\prod_{i=1}^kZ_{B_i}X^{D_i} \prod_{j =k+1}^s Y_{B_j}^{c_j}Y_{c_j}^{D_j}\right) 
	\end{multline*}
	for any two disjoint subsets $I,J$ of $\{1,\ldots,s\}$ whose union is the whole set and where $I$ has cardinality $k$. For brevity we introduce the symbols $X_{A_1\cdots A_s}$, $ Y_{B_1\cdots B_s}^{c_1\cdots c_s}$, $Z_{B_1\cdots B_s}$ as shorthands for $X_{A_1} \cdots X_{A_s}$ etc. 

 So far, we've got the following expression for $\aD_V$ on the image of $\phi$
	\[
	  \sum_{k=0}^s \binom{s}{k} V^{A_1 B_1 \cdots  A_s B_s} X_{A_1 \cdots A_s} Z_{B_1\cdots B_k}  Y_{B_{k+1}\cdots B_s}^{c_{k+1}\cdots c_s} X^{D_1\cdots D_k}Y_{c_{k+1}\cdots c_s}^{D_{k+1}\cdots D_s} \p{D_1} \cdots \p{D_s} \tilde{f},
	\]
	where it is understood that for $k=0$ the term under the sum equals to 
	\[
	V^{A_1 B_1 \cdots  A_s B_s} X_{A_1 \cdots A_s}  Y_{B_{1}}^{c_{1}}\cdots Y_{B_s}^{c_s}X^{D_1}\cdots{X^{D_k}}Y_{c_{k+1}}^{D_{k+1}}\cdots Y_{c_s}^{D_s} \p{D_1} \cdots \p{D_s} f.
	\] and analogously there are only `$Z$ terms' for $k=s$.

Let $S(k) = V^{A_1 B_1 \cdots  A_s B_s} X_{A_1 \cdots A_s} Z_{B_1\cdots B_k}  Y_{B_{k+1}\cdots B_s}^{c_{k+1}\cdots c_s}$ be the symbol part of the operator. Using the chain rule and the Leibniz rule we get
\begin{align*}
	\p{c_{k+1}}(S(k)\circ \phi) & = (Y_{c_{k+1}}^D\p{D}S(k))\circ \phi\\
	& = \Bigl[ V^{A_1 B_1 \cdots  A_s B_s} Y_{c_{k+1}}^D \bigl(\p{D}(X_{A_1 \cdots  A_s})  Z_{B_1\cdots B_k} Y_{B_{k+1}\cdots B_s}^{c_{k+1}\cdots c_s}+\\
      &\quad  + X_{A_1 \cdots  A_s}Z_{B_1\cdots B_k} \p{D}(  Y_{B_{k+1}\cdots B_s}^{c_{k+1}\cdots c_s})\bigr) \Bigr] \circ \phi,
\end{align*}
because $Y_a^D\p{D}Z_B$ equals zero by a straightforward computation.

Using the identity $Y_{c_1}^D\p{D} X_{A_i} = Y_{c_1A_i}$  and the Leibniz rule again we write the first summand as
\[
 \sum_{i=1}^s (V^{A_1 B_1 \cdots  A_s B_s} X_{A_1 \cdots \widehat{A_i} \cdots  A_s} (Y_{c_1}^D\p{D} X_{A_i}) Z_{B_1\cdots B_k} Y_{B_{k+1}\cdots B_s}^{c_{k+1}\cdots c_s}) \circ \phi.
\] 
Plugging in the identity $Y_{c_1}^D\p{D} X_{A_i} = Y_{c_1A_i}$ we can simplify it to 
\[
  \sum_{i=1}^s\left(  V^{A_1 B_1 \cdots  A_s B_s} X_{A_1 \cdots \widehat{A_i} \cdots  A_s}(g_{A_iB_1} - X_{A_i}Z_{B_1} - Z_{A_i}X_{B_1}) Z_{B_1\cdots B_k} Y_{B_{k+1}\cdots B_s}^{c_{k+1}\cdots c_s}\right)\circ \phi,
\]
because $Y_{c_1A_i}Y^{c_1}_{B_1}$ equals to $g_{A_iB_1} - X_{A_i}Z_{B_1} - Z_{A_i}X_{B_1}$ on the image of $\phi$ due to the \eqref{eq:bi}. Using trace-freeness of $V^{A_1 B_1 \cdots  A_s B_s}$ and its antisymmetry in $A_1B_1$, we see that for $i\neq1$ the only nontrivial contraction is with the term $X_{A_i}Z_{B_1}$; for $i=1$ we contract $V$ with the tensor field $Z_{A_1}X_{B_1}+X_{A_1}Z_{B_1}$ which is symmetric in its indices. Therefore the first summand  equals to $-(s-1)S(k+1)$.

With the help of the identity \eqref{eq:derivace_ypsilonu}, the second summand yields
\[
	\sum_{i=k+1}^s \left( V^{A_1 B_1 \cdots  A_s B_s} X_{A_1 \cdots  A_s} Z_{B_1\cdots B_k} Y_{B_{k+1}}^{c_{k+1}}\cdots \widehat{Y_{B_i}^{c_i}} \cdots Y_{B_s}^{c_s} (- \di{c_1}{c_i} Z_{B_i}) \right) \circ \phi.
\]
For $i=1$ the contraction results in $-nS(k+1)$ whereas for $i \neq 1$ we get $-S(k+1)$ because of the symmetry in pairs $A_1B_1 \leftrightarrow A_iB_i$. Thus the second summand equals to $-(n+s-k-1)S(k+1)$. Putting it all together we arrive at
\begin{equation*}
	 \p{c_{k+1}}(S(k)\circ \phi)  = - (n+2s-2-k) S(k+1) \circ \phi.
\end{equation*}

Since $S(0) \circ \phi = V^{c_1\cdots c_s}$, we see that 
\[
 S(k+1) = \frac{(-1)^{k+1}}{(n+2s-2)\cdots(n+2s-2-k)} \p{c_{k+1}}\cdots \p{c_1}V^{c_1\cdots c_s}
\] and using the lemma \ref{lem:derivacni_cast} we finally conclude 
\begin{align*}
	\D^w_V f & = (\aD_V \tilde{f}) \circ \phi \\
		      & = \left(\sum_{k=0}^{s} \binom{s}{k} S(k)D(k,s)\tilde{f}\right)\circ \phi \\
				& = \sum_{k=0}^{s} \binom{s}{k} S(k)\circ \phi \cdot (D(k,s) \tilde{f})\circ \phi  \\
				& = \sum_{k=0}^{s} \binom{s}{k} A(k,s,w)  ( \p{c_k}\cdots \p{c_1}V^{c_1\cdots c_k}) \p{c_{k+1}}\cdots \p{c_s} f
\end{align*}
where
\[
A(k,s,w) = (-1)^k \frac{ (w-s+1)\cdots(w-s+k) }{(n+2s-2)\cdots(n+2s-1-k)}.
\]
\end{proof}

\begin{remark}
 The formula \eqref{eq:res} agrees with the formula on the page 1659 of \cite{Eastwood}, where the author uses a rather sophisticated notion of naturality in order to obtain the result.
\end{remark}


%biblio
\begin{thebibliography}{99}
\bibitem{Baston} Baston, R.~J., Conformally Invariant Operators: Singular Cases, Bull. London Math. Soc., 23 (1991), 153-159
\bibitem{be} Baston, R.~J., Eastwood, M.~G., Invariant operators,  Twistors in mathematics and physics,  129--163, London Math. Soc. Lecture Note Ser., 156, Cambridge Univ. Press, Cambridge, 1990.
\bibitem{cd} Calderbank, D., Diemmer, T., Differential invariants and curved Bernstein-Gelfand-Gelfand sequences, J. Reine Angew. Math. 537 (2001), 67--103.
\bibitem{css} \v{C}ap, A., Slov\'ak, J., Sou\v cek, V., Bernstein-Gelfand-Gelfand sequences, Ann.Math., 154 (2001), 97--113. 
\bibitem{cg} \v{C}ap, A., Gover, A.~R., Standard tractors and the conformal ambient metric construction, Ann. Global Anal. Geom. 24 (2003), no. 3, 231--259.
\bibitem{Eastwood}  Eastwood, M.~G.,  Higher symmetries of the Laplacian, Annals Math.\  {\bf 161}, 1645 (2005) [arXiv:hep-th/0206233].
\bibitem{er}  Eastwood, M.~G.,  Rice, J.~W., Conformally invariant differential operators on Minkowski space and their curved analogues,  Comm. Math. Phys. Volume 109, Number 2 (1987), 207-228.
\bibitem{es}  Eastwood, M.~G., Slov\'ak, J., Semiholonomic Verma modules. Journal of Algebra,  vol. 197 (1997), no. 2, 424--448.
\bibitem{pf} Franek, P., Dirac operator in two variables from the viewpoint of parabolic geometry, Adv. Appl. Clifford Algebr. 17 (2007), no. 3, 469--480.
\bibitem{pf2} Franek, P., Generalized Dolbeault sequences in parabolic geometry, J. Lie Theory 18 (2008), no. 4, 757--774.
\bibitem{ks} Kr\'ysl, S., Classification of 1st order symplectic spinor operators over contact projective geometries, Differential Geom. Appl. 26 (2008), no. 5, 553--565.
\bibitem{ks2} Kr\'ysl, S., Decomposition of a tensor product of a higher symplectic spinor module and the defining representation of ${\mathfrak{sp}}(2n,\mathbb{C})$, J. Lie Theory 17 (2007), no. 1, 63--72.
\bibitem{slovk_invariant_1992} Slov\'ak, J. , Invariant operators on conformal manifolds, Research Lecture Notes, University of Vienna, 138 pages, 1992. (extended and revised version submitted as habilitation thesis at Masaryk University in Brno, 1993) \\
 available online at \url{http://www.math.muni.cz/~slovak/ftp/papers/vienna.ps}
\end{thebibliography}
\end{document}
