\chapter{Invariant differential operators}
\section{Homomorphisms of parabolic Verma modules}

\section{Unitarizable highest weight modules and invariant differential operators}

Let us change a notation a little bit for this section and let $\alpha$ denote multiindices of integers.

Let $P(V,W)$ denote the space of polynomials between two complex vector spaces $V$ and $W$ which are endowed with a Hermitian inner product. For $p\in P(V,W)$ define $p(\partial):P(V,W)\to P(V,\C)$ by equality $p(\partial) e^{(s|t)_V} = p(\overline{s})e^{(s|t)_V}$. In coordinates we get that for $p(t) = \sum_{\alpha} a_{\alpha} t^{\alpha}$ the resulting linear differential operator has the expression $p(\partial)= \sum_{\alpha} \overline{a_{\alpha}} t^{\alpha}$. The \emph{Fischer inner product} on $P(V,W)$ is defined by $\langle p,q\rangle := (q(\partial)p)(0)$. Explicitly in coordinates we have for  $p(t) = \sum_{\alpha} a_{\alpha} t^{\alpha}$ and  $q(t) = \sum_{\alpha} b_{\alpha} t^{\alpha}$ that $\langle p,q\rangle = \sum_{\alpha} \alpha! (a_{\alpha},b_{\alpha})_W$.

\begin{lemma}
 Let $p,q\in P(V,W)$ and let $f\in P(V,\C)$. Then
 \[
  \langle p,fq\rangle = \langle f(\partial)p, q \rangle = \langle q(\partial)p,f \rangle.
 \]
\end{lemma}
\begin{proof}
 All three expression are euqal to $((fq)(\partial)p)(0)$.
\end{proof}

As was argued in \cite{davidson_differential_1991} this inner product can be used to define a nondegenerate pairing between $M(\lambda)$ and its conjugate dual $M(\lambda)^*$. In this duality one gets that $J(\lambda)$ is orthogonal to $L(\lambda)$.

\begin{theorem}[Theorem 2.9 of \cite{davidson_differential_1991}]
 Let $m_1,\ldots,m_k$ be any set of generators for $J(\lambda)$ as an $S(\lie{p}_+)$-module. Then the simple submodule $L(\lambda)$ of the conjugate dual $M(\lambda)^*$ is the kernel of the constant coefficient operators $m_i(\partial)$, $1\leq i\leq t.$
\end{theorem}
\begin{proof}
 A polynomial $f$ is in $L$ if and only if $\langle f, p m_i \rangle = 0$ for all $p\in P(\lie{p}_-,\C) = S(\lie{p}_+)$, $1\leq i\leq t$. According to the previous lemma this is equivalent to $p(\partial)  m_i(\partial)f)(0)$ for all relevant $p$ and $i$. From the polynomiality of $f$ follows that $m_i(\partial)f = 0$ for all $1\leq i\leq t.$
\end{proof}

\section{F-method}
\section{Scalar valued invariant differential operators for classical hermitian symmetric spaces}